%This template consists of the minimum of a single book.
%Please do not think this template is mandatory and the format must be followed strictly.
%We expect the author adds what he needs.
\documentclass[oneside,11pt,pdftex]{book}%Remove draft when book editing is completed.
\usepackage{graphicx}
\usepackage{amsmath}
%\usepackage{fontawesome5}
\usepackage{booktabs}
\usepackage{amssymb}	
\usepackage{longtable}
\usepackage{amsthm}
\usepackage{multirow}
\usepackage[activate={true,nocompatibility},final,tracking=true,kerning=true,spacing=true,factor=1100,stretch=10,shrink=10]{microtype}
\usepackage[toc,page]{appendix}
\usepackage[nottoc]{tocbibind}
\numberwithin{equation}{section}
\graphicspath{ {./Images/} }
%\usepackage[raggedright]{titlesec}
\usepackage{placeins}
\usepackage{mathtools}
\usepackage{tikz}

\usepackage{fancyhdr}
\usepackage{hyperref}
%Be careful when you use commands which align formulas.
%If aligned formulas range to two pages, the formulas should be divided into two environments.
%\makeatletter
%\AtBeginDocument{\let\mathaccentV\AMS@mathaccentV}
%\makeatother
%This is a patch for double bar.
%Activate it if \bar{\bar{a}} doesn't work.

\newskip\thskip
\thskip=0.5\baselineskip plus 0.2\baselineskip minus 0.2\baselineskip

\newdimen\dtest%Remove this when book editing is completed.
\settowidth{\dtest}{letters and symbols here}
\typeout{<<<\the\dtest>>>}

\newtheorem{theorem}{Theorem}[chapter]%Modify these declarations for your need.
\newtheorem{lemma}[theorem]{Lemma}
\newtheorem{corollary}[theorem]{Corollary}
\newtheorem{example}[theorem]{Example}
\newtheorem{definition}[theorem]{Definition}

\newtheorem{xca}[theorem]{Exercise}

\newtheorem{remark}[theorem]{Remark}

\numberwithin{section}{chapter}
\numberwithin{equation}{chapter}

\makeindex

\newcommand{\R}{\mathbb{R}}
\newcommand{\Q}{\mathbb{Q}}
\newcommand{\C}{\mathbb{C}}
\newcommand{\Z}{\mathbb{Z}}
\newcommand{\N}{\mathbb{N}}
\newcommand{\D}{\mathbb{D}}
\newcommand{\F}{\mathbb{F}}

\graphicspath{ {graphics/} }

\begin{document}


\frontmatter

\thispagestyle{empty}
\begin{flushright}
{\LARGE \textbf{Bhoris Dhanjal}}%Input your name here.
\end{flushright}
\vfill
\begin{center}
{\fontsize{29.86truept}{0truept}\selectfont \textbf{Industrial statistics}}%Input the book title here.
%Below is for a book with a subtitle.
%{\fontsize{29.86truept}{0truept}\selectfont \textbf{The Book Title}} \\
%\vspace{6.5truept}
%{\Large, \LARGE, etc. \textbf{The Subtitle}}
\end{center}
\vfill
\begin{flushleft}
{\LARGE \textbf{Lecture Notes}} \\
\hspace{-1.75truept}
{\large \textbf{for SSTA403}}
\end{flushleft}
\newpage

\tableofcontents


\mainmatter

\chapter{CPM and PERT}
\section{Network analysis}
Large and complex projects of any organization involve a number of interrelated activities with limited resources such as labour, machines, material, money and time. It is not possible for the management to find and execute an optimum solution just by intuition and work experience.
\par 
The main objective of project management can hence be described in terms of successful completion of a project within time, budgeted cost and t the technical specifications/ It also becomes essential to incorporate any change in the initial plan and immediately know the effect of the change from time to time.
\par
\begin{definition}[Network analysis]\label{def:netanal}
	The statistical tool develop to help in planning, managing and controlling the project is known as \textbf{network analysis.}
\end{definition}

\par Thus network analysis deals with dividing the given project into smaller well-defined tasks and studying the interdependencies of these tasks. When the interdependencies of these tasks are best represented, they form a \textbf{network}; and hence the name \textbf{network analysis.} It involves three main steps:
\begin{enumerate}
	\item It defines the jobs to be done.
	\item It integrates the jobs in a logical time sequence.
	\item It controls the progress of the project plan.
\end{enumerate}
\textbf{When all the tasks are accomplished, the project is said to be completed.}
\par

Project scheduling consists of four main steps:
\begin{enumerate}
	\item \textbf{Planning: } The project is divided into a number of well defined tasks / activities. The time estimation for the completion of each activities and the sequential order in which the activities should be executed is indicated.
	
	\item \textbf{Scheduling: }The objective of this phase is to construct a time chart showing the start and finish time for each activity by applying forward and backward pass techniques and identify the critical path to indicate the critical activities; which require special attention if the project is to be completed on time; along with the slack and float for the non-critical paths.
	
	\item \textbf{Allocation of resources: } A resource is a physical variable required such as labour, equipment, space, etc. In this phase, the resources are allocated to difference activities to achieve the desired objective.
	
	\item \textbf{Controlling: }It refers to analysing and evaluating the actual progress against the plan. In this final phase of project management, a financial and technical control is obtained over the project, by having progress reports from time to time and updating the networking continuously. Reallocation of resources, crashing and review of project with periodical reports are carried out in this phase of the project.
	
	
\end{enumerate}

\subsection{Required definitions: }
\begin{enumerate}
	\item \textbf{Activity: }An individual operation which consumes resources, has a beginning and an end, is called an activity.\par
	An arrow is used to represent an activity, with its head indicating the direction of the progress of the project.
	\begin{center}
	\tikzset{every picture/.style={line width=0.75pt}} %set default line width to 0.75pt       
	
	\begin{tikzpicture}[x=0.75pt,y=0.75pt,yscale=-1,xscale=1]
		%uncomment if require: \path (0,300); %set diagram left start at 0, and has height of 300
		
		%Shape: Circle [id:dp8392507772232722] 
		\draw   (100,145) .. controls (100,131.19) and (111.19,120) .. (125,120) .. controls (138.81,120) and (150,131.19) .. (150,145) .. controls (150,158.81) and (138.81,170) .. (125,170) .. controls (111.19,170) and (100,158.81) .. (100,145) -- cycle ;
		%Shape: Circle [id:dp5153110312528408] 
		\draw   (212.16,145) .. controls (212.16,131.19) and (223.35,120) .. (237.16,120) .. controls (250.97,120) and (262.16,131.19) .. (262.16,145) .. controls (262.16,158.81) and (250.97,170) .. (237.16,170) .. controls (223.35,170) and (212.16,158.81) .. (212.16,145) -- cycle ;
		%Straight Lines [id:da6108163392134129] 
		\draw    (150,145) -- (210.16,145) ;
		\draw [shift={(212.16,145)}, rotate = 180] [color={rgb, 255:red, 0; green, 0; blue, 0 }  ][line width=0.75]    (10.93,-3.29) .. controls (6.95,-1.4) and (3.31,-0.3) .. (0,0) .. controls (3.31,0.3) and (6.95,1.4) .. (10.93,3.29)   ;
		
		% Text Node
		\draw (122,136.4) node [anchor=north west][inner sep=0.75pt]    {$i$};
		% Text Node
		\draw (172,124.4) node [anchor=north west][inner sep=0.75pt]    {$A$};
		% Text Node
		\draw (232,136.4) node [anchor=north west][inner sep=0.75pt]    {$j$};
		
		
	\end{tikzpicture}
	\end{center}
	Where A is the activity, i and j are events.
	\item \textbf{Event: }An event represents a point in time satisfying the completion of some activities and the beginning of the new ones.\par
	It is also called as a \textbf{node} or \textbf{connector} and is usually represented by a circle in the network.
	\item \textbf{Tail event: }It is the event from which the activity begins, i.e. the event from which the arrow emerges.
	\item \textbf{Head event: }It is the event into which the activity ends, ie/ te event into which the arrow enters.
	\item \textbf{Dummy activity: }An activity which does not consume any resources but only shows the technological dependence of activities is called a dummy activity. It is represented by a dotted line in the network.
	\item \textbf{Network: }When all the activities and events are connected logically and sequentially, they form a network.
	\item \textbf{Start event:} An event which does not have a preceding activity si called the start event of the network.
	\par Thus, the start event will have no arrows entering into it, but will only have arrows emerging out of it.
	\item \textbf{End event: }An event which does not have a succeeding activity is called the end event of the network.
	\item \textbf{Merge event: }Merge event is the vent into which more than one activities end, i.e. the event into which many arrows enter.
	\item \textbf{Burst event: }Burst event is the event from which more than one activities start, i.e. the event from which many arrows emerge
	\item \textbf{Predecessor activity:} Activity which must be completed before a particular activity starts is know na as a predecessor activity.
	\item \textbf{Successor activity: }Activity which must follow a particular activity is known as a successor activity.
	\item \textbf{Critical activity:} An activity which cannot be delayed if the project is to be completed on time is known as a critical activity.\par Total float for every critical activity is zero.
	\item \textbf{Critical event: }An event which cannot be delayed if the project is to completed is known as a critical event.\par
	Slack of every critical event is zero
	\item \textbf{Path:} A path is connected sequence of activities from start event to end event.
	\item \textbf{Critical path: }A path connecting the start event and the end event through the critical events or critical activities is called the critical path.
	\par 
	Duration-wise it is the longest path on the network. It is represented using \textbf{bolder} lines on the network.
\end{enumerate}

\subsection{Rules for drawing the network}
\begin{enumerate}
	\item Each activity is represented by only one arrow in the network.
	\item The arrows should be drawn as straight lines.
	\item The arrows should not cross each other.
	\item The length of an arrow is of no significance.
	\item Time flows from left to right.
	\item Arrows pointing in opposite direction are to be avoided.
	\item Angles between arrows should be as large as possible.
	\item Every network should have a unique start event.
	\item Every network should have a unique end event.
	\item Before an activity can be undertaken, all activities preceding it must be completed.
	\item There should be no duplication in event numbers.
	\item No two activities can have same start event and end event..
	\item No loops allowed.
\end{enumerate}

\begin{example}
	Event$ - 2;3;4;5;6;7 $, preceded by $ 1;1;3;3,4;2,3,5;5,6 $
\end{example}

	
	\begin{center}
	\tikzset{every picture/.style={line width=0.75pt}} %set default line width to 0.75pt        
	
	\begin{tikzpicture}[x=0.75pt,y=0.75pt,yscale=-0.75,xscale=0.75]
		%uncomment if require: \path (0,300); %set diagram left start at 0, and has height of 300
		
		%Shape: Circle [id:dp6836186293707955] 
		\draw   (116,141.4) .. controls (116,127.59) and (127.19,116.4) .. (141,116.4) .. controls (154.81,116.4) and (166,127.59) .. (166,141.4) .. controls (166,155.21) and (154.81,166.4) .. (141,166.4) .. controls (127.19,166.4) and (116,155.21) .. (116,141.4) -- cycle ;
		%Shape: Circle [id:dp019083043223606877] 
		\draw   (181,79.4) .. controls (181,65.59) and (192.19,54.4) .. (206,54.4) .. controls (219.81,54.4) and (231,65.59) .. (231,79.4) .. controls (231,93.21) and (219.81,104.4) .. (206,104.4) .. controls (192.19,104.4) and (181,93.21) .. (181,79.4) -- cycle ;
		%Shape: Circle [id:dp4278571199544363] 
		\draw   (181,202.4) .. controls (181,188.59) and (192.19,177.4) .. (206,177.4) .. controls (219.81,177.4) and (231,188.59) .. (231,202.4) .. controls (231,216.21) and (219.81,227.4) .. (206,227.4) .. controls (192.19,227.4) and (181,216.21) .. (181,202.4) -- cycle ;
		%Shape: Circle [id:dp9120811337466117] 
		\draw   (301,205.4) .. controls (301,191.59) and (312.19,180.4) .. (326,180.4) .. controls (339.81,180.4) and (351,191.59) .. (351,205.4) .. controls (351,219.21) and (339.81,230.4) .. (326,230.4) .. controls (312.19,230.4) and (301,219.21) .. (301,205.4) -- cycle ;
		%Shape: Circle [id:dp4504776507476198] 
		\draw   (301,135.4) .. controls (301,121.59) and (312.19,110.4) .. (326,110.4) .. controls (339.81,110.4) and (351,121.59) .. (351,135.4) .. controls (351,149.21) and (339.81,160.4) .. (326,160.4) .. controls (312.19,160.4) and (301,149.21) .. (301,135.4) -- cycle ;
		%Shape: Circle [id:dp6586155730537133] 
		\draw   (301,37.4) .. controls (301,23.59) and (312.19,12.4) .. (326,12.4) .. controls (339.81,12.4) and (351,23.59) .. (351,37.4) .. controls (351,51.21) and (339.81,62.4) .. (326,62.4) .. controls (312.19,62.4) and (301,51.21) .. (301,37.4) -- cycle ;
		%Straight Lines [id:da9642228799195385] 
		\draw    (141,116.4) -- (179.53,80.76) ;
		\draw [shift={(181,79.4)}, rotate = 137.23] [color={rgb, 255:red, 0; green, 0; blue, 0 }  ][line width=0.75]    (10.93,-3.29) .. controls (6.95,-1.4) and (3.31,-0.3) .. (0,0) .. controls (3.31,0.3) and (6.95,1.4) .. (10.93,3.29)   ;
		%Straight Lines [id:da3744212091001955] 
		\draw    (141,166.4) -- (179.51,201.06) ;
		\draw [shift={(181,202.4)}, rotate = 221.99] [color={rgb, 255:red, 0; green, 0; blue, 0 }  ][line width=0.75]    (10.93,-3.29) .. controls (6.95,-1.4) and (3.31,-0.3) .. (0,0) .. controls (3.31,0.3) and (6.95,1.4) .. (10.93,3.29)   ;
		%Straight Lines [id:da7116312977621821] 
		\draw    (231,202.4) -- (299,205.31) ;
		\draw [shift={(301,205.4)}, rotate = 182.45] [color={rgb, 255:red, 0; green, 0; blue, 0 }  ][line width=0.75]    (10.93,-3.29) .. controls (6.95,-1.4) and (3.31,-0.3) .. (0,0) .. controls (3.31,0.3) and (6.95,1.4) .. (10.93,3.29)   ;
		%Straight Lines [id:da06927021199163863] 
		\draw    (231,202.4) -- (299.56,136.78) ;
		\draw [shift={(301,135.4)}, rotate = 136.25] [color={rgb, 255:red, 0; green, 0; blue, 0 }  ][line width=0.75]    (10.93,-3.29) .. controls (6.95,-1.4) and (3.31,-0.3) .. (0,0) .. controls (3.31,0.3) and (6.95,1.4) .. (10.93,3.29)   ;
		%Straight Lines [id:da32032307138484484] 
		\draw    (231,202.4) -- (300.22,39.24) ;
		\draw [shift={(301,37.4)}, rotate = 112.99] [color={rgb, 255:red, 0; green, 0; blue, 0 }  ][line width=0.75]    (10.93,-3.29) .. controls (6.95,-1.4) and (3.31,-0.3) .. (0,0) .. controls (3.31,0.3) and (6.95,1.4) .. (10.93,3.29)   ;
		%Straight Lines [id:da9480584772654082] 
		\draw    (231,79.4) -- (299.29,38.43) ;
		\draw [shift={(301,37.4)}, rotate = 149.04] [color={rgb, 255:red, 0; green, 0; blue, 0 }  ][line width=0.75]    (10.93,-3.29) .. controls (6.95,-1.4) and (3.31,-0.3) .. (0,0) .. controls (3.31,0.3) and (6.95,1.4) .. (10.93,3.29)   ;
		%Shape: Circle [id:dp9485676822364819] 
		\draw   (398,135.4) .. controls (398,121.59) and (409.19,110.4) .. (423,110.4) .. controls (436.81,110.4) and (448,121.59) .. (448,135.4) .. controls (448,149.21) and (436.81,160.4) .. (423,160.4) .. controls (409.19,160.4) and (398,149.21) .. (398,135.4) -- cycle ;
		%Straight Lines [id:da1526254291964464] 
		\draw    (351,135.4) -- (396,135.4) ;
		\draw [shift={(398,135.4)}, rotate = 180] [color={rgb, 255:red, 0; green, 0; blue, 0 }  ][line width=0.75]    (10.93,-3.29) .. controls (6.95,-1.4) and (3.31,-0.3) .. (0,0) .. controls (3.31,0.3) and (6.95,1.4) .. (10.93,3.29)   ;
		%Straight Lines [id:da7273009892996174] 
		\draw    (351,37.4) -- (422.59,108.99) ;
		\draw [shift={(424,110.4)}, rotate = 225] [color={rgb, 255:red, 0; green, 0; blue, 0 }  ][line width=0.75]    (10.93,-3.29) .. controls (6.95,-1.4) and (3.31,-0.3) .. (0,0) .. controls (3.31,0.3) and (6.95,1.4) .. (10.93,3.29)   ;
		
		% Text Node
		\draw (135.6,133.8) node [anchor=north west][inner sep=0.75pt]    {$1$};
		% Text Node
		\draw (200.6,71.8) node [anchor=north west][inner sep=0.75pt]    {$2$};
		% Text Node
		\draw (200.6,194.8) node [anchor=north west][inner sep=0.75pt]    {$3$};
		% Text Node
		\draw (320.6,197.8) node [anchor=north west][inner sep=0.75pt]    {$4$};
		% Text Node
		\draw (320.6,127.8) node [anchor=north west][inner sep=0.75pt]    {$5$};
		% Text Node
		\draw (320.6,29.8) node [anchor=north west][inner sep=0.75pt]    {$6$};
		% Text Node
		\draw (417.6,127.8) node [anchor=north west][inner sep=0.75pt]    {$7$};
		
		
	\end{tikzpicture}
	
	\end{center}

\begin{example}
	Make a diagram based on the following table
	\begin{table}[h]
		\centering
		\begin{tabular}{@{}rcl@{}}
			\toprule
			\multicolumn{1}{c}{Sr. No.} & Activity & \multicolumn{1}{c}{Duration (hrs)} \\ \midrule
			a) & \begin{tabular}[c]{@{}c@{}}Preparing a pattern\\ for casting item `A'\end{tabular} & 4 \\
			b) & Preparing a mould & 2 \\
			c) & \begin{tabular}[c]{@{}c@{}}Casting and cleaning \\ item `A'\end{tabular} & 1 \\
			d) & Heat treatment for `A' & 2 \\
			e) & \begin{tabular}[c]{@{}c@{}}Obtaining and installing \\ machine `M'\end{tabular} & 7 \\
			f) & Preparing item `B' by `M' & 5 \\
			g) & Assembling `A' and `B' & 3 \\
			h) & Preparing test rig & 4 \\
			i) & Testing the assembly & 2 \\
			j) & Packing & 1 \\ \bottomrule
		\end{tabular}
	\end{table}

	\begin{center}
		
		
		\tikzset{every picture/.style={line width=0.75pt}} %set default line width to 0.75pt        
		
		\begin{tikzpicture}[x=0.75pt,y=0.75pt,yscale=-0.5,xscale=0.5]
			%uncomment if require: \path (0,300); %set diagram left start at 0, and has height of 300
			
			%Shape: Circle [id:dp6836186293707955] 
			\draw   (116,141.4) .. controls (116,127.59) and (127.19,116.4) .. (141,116.4) .. controls (154.81,116.4) and (166,127.59) .. (166,141.4) .. controls (166,155.21) and (154.81,166.4) .. (141,166.4) .. controls (127.19,166.4) and (116,155.21) .. (116,141.4) -- cycle ;
			%Straight Lines [id:da9642228799195385] 
			\draw    (141,116.4) -- (179.16,34.21) ;
			\draw [shift={(180,32.4)}, rotate = 114.9] [color={rgb, 255:red, 0; green, 0; blue, 0 }  ][line width=0.75]    (10.93,-3.29) .. controls (6.95,-1.4) and (3.31,-0.3) .. (0,0) .. controls (3.31,0.3) and (6.95,1.4) .. (10.93,3.29)   ;
			%Straight Lines [id:da3744212091001955] 
			\draw    (166,141.4) -- (211,141.4) ;
			\draw [shift={(213,141.4)}, rotate = 180] [color={rgb, 255:red, 0; green, 0; blue, 0 }  ][line width=0.75]    (10.93,-3.29) .. controls (6.95,-1.4) and (3.31,-0.3) .. (0,0) .. controls (3.31,0.3) and (6.95,1.4) .. (10.93,3.29)   ;
			%Shape: Circle [id:dp6383570048211147] 
			\draw   (180,32.4) .. controls (180,18.59) and (191.19,7.4) .. (205,7.4) .. controls (218.81,7.4) and (230,18.59) .. (230,32.4) .. controls (230,46.21) and (218.81,57.4) .. (205,57.4) .. controls (191.19,57.4) and (180,46.21) .. (180,32.4) -- cycle ;
			%Straight Lines [id:da6612504067380269] 
			\draw    (230,32.4) -- (276.2,32.4) ;
			\draw [shift={(278.2,32.4)}, rotate = 180] [color={rgb, 255:red, 0; green, 0; blue, 0 }  ][line width=0.75]    (10.93,-3.29) .. controls (6.95,-1.4) and (3.31,-0.3) .. (0,0) .. controls (3.31,0.3) and (6.95,1.4) .. (10.93,3.29)   ;
			%Shape: Circle [id:dp44106908707634873] 
			\draw   (278.2,32.4) .. controls (278.2,18.59) and (289.39,7.4) .. (303.2,7.4) .. controls (317.01,7.4) and (328.2,18.59) .. (328.2,32.4) .. controls (328.2,46.21) and (317.01,57.4) .. (303.2,57.4) .. controls (289.39,57.4) and (278.2,46.21) .. (278.2,32.4) -- cycle ;
			%Straight Lines [id:da0656680567878658] 
			\draw    (328.2,32.4) -- (374.4,32.4) ;
			\draw [shift={(376.4,32.4)}, rotate = 180] [color={rgb, 255:red, 0; green, 0; blue, 0 }  ][line width=0.75]    (10.93,-3.29) .. controls (6.95,-1.4) and (3.31,-0.3) .. (0,0) .. controls (3.31,0.3) and (6.95,1.4) .. (10.93,3.29)   ;
			%Shape: Circle [id:dp7017455058555764] 
			\draw   (376.4,32.4) .. controls (376.4,18.59) and (387.59,7.4) .. (401.4,7.4) .. controls (415.21,7.4) and (426.4,18.59) .. (426.4,32.4) .. controls (426.4,46.21) and (415.21,57.4) .. (401.4,57.4) .. controls (387.59,57.4) and (376.4,46.21) .. (376.4,32.4) -- cycle ;
			%Shape: Circle [id:dp40567221833206335] 
			\draw   (213,141.4) .. controls (213,127.59) and (224.19,116.4) .. (238,116.4) .. controls (251.81,116.4) and (263,127.59) .. (263,141.4) .. controls (263,155.21) and (251.81,166.4) .. (238,166.4) .. controls (224.19,166.4) and (213,155.21) .. (213,141.4) -- cycle ;
			%Straight Lines [id:da5732749652132056] 
			\draw    (426.4,32.4) -- (472.6,32.4) ;
			\draw [shift={(474.6,32.4)}, rotate = 180] [color={rgb, 255:red, 0; green, 0; blue, 0 }  ][line width=0.75]    (10.93,-3.29) .. controls (6.95,-1.4) and (3.31,-0.3) .. (0,0) .. controls (3.31,0.3) and (6.95,1.4) .. (10.93,3.29)   ;
			%Shape: Circle [id:dp04645686264293958] 
			\draw   (474.6,32.4) .. controls (474.6,18.59) and (485.79,7.4) .. (499.6,7.4) .. controls (513.41,7.4) and (524.6,18.59) .. (524.6,32.4) .. controls (524.6,46.21) and (513.41,57.4) .. (499.6,57.4) .. controls (485.79,57.4) and (474.6,46.21) .. (474.6,32.4) -- cycle ;
			%Straight Lines [id:da39345368189022856] 
			\draw    (497.6,59.4) -- (497.6,168.3) ;
			\draw [shift={(497.6,170.3)}, rotate = 270] [color={rgb, 255:red, 0; green, 0; blue, 0 }  ][line width=0.75]    (10.93,-3.29) .. controls (6.95,-1.4) and (3.31,-0.3) .. (0,0) .. controls (3.31,0.3) and (6.95,1.4) .. (10.93,3.29)   ;
			%Shape: Circle [id:dp2889219974848283] 
			\draw   (472.6,195.3) .. controls (472.6,181.49) and (483.79,170.3) .. (497.6,170.3) .. controls (511.41,170.3) and (522.6,181.49) .. (522.6,195.3) .. controls (522.6,209.11) and (511.41,220.3) .. (497.6,220.3) .. controls (483.79,220.3) and (472.6,209.11) .. (472.6,195.3) -- cycle ;
			%Straight Lines [id:da42243423156018167] 
			\draw    (263,141.4) -- (472.82,33.32) ;
			\draw [shift={(474.6,32.4)}, rotate = 152.75] [color={rgb, 255:red, 0; green, 0; blue, 0 }  ][line width=0.75]    (10.93,-3.29) .. controls (6.95,-1.4) and (3.31,-0.3) .. (0,0) .. controls (3.31,0.3) and (6.95,1.4) .. (10.93,3.29)   ;
			%Straight Lines [id:da8633361948514946] 
			\draw    (141,166.4) -- (470.61,195.13) ;
			\draw [shift={(472.6,195.3)}, rotate = 184.98] [color={rgb, 255:red, 0; green, 0; blue, 0 }  ][line width=0.75]    (10.93,-3.29) .. controls (6.95,-1.4) and (3.31,-0.3) .. (0,0) .. controls (3.31,0.3) and (6.95,1.4) .. (10.93,3.29)   ;
			%Straight Lines [id:da6095523406552852] 
			\draw    (497.6,220.3) -- (497.6,249.3) ;
			\draw [shift={(497.6,251.3)}, rotate = 270] [color={rgb, 255:red, 0; green, 0; blue, 0 }  ][line width=0.75]    (10.93,-3.29) .. controls (6.95,-1.4) and (3.31,-0.3) .. (0,0) .. controls (3.31,0.3) and (6.95,1.4) .. (10.93,3.29)   ;
			%Shape: Circle [id:dp7351927539241412] 
			\draw   (472.6,276.3) .. controls (472.6,262.49) and (483.79,251.3) .. (497.6,251.3) .. controls (511.41,251.3) and (522.6,262.49) .. (522.6,276.3) .. controls (522.6,290.11) and (511.41,301.3) .. (497.6,301.3) .. controls (483.79,301.3) and (472.6,290.11) .. (472.6,276.3) -- cycle ;
			%Shape: Circle [id:dp3349461903691655] 
			\draw   (566.8,276.3) .. controls (566.8,262.49) and (577.99,251.3) .. (591.8,251.3) .. controls (605.61,251.3) and (616.8,262.49) .. (616.8,276.3) .. controls (616.8,290.11) and (605.61,301.3) .. (591.8,301.3) .. controls (577.99,301.3) and (566.8,290.11) .. (566.8,276.3) -- cycle ;
			%Straight Lines [id:da07404378903096265] 
			\draw    (522.6,276.3) -- (564.8,276.3) ;
			\draw [shift={(566.8,276.3)}, rotate = 180] [color={rgb, 255:red, 0; green, 0; blue, 0 }  ][line width=0.75]    (10.93,-3.29) .. controls (6.95,-1.4) and (3.31,-0.3) .. (0,0) .. controls (3.31,0.3) and (6.95,1.4) .. (10.93,3.29)   ;
			
			% Text Node
			\draw (124,59.4) node [anchor=north west][inner sep=0.75pt]  [xscale=0.5,yscale=0.5]  {$a\ ( 4)$};
			% Text Node
			\draw (173,118.4) node [anchor=north west][inner sep=0.75pt]  [xscale=0.5,yscale=0.5]  {$e\ ( 7)$};
			% Text Node
			\draw (235,11.4) node [anchor=north west][inner sep=0.75pt]  [xscale=0.5,yscale=0.5]  {$b\ ( 2)$};
			% Text Node
			\draw (333,11.4) node [anchor=north west][inner sep=0.75pt]  [xscale=0.5,yscale=0.5]  {$c\ ( 1)$};
			% Text Node
			\draw (433,11.4) node [anchor=north west][inner sep=0.75pt]  [xscale=0.5,yscale=0.5]  {$d\ ( 2)$};
			% Text Node
			\draw (502.6,99.4) node [anchor=north west][inner sep=0.75pt]  [xscale=0.5,yscale=0.5]  {$g\ ( 3)$};
			% Text Node
			\draw (370.8,90.3) node [anchor=north west][inner sep=0.75pt]  [xscale=0.5,yscale=0.5]  {$f\ ( 5)$};
			% Text Node
			\draw (503.6,224.4) node [anchor=north west][inner sep=0.75pt]  [xscale=0.5,yscale=0.5]  {$i\ ( 2)$};
			% Text Node
			\draw (526.6,254.4) node [anchor=north west][inner sep=0.75pt]  [xscale=0.5,yscale=0.5]  {$j\ ( 1)$};
			% Text Node
			\draw (308.8,184.25) node [anchor=north west][inner sep=0.75pt]  [xscale=0.5,yscale=0.5]  {$h\ ( 4)$};
			
			
		\end{tikzpicture}
	\end{center}
\end{example}

\section{Fulkerson's Rule for Numbering the events}
\begin{enumerate}
	\item A start event is an event is an event which has only emerging arrows, and no entering arrows. The start event is numbered as $ 1 $.
	\item Strike out all the arrows that emerge out of event 1. This will create at least one event that has no entering arrows. Number these events as $ 2,3,4,\dots $.
	\item Continue this procedure till all the events are numbered.
\end{enumerate}
\textbf{Note: }Events can be numbered as $ 1,2,3,\dots $ or $ 10, 20, 30, \dots $.

\begin{example}
	Draw a network for the following project and number the events according to Fulkerson's rule.\\
	`A' is a start event and, `K' is end event. `A' precedes the event `B', `J' is the successor event to `F', `C' and `D' are successor events to `B', `D' is preceding event to `G', `E' and `F' occurs after event `C', `E' precedes `F', `C' restrains the occurrence of `G' and `G' precedes `F'.\\ `H' precedes `J' and `K' succeeds `J', `F' restrains the occurrence of `H'.
	\begin{center}
		
		
		\tikzset{every picture/.style={line width=0.75pt}} %set default line width to 0.75pt        
		
		\begin{tikzpicture}[x=0.75pt,y=0.75pt,yscale=-0.5,xscale=0.5]
			%uncomment if require: \path (0,392); %set diagram left start at 0, and has height of 392
			
			%Shape: Ellipse [id:dp4168223699738398] 
			\draw   (14.67,281.66) .. controls (14.67,269.4) and (24.6,259.46) .. (36.86,259.46) .. controls (49.12,259.46) and (59.05,269.4) .. (59.05,281.66) .. controls (59.05,293.91) and (49.12,303.85) .. (36.86,303.85) .. controls (24.6,303.85) and (14.67,293.91) .. (14.67,281.66) -- cycle ;
			%Straight Lines [id:da6688149638976573] 
			\draw    (59.05,281.66) -- (113.1,281.66) ;
			\draw [shift={(115.1,281.66)}, rotate = 180] [color={rgb, 255:red, 0; green, 0; blue, 0 }  ][line width=0.75]    (10.93,-3.29) .. controls (6.95,-1.4) and (3.31,-0.3) .. (0,0) .. controls (3.31,0.3) and (6.95,1.4) .. (10.93,3.29)   ;
			%Shape: Ellipse [id:dp10818224336569138] 
			\draw   (115.1,281.66) .. controls (115.1,269.4) and (125.04,259.46) .. (137.29,259.46) .. controls (149.55,259.46) and (159.49,269.4) .. (159.49,281.66) .. controls (159.49,293.91) and (149.55,303.85) .. (137.29,303.85) .. controls (125.04,303.85) and (115.1,293.91) .. (115.1,281.66) -- cycle ;
			%Shape: Ellipse [id:dp4977396967579699] 
			\draw   (189.67,216.56) .. controls (189.67,204.3) and (199.61,194.36) .. (211.86,194.36) .. controls (224.12,194.36) and (234.06,204.3) .. (234.06,216.56) .. controls (234.06,228.81) and (224.12,238.75) .. (211.86,238.75) .. controls (199.61,238.75) and (189.67,228.81) .. (189.67,216.56) -- cycle ;
			%Shape: Circle [id:dp9130037394045596] 
			\draw   (195.59,338.47) .. controls (195.59,326.22) and (205.53,316.28) .. (217.78,316.28) .. controls (230.04,316.28) and (239.98,326.22) .. (239.98,338.47) .. controls (239.98,350.73) and (230.04,360.67) .. (217.78,360.67) .. controls (205.53,360.67) and (195.59,350.73) .. (195.59,338.47) -- cycle ;
			%Straight Lines [id:da7994304956075431] 
			\draw    (137.29,303.85) -- (193.87,337.45) ;
			\draw [shift={(195.59,338.47)}, rotate = 210.71] [color={rgb, 255:red, 0; green, 0; blue, 0 }  ][line width=0.75]    (10.93,-3.29) .. controls (6.95,-1.4) and (3.31,-0.3) .. (0,0) .. controls (3.31,0.3) and (6.95,1.4) .. (10.93,3.29)   ;
			%Straight Lines [id:da01890580611725956] 
			\draw    (137.29,259.46) -- (188.12,217.82) ;
			\draw [shift={(189.67,216.56)}, rotate = 140.68] [color={rgb, 255:red, 0; green, 0; blue, 0 }  ][line width=0.75]    (10.93,-3.29) .. controls (6.95,-1.4) and (3.31,-0.3) .. (0,0) .. controls (3.31,0.3) and (6.95,1.4) .. (10.93,3.29)   ;
			%Shape: Circle [id:dp12050722385470247] 
			\draw   (303.3,338.47) .. controls (303.3,326.22) and (313.24,316.28) .. (325.49,316.28) .. controls (337.75,316.28) and (347.69,326.22) .. (347.69,338.47) .. controls (347.69,350.73) and (337.75,360.67) .. (325.49,360.67) .. controls (313.24,360.67) and (303.3,350.73) .. (303.3,338.47) -- cycle ;
			%Straight Lines [id:da7298081092979858] 
			\draw    (239.98,338.47) -- (301.3,338.47) ;
			\draw [shift={(303.3,338.47)}, rotate = 180] [color={rgb, 255:red, 0; green, 0; blue, 0 }  ][line width=0.75]    (10.93,-3.29) .. controls (6.95,-1.4) and (3.31,-0.3) .. (0,0) .. controls (3.31,0.3) and (6.95,1.4) .. (10.93,3.29)   ;
			%Straight Lines [id:da22311691076612306] 
			\draw    (234.06,216.56) -- (291.83,216.56) ;
			\draw [shift={(293.83,216.56)}, rotate = 180] [color={rgb, 255:red, 0; green, 0; blue, 0 }  ][line width=0.75]    (10.93,-3.29) .. controls (6.95,-1.4) and (3.31,-0.3) .. (0,0) .. controls (3.31,0.3) and (6.95,1.4) .. (10.93,3.29)   ;
			%Straight Lines [id:da11554236355207959] 
			\draw    (234.06,216.56) -- (324.14,314.81) ;
			\draw [shift={(325.49,316.28)}, rotate = 227.48] [color={rgb, 255:red, 0; green, 0; blue, 0 }  ][line width=0.75]    (10.93,-3.29) .. controls (6.95,-1.4) and (3.31,-0.3) .. (0,0) .. controls (3.31,0.3) and (6.95,1.4) .. (10.93,3.29)   ;
			%Shape: Ellipse [id:dp29005173263070017] 
			\draw   (293.83,216.56) .. controls (293.83,204.3) and (303.77,194.36) .. (316.03,194.36) .. controls (328.28,194.36) and (338.22,204.3) .. (338.22,216.56) .. controls (338.22,228.81) and (328.28,238.75) .. (316.03,238.75) .. controls (303.77,238.75) and (293.83,228.81) .. (293.83,216.56) -- cycle ;
			%Shape: Ellipse [id:dp014715463265752682] 
			\draw   (293.83,117.54) .. controls (293.83,105.29) and (303.77,95.35) .. (316.03,95.35) .. controls (328.28,95.35) and (338.22,105.29) .. (338.22,117.54) .. controls (338.22,129.8) and (328.28,139.74) .. (316.03,139.74) .. controls (303.77,139.74) and (293.83,129.8) .. (293.83,117.54) -- cycle ;
			%Straight Lines [id:da29169005489774213] 
			\draw    (316.03,139.74) -- (316.03,192.36) ;
			\draw [shift={(316.03,194.36)}, rotate = 270] [color={rgb, 255:red, 0; green, 0; blue, 0 }  ][line width=0.75]    (10.93,-3.29) .. controls (6.95,-1.4) and (3.31,-0.3) .. (0,0) .. controls (3.31,0.3) and (6.95,1.4) .. (10.93,3.29)   ;
			%Straight Lines [id:da7796938725436213] 
			\draw    (211.86,194.36) -- (292.37,118.91) ;
			\draw [shift={(293.83,117.54)}, rotate = 136.86] [color={rgb, 255:red, 0; green, 0; blue, 0 }  ][line width=0.75]    (10.93,-3.29) .. controls (6.95,-1.4) and (3.31,-0.3) .. (0,0) .. controls (3.31,0.3) and (6.95,1.4) .. (10.93,3.29)   ;
			%Shape: Ellipse [id:dp6990843920921903] 
			\draw   (411.01,338.47) .. controls (411.01,326.22) and (420.95,316.28) .. (433.21,316.28) .. controls (445.46,316.28) and (455.4,326.22) .. (455.4,338.47) .. controls (455.4,350.73) and (445.46,360.67) .. (433.21,360.67) .. controls (420.95,360.67) and (411.01,350.73) .. (411.01,338.47) -- cycle ;
			%Straight Lines [id:da38712869850045983] 
			\draw    (347.69,338.47) -- (409.01,338.47) ;
			\draw [shift={(411.01,338.47)}, rotate = 180] [color={rgb, 255:red, 0; green, 0; blue, 0 }  ][line width=0.75]    (10.93,-3.29) .. controls (6.95,-1.4) and (3.31,-0.3) .. (0,0) .. controls (3.31,0.3) and (6.95,1.4) .. (10.93,3.29)   ;
			%Straight Lines [id:da2567606335572017] 
			\draw    (338.22,216.56) -- (431.83,314.83) ;
			\draw [shift={(433.21,316.28)}, rotate = 226.39] [color={rgb, 255:red, 0; green, 0; blue, 0 }  ][line width=0.75]    (10.93,-3.29) .. controls (6.95,-1.4) and (3.31,-0.3) .. (0,0) .. controls (3.31,0.3) and (6.95,1.4) .. (10.93,3.29)   ;
			%Straight Lines [id:da6895837440867616] 
			\draw    (455.4,338.47) -- (516.73,338.47) ;
			\draw [shift={(518.73,338.47)}, rotate = 180] [color={rgb, 255:red, 0; green, 0; blue, 0 }  ][line width=0.75]    (10.93,-3.29) .. controls (6.95,-1.4) and (3.31,-0.3) .. (0,0) .. controls (3.31,0.3) and (6.95,1.4) .. (10.93,3.29)   ;
			%Shape: Circle [id:dp87710523129537] 
			\draw   (518.73,338.47) .. controls (518.73,326.22) and (528.66,316.28) .. (540.92,316.28) .. controls (553.18,316.28) and (563.11,326.22) .. (563.11,338.47) .. controls (563.11,350.73) and (553.18,360.67) .. (540.92,360.67) .. controls (528.66,360.67) and (518.73,350.73) .. (518.73,338.47) -- cycle ;
			%Straight Lines [id:da7220218823085904] 
			\draw    (563.11,338.47) -- (624.44,338.47) ;
			\draw [shift={(626.44,338.47)}, rotate = 180] [color={rgb, 255:red, 0; green, 0; blue, 0 }  ][line width=0.75]    (10.93,-3.29) .. controls (6.95,-1.4) and (3.31,-0.3) .. (0,0) .. controls (3.31,0.3) and (6.95,1.4) .. (10.93,3.29)   ;
			%Shape: Circle [id:dp708777401774799] 
			\draw   (626.44,338.47) .. controls (626.44,326.22) and (636.38,316.28) .. (648.63,316.28) .. controls (660.89,316.28) and (670.83,326.22) .. (670.83,338.47) .. controls (670.83,350.73) and (660.89,360.67) .. (648.63,360.67) .. controls (636.38,360.67) and (626.44,350.73) .. (626.44,338.47) -- cycle ;
			
			% Text Node
			\draw (27.73,272.87) node [anchor=north west][inner sep=0.75pt]  [xscale=0.5,yscale=0.5]  {$A$};
			% Text Node
			\draw (129.34,272.87) node [anchor=north west][inner sep=0.75pt]  [xscale=0.5,yscale=0.5]  {$B$};
			% Text Node
			\draw (203.86,207.77) node [anchor=north west][inner sep=0.75pt]  [xscale=0.5,yscale=0.5]  {$C$};
			% Text Node
			\draw (209.83,329.69) node [anchor=north west][inner sep=0.75pt]  [xscale=0.5,yscale=0.5]  {$D$};
			% Text Node
			\draw (317.49,330.58) node [anchor=north west][inner sep=0.75pt]  [xscale=0.5,yscale=0.5]  {$G$};
			% Text Node
			\draw (308.67,208.66) node [anchor=north west][inner sep=0.75pt]  [xscale=0.5,yscale=0.5]  {$F$};
			% Text Node
			\draw (308.08,108.05) node [anchor=north west][inner sep=0.75pt]  [xscale=0.5,yscale=0.5]  {$E$};
			% Text Node
			\draw (424.85,330.87) node [anchor=north west][inner sep=0.75pt]  [xscale=0.5,yscale=0.5]  {$H$};
			% Text Node
			\draw (534.45,329.87) node [anchor=north west][inner sep=0.75pt]  [xscale=0.5,yscale=0.5]  {$J$};
			% Text Node
			\draw (639.78,329.54) node [anchor=north west][inner sep=0.75pt]  [xscale=0.5,yscale=0.5]  {$K$};
			
			
		\end{tikzpicture}
		
	\end{center}
	
	
\end{example}

\begin{example}
	B,C,D preceeding A; E,F preceeding B; G preceeding E; H preceeding F; I preceeding G,H; J preceeding G,H,C,D; K preceeding D; L preceeding I,J,K
\end{example}
\begin{proof}
	\begin{center}
		
		
		\tikzset{every picture/.style={line width=0.75pt}} %set default line width to 0.75pt        
		
		\begin{tikzpicture}[x=0.75pt,y=0.75pt,yscale=-0.5,xscale=0.5]
			%uncomment if require: \path (0,392); %set diagram left start at 0, and has height of 392
			
			%Shape: Ellipse [id:dp4168223699738398] 
			\draw   (14.67,281.66) .. controls (14.67,269.4) and (24.6,259.46) .. (36.86,259.46) .. controls (49.12,259.46) and (59.05,269.4) .. (59.05,281.66) .. controls (59.05,293.91) and (49.12,303.85) .. (36.86,303.85) .. controls (24.6,303.85) and (14.67,293.91) .. (14.67,281.66) -- cycle ;
			%Straight Lines [id:da6688149638976573] 
			\draw    (59.05,281.66) -- (113.1,281.66) ;
			\draw [shift={(115.1,281.66)}, rotate = 180] [color={rgb, 255:red, 0; green, 0; blue, 0 }  ][line width=0.75]    (10.93,-3.29) .. controls (6.95,-1.4) and (3.31,-0.3) .. (0,0) .. controls (3.31,0.3) and (6.95,1.4) .. (10.93,3.29)   ;
			%Shape: Ellipse [id:dp5042074057861572] 
			\draw   (115.1,281.66) .. controls (115.1,269.4) and (125.04,259.46) .. (137.29,259.46) .. controls (149.55,259.46) and (159.49,269.4) .. (159.49,281.66) .. controls (159.49,293.91) and (149.55,303.85) .. (137.29,303.85) .. controls (125.04,303.85) and (115.1,293.91) .. (115.1,281.66) -- cycle ;
			%Straight Lines [id:da024222813025277734] 
			\draw    (159.49,281.66) -- (218.78,222.36) ;
			\draw [shift={(220.19,220.95)}, rotate = 135] [color={rgb, 255:red, 0; green, 0; blue, 0 }  ][line width=0.75]    (10.93,-3.29) .. controls (6.95,-1.4) and (3.31,-0.3) .. (0,0) .. controls (3.31,0.3) and (6.95,1.4) .. (10.93,3.29)   ;
			%Straight Lines [id:da32740097830614157] 
			\draw    (159.49,281.66) -- (217.99,340.16) ;
			\draw [shift={(219.4,341.57)}, rotate = 225] [color={rgb, 255:red, 0; green, 0; blue, 0 }  ][line width=0.75]    (10.93,-3.29) .. controls (6.95,-1.4) and (3.31,-0.3) .. (0,0) .. controls (3.31,0.3) and (6.95,1.4) .. (10.93,3.29)   ;
			%Shape: Ellipse [id:dp43967421046272714] 
			\draw   (220.19,220.95) .. controls (220.19,208.69) and (230.13,198.76) .. (242.39,198.76) .. controls (254.65,198.76) and (264.58,208.69) .. (264.58,220.95) .. controls (264.58,233.21) and (254.65,243.14) .. (242.39,243.14) .. controls (230.13,243.14) and (220.19,233.21) .. (220.19,220.95) -- cycle ;
			%Shape: Ellipse [id:dp21703094796421651] 
			\draw   (219.4,341.57) .. controls (219.4,329.31) and (229.34,319.38) .. (241.59,319.38) .. controls (253.85,319.38) and (263.79,329.31) .. (263.79,341.57) .. controls (263.79,353.83) and (253.85,363.76) .. (241.59,363.76) .. controls (229.34,363.76) and (219.4,353.83) .. (219.4,341.57) -- cycle ;
			%Straight Lines [id:da9312411828457061] 
			\draw    (242.39,198.76) -- (301.68,139.46) ;
			\draw [shift={(303.1,138.05)}, rotate = 135] [color={rgb, 255:red, 0; green, 0; blue, 0 }  ][line width=0.75]    (10.93,-3.29) .. controls (6.95,-1.4) and (3.31,-0.3) .. (0,0) .. controls (3.31,0.3) and (6.95,1.4) .. (10.93,3.29)   ;
			%Straight Lines [id:da37565395534814083] 
			\draw    (242.39,198.76) -- (299.8,198.76) ;
			\draw [shift={(301.8,198.76)}, rotate = 180] [color={rgb, 255:red, 0; green, 0; blue, 0 }  ][line width=0.75]    (10.93,-3.29) .. controls (6.95,-1.4) and (3.31,-0.3) .. (0,0) .. controls (3.31,0.3) and (6.95,1.4) .. (10.93,3.29)   ;
			%Shape: Ellipse [id:dp9554454604461127] 
			\draw   (301.8,198.76) .. controls (301.8,186.5) and (311.74,176.56) .. (323.99,176.56) .. controls (336.25,176.56) and (346.19,186.5) .. (346.19,198.76) .. controls (346.19,211.01) and (336.25,220.95) .. (323.99,220.95) .. controls (311.74,220.95) and (301.8,211.01) .. (301.8,198.76) -- cycle ;
			%Shape: Ellipse [id:dp08673949763936184] 
			\draw   (303.1,138.05) .. controls (303.1,125.79) and (313.03,115.86) .. (325.29,115.86) .. controls (337.55,115.86) and (347.48,125.79) .. (347.48,138.05) .. controls (347.48,150.31) and (337.55,160.24) .. (325.29,160.24) .. controls (313.03,160.24) and (303.1,150.31) .. (303.1,138.05) -- cycle ;
			%Straight Lines [id:da10933627995525486] 
			\draw    (347.48,138.05) -- (404.9,138.05) ;
			\draw [shift={(406.9,138.05)}, rotate = 180] [color={rgb, 255:red, 0; green, 0; blue, 0 }  ][line width=0.75]    (10.93,-3.29) .. controls (6.95,-1.4) and (3.31,-0.3) .. (0,0) .. controls (3.31,0.3) and (6.95,1.4) .. (10.93,3.29)   ;
			%Shape: Ellipse [id:dp41750581663478936] 
			\draw   (406.9,138.05) .. controls (406.9,125.79) and (416.83,115.86) .. (429.09,115.86) .. controls (441.35,115.86) and (451.28,125.79) .. (451.28,138.05) .. controls (451.28,150.31) and (441.35,160.24) .. (429.09,160.24) .. controls (416.83,160.24) and (406.9,150.31) .. (406.9,138.05) -- cycle ;
			%Straight Lines [id:da5491558344199223] 
			\draw    (346.19,198.76) -- (405.48,139.46) ;
			\draw [shift={(406.9,138.05)}, rotate = 135] [color={rgb, 255:red, 0; green, 0; blue, 0 }  ][line width=0.75]    (10.93,-3.29) .. controls (6.95,-1.4) and (3.31,-0.3) .. (0,0) .. controls (3.31,0.3) and (6.95,1.4) .. (10.93,3.29)   ;
			%Straight Lines [id:da2337252989807661] 
			\draw    (451.28,138.05) -- (509.78,196.55) ;
			\draw [shift={(511.2,197.96)}, rotate = 225] [color={rgb, 255:red, 0; green, 0; blue, 0 }  ][line width=0.75]    (10.93,-3.29) .. controls (6.95,-1.4) and (3.31,-0.3) .. (0,0) .. controls (3.31,0.3) and (6.95,1.4) .. (10.93,3.29)   ;
			%Shape: Ellipse [id:dp05086276701496906] 
			\draw   (511.2,197.96) .. controls (511.2,185.7) and (521.13,175.77) .. (533.39,175.77) .. controls (545.65,175.77) and (555.58,185.7) .. (555.58,197.96) .. controls (555.58,210.22) and (545.65,220.16) .. (533.39,220.16) .. controls (521.13,220.16) and (511.2,210.22) .. (511.2,197.96) -- cycle ;
			%Shape: Ellipse [id:dp5731077109923937] 
			\draw   (313.9,267.05) .. controls (313.9,254.79) and (323.83,244.86) .. (336.09,244.86) .. controls (348.35,244.86) and (358.28,254.79) .. (358.28,267.05) .. controls (358.28,279.31) and (348.35,289.24) .. (336.09,289.24) .. controls (323.83,289.24) and (313.9,279.31) .. (313.9,267.05) -- cycle ;
			%Straight Lines [id:da4654288608549959] 
			\draw    (263.79,341.57) -- (531.57,220.98) ;
			\draw [shift={(533.39,220.16)}, rotate = 155.76] [color={rgb, 255:red, 0; green, 0; blue, 0 }  ][line width=0.75]    (10.93,-3.29) .. controls (6.95,-1.4) and (3.31,-0.3) .. (0,0) .. controls (3.31,0.3) and (6.95,1.4) .. (10.93,3.29)   ;
			%Straight Lines [id:da2613598554461163] 
			\draw    (159.49,281.66) -- (311.9,267.24) ;
			\draw [shift={(313.9,267.05)}, rotate = 174.6] [color={rgb, 255:red, 0; green, 0; blue, 0 }  ][line width=0.75]    (10.93,-3.29) .. controls (6.95,-1.4) and (3.31,-0.3) .. (0,0) .. controls (3.31,0.3) and (6.95,1.4) .. (10.93,3.29)   ;
			%Straight Lines [id:da6022699402200393] 
			\draw  [dash pattern={on 4.5pt off 4.5pt}]  (241.59,319.38) -- (334.18,289.85) ;
			\draw [shift={(336.09,289.24)}, rotate = 162.31] [color={rgb, 255:red, 0; green, 0; blue, 0 }  ][line width=0.75]    (10.93,-3.29) .. controls (6.95,-1.4) and (3.31,-0.3) .. (0,0) .. controls (3.31,0.3) and (6.95,1.4) .. (10.93,3.29)   ;
			%Straight Lines [id:da4169407271384227] 
			\draw  [dash pattern={on 4.5pt off 4.5pt}]  (429.09,160.24) -- (337.57,243.51) ;
			\draw [shift={(336.09,244.86)}, rotate = 317.7] [color={rgb, 255:red, 0; green, 0; blue, 0 }  ][line width=0.75]    (10.93,-3.29) .. controls (6.95,-1.4) and (3.31,-0.3) .. (0,0) .. controls (3.31,0.3) and (6.95,1.4) .. (10.93,3.29)   ;
			%Straight Lines [id:da1570643072347937] 
			\draw    (358.28,267.05) -- (509.37,198.79) ;
			\draw [shift={(511.2,197.96)}, rotate = 155.69] [color={rgb, 255:red, 0; green, 0; blue, 0 }  ][line width=0.75]    (10.93,-3.29) .. controls (6.95,-1.4) and (3.31,-0.3) .. (0,0) .. controls (3.31,0.3) and (6.95,1.4) .. (10.93,3.29)   ;
			%Straight Lines [id:da19330015847509174] 
			\draw    (555.58,197.96) -- (613,197.96) ;
			\draw [shift={(615,197.96)}, rotate = 180] [color={rgb, 255:red, 0; green, 0; blue, 0 }  ][line width=0.75]    (10.93,-3.29) .. controls (6.95,-1.4) and (3.31,-0.3) .. (0,0) .. controls (3.31,0.3) and (6.95,1.4) .. (10.93,3.29)   ;
			%Shape: Ellipse [id:dp5452102116023161] 
			\draw   (615,197.96) .. controls (615,185.7) and (624.93,175.77) .. (637.19,175.77) .. controls (649.45,175.77) and (659.38,185.7) .. (659.38,197.96) .. controls (659.38,210.22) and (649.45,220.16) .. (637.19,220.16) .. controls (624.93,220.16) and (615,210.22) .. (615,197.96) -- cycle ;
			
			% Text Node
			\draw (74.73,262.87) node [anchor=north west][inner sep=0.75pt]  [xscale=0.5,yscale=0.5]  {$A$};
			% Text Node
			\draw (175.73,229.87) node [anchor=north west][inner sep=0.75pt]  [xscale=0.5,yscale=0.5]  {$B$};
			% Text Node
			\draw (216.19,270.26) node [anchor=south] [inner sep=0.75pt]  [xscale=0.5,yscale=0.5]  {$C$};
			% Text Node
			\draw (187.44,315.01) node [anchor=north east] [inner sep=0.75pt]  [xscale=0.5,yscale=0.5]  {$D$};
			% Text Node
			\draw (283.19,300.26) node [anchor=south] [inner sep=0.75pt]  [xscale=0.5,yscale=0.5]  {$d$};
			% Text Node
			\draw (396.19,213.26) node [anchor=south] [inner sep=0.75pt]  [xscale=0.5,yscale=0.5]  {$d$};
			% Text Node
			\draw (259.73,148.87) node [anchor=north west][inner sep=0.75pt]  [xscale=0.5,yscale=0.5]  {$E$};
			% Text Node
			\draw (270.74,178.8) node [anchor=north west][inner sep=0.75pt]  [xscale=0.5,yscale=0.5]  {$F$};
			% Text Node
			\draw (356.74,152.8) node [anchor=north west][inner sep=0.75pt]  [xscale=0.5,yscale=0.5]  {$H$};
			% Text Node
			\draw (363.74,117.8) node [anchor=north west][inner sep=0.75pt]  [xscale=0.5,yscale=0.5]  {$G$};
			% Text Node
			\draw (477.74,141.8) node [anchor=north west][inner sep=0.75pt]  [xscale=0.5,yscale=0.5]  {$I$};
			% Text Node
			\draw (427.74,210.8) node [anchor=north west][inner sep=0.75pt]  [xscale=0.5,yscale=0.5]  {$J$};
			% Text Node
			\draw (408.74,279.8) node [anchor=north west][inner sep=0.75pt]  [xscale=0.5,yscale=0.5]  {$K$};
			% Text Node
			\draw (577.74,175.8) node [anchor=north west][inner sep=0.75pt]  [xscale=0.5,yscale=0.5]  {$L$};
			% Text Node
			\draw (31,272.21) node [anchor=north west][inner sep=0.75pt]  [xscale=0.5,yscale=0.5]  {$1$};
			% Text Node
			\draw (131,272.21) node [anchor=north west][inner sep=0.75pt]  [xscale=0.5,yscale=0.5]  {$2$};
			% Text Node
			\draw (236,212.21) node [anchor=north west][inner sep=0.75pt]  [xscale=0.5,yscale=0.5]  {$3$};
			% Text Node
			\draw (235,334.21) node [anchor=north west][inner sep=0.75pt]  [xscale=0.5,yscale=0.5]  {$4$};
			% Text Node
			\draw (319,129.21) node [anchor=north west][inner sep=0.75pt]  [xscale=0.5,yscale=0.5]  {$5$};
			% Text Node
			\draw (318,191.21) node [anchor=north west][inner sep=0.75pt]  [xscale=0.5,yscale=0.5]  {$6$};
			% Text Node
			\draw (423,129.21) node [anchor=north west][inner sep=0.75pt]  [xscale=0.5,yscale=0.5]  {$7$};
			% Text Node
			\draw (330,258.21) node [anchor=north west][inner sep=0.75pt]  [xscale=0.5,yscale=0.5]  {$8$};
			% Text Node
			\draw (527,189.21) node [anchor=north west][inner sep=0.75pt]  [xscale=0.5,yscale=0.5]  {$9$};
			% Text Node
			\draw (628,190.21) node [anchor=north west][inner sep=0.75pt]  [xscale=0.5,yscale=0.5]  {$10$};
			
			
		\end{tikzpicture}
		
	\end{center}
\end{proof}
\section{Management tools for scheduling a project}
\begin{enumerate}
	\item Project evaluation and review technique (PERT)
	\item Critical path method (CRT)
\end{enumerate}
They help in reducing project execution time.
\section{Time estimates for the exceution of each activity}

Activities which can be performed but no data is available for their time estimates such activities are called variable time or probabilistic activity.
\par On the other hand, there are activities for which the time estimates can be accurately determined.
\par Such activities are called fixed type or deterministic activities. 
\par The projects with probabilistic activities employ PERT, whereas the projects which involve deterministic actives are handled by CPM techniques.
\par PERT, developed during 1950's was first used in conjunction with planning and designing of the Polaris missile project.
\par CPM was developed by DuPont company and applied first to the construction projects in the chemical industry.

\section{PERT}
This network analysis technique gives more emphasis on events. Since the time estimates are not known, three different types of time estimates are defiend for each activity.

\begin{enumerate}
	\item \textbf{Optimistic time estimate ($ t_0 $)}\\
	This is the shortest time within which an activity can be completed.
	\item \textbf{Pessimistic time estimate ($ t_p $)}\\
	This is the maximum time required for the completion of an activity.
	\item\textbf{ Most likely time estimate ($ t_m $)}\\
	This is the estimate of time, which the activity will require under normal circumstances. The frequency curve of the activity time estimates resembles the beta distribution curve.\\
	The mean time required for the completion of the activity ($ t^{ij} $) is given by 
	\[ t^{ij}=\frac{t_0+4t_m+t_p}{6} \]
	\[ \text{Variance}=\left( \frac{t_p-t_0}{6} \right)^2 \]
\end{enumerate}

\subsection{Calculation of the critical path}

\begin{definition}[Earliest occurrence time]
	Earliest occurrence time ($ T_E $) is the earliest time at which an event can occur.
\end{definition}

\begin{definition}[Latest Occurrence time]
	Latest occurrence time ($ T_L $) is the latest time at which an event can occur.
\end{definition}

\begin{definition}[Forward pass]
	The process of calculating $ T_E$ from start event to end event. Initially $ T_E $ is taken as zero. Then it is calculated by adding the previous $ T_E $ and duration of the activity. If there are more than one activity preceding then the earliest time for each event is taken as the maximum of $ T_E $.
\end{definition}
\begin{definition}[Backward pass]
	content...
\end{definition}
\begin{definition}[Slack]
	content...
\end{definition}
\begin{definition}[Critical path]
	content...
\end{definition}


\begin{example}
	\begin{table}[]
		\centering
		\begin{tabular}{lllll}
			t0 & tm & tp & expected time & variance \\
			4 & 5 & 8 & 5.333333333 & 0.4444444444 \\
			5 & 7 & 10 & 7.166666667 & 0.6944444444 \\
			2 & 3 & 7 & 3.5 & 0.6944444444 \\
			8 & 11 & 12 & 10.66666667 & 0.4444444444 \\
			4 & 7 & 10 & 7 & 1 \\
			6 & 8 & 15 & 8.833333333 & 2.25 \\
			8 & 12 & 16 & 12 & 1.777777778 \\
			5 &  & 9 & 2.333333333 & 0.4444444444 \\
			3 &  & 7 & 1.666666667 & 0.4444444444 \\
			3 &  & 11 & 2.333333333 & 1.777777778 \\
			6 &  & 13 & 3.166666667 & 1.361111111
		\end{tabular}
	\end{table}
\end{example}

\section{CPM}
\begin{enumerate}
	\item \textbf{Earliest start time (EST)} The EST of an activity is the earliest time of the event from which it emerges. For activity $ i-j, (EST)_{ij}=T_E^i$
	\item \textbf{Earliest finish time (EFT)} The EFT of an activity is defined as the sum of the earliest time at which the activity starts and the duration of the activity. For activity $ i-j, (EFT)_{ij}=T_E^i+t^{ij} $
	\item \textbf{Latest finish time (LFT)} The latest time by which an activity should be completed is LFT, i.e. the latest time at which the event indicating the end of the activity should occur. For activity $ i-j; (LFT)_{ij}=T_L^j$
	\item \textbf{Latest start time (LST)}
	The LST for an activity is defined as teh difference between LFT and its duration. For activity $ i-j; (EST)_{ij}=T_L^j-t^{ij}$
\end{enumerate}
\begin{definition}[Float]
	It indicates the time by which the activity can be delayed without affecting the schedule of the project. Float for the activity has the same significance as slack for an event.
\end{definition}

\begin{definition}[Total float]
	It is defined as the difference between the maximum time avalible to perform an activity and the actual duration of an activity.
	\begin{align*}
		(TF)_{ij}&=(T_L^j-T_E^i)-t^{ij}\\
		&=T_L^j-(T_E^i+t^{ij})\\
		&=(LFT)_{ij}-(EFT)_{ij}
	\end{align*}
For a particular activity, if the total float is zero, it indicates that it is not affordable to delay the activity.
\end{definition}

\begin{definition}[Critical activity]
	The activity for which float is zero.
\end{definition}

\begin{definition}[Critical path]
	A path connecting all the critical activities.
\end{definition}

\begin{definition}[Free float (FF)]
	The time by which an activity can be delayed under the assumption that all events in the network occur at their eraliest is known as the Free Float.\\
	\begin{align*}
		(FF)_{ij}&=(T_E^j-T_E^i)-t^{ij}\\
		&=(T_E^j-T_E^i)-t^{ij}+T_L^{j}-T_L^j\\
		&=\{(T_L^j-T_e^i)-t^{ij}\}-(T_L^j-T_E^j)\\
		&=(TF)_{ij}-\text{slack of the head event}
	\end{align*}
\end{definition}

\begin{definition}[Independent float (IF)]
	The time by which an activity can be delayed without affecting the preceding and the succeeding activities, is known as the independent float.
	\begin{align*}
		(IF)_{ij}&=(T_E^j-T_L^i)-t^{ij}\\
		&=(T_E^j-T_L^i)-t^{ij}+T_E^i-T_E^i\\
		&=(FF)_{ij}-\text{slack of the tail event}
	\end{align*} 
Independent float can be negative, if the duration of the activity is greater than the minimum time requred to complete the activity. If negative, indepednet float is considered as zero.
\end{definition}

\subsection{Example of CPM}
\[ \begin{matrix}
	Activity & Duration & Preceeding Activity\\
	A & 15 & - \\
	B & 15& - \\
	C & 3& A \\
	D & 5&A\\
	E & 8& B,C \\
	F & 12& B,C \\
	G & 1& E\\
	H & 4& E\\
	I & 3& D,G\\
	J & 14 & F,H,I
\end{matrix} \]
Draw the network for above project and find $ EST, LST, EFT, LFT $ and the three types of float. Hence, determine the critical path.

\begin{proof}
	First draw the network diagram as follows,\\
	
	\begin{center}
	
	
	
	
	\tikzset{every picture/.style={line width=0.75pt}} %set default line width to 0.75pt        
	
	\begin{tikzpicture}[x=0.75pt,y=0.75pt,yscale=-1,xscale=1]
		%uncomment if require: \path (0,392); %set diagram left start at 0, and has height of 392
		
		%Shape: Circle [id:dp7676206325459174] 
		\draw   (1.25,180.5) .. controls (1.25,166.69) and (12.44,155.5) .. (26.25,155.5) .. controls (40.06,155.5) and (51.25,166.69) .. (51.25,180.5) .. controls (51.25,194.31) and (40.06,205.5) .. (26.25,205.5) .. controls (12.44,205.5) and (1.25,194.31) .. (1.25,180.5) -- cycle ;
		%Straight Lines [id:da36598719868013374] 
		\draw    (51.25,180.5) -- (109.27,238.52) ;
		\draw [shift={(110.68,239.93)}, rotate = 225] [color={rgb, 255:red, 0; green, 0; blue, 0 }  ][line width=0.75]    (10.93,-3.29) .. controls (6.95,-1.4) and (3.31,-0.3) .. (0,0) .. controls (3.31,0.3) and (6.95,1.4) .. (10.93,3.29)   ;
		%Straight Lines [id:da16356802190510478] 
		\draw    (51.25,180.5) -- (109.09,122.66) ;
		\draw [shift={(110.5,121.25)}, rotate = 135] [color={rgb, 255:red, 0; green, 0; blue, 0 }  ][line width=0.75]    (10.93,-3.29) .. controls (6.95,-1.4) and (3.31,-0.3) .. (0,0) .. controls (3.31,0.3) and (6.95,1.4) .. (10.93,3.29)   ;
		%Shape: Circle [id:dp4619057660637891] 
		\draw   (110.68,239.93) .. controls (110.68,226.12) and (121.87,214.93) .. (135.68,214.93) .. controls (149.49,214.93) and (160.68,226.12) .. (160.68,239.93) .. controls (160.68,253.74) and (149.49,264.93) .. (135.68,264.93) .. controls (121.87,264.93) and (110.68,253.74) .. (110.68,239.93) -- cycle ;
		%Shape: Circle [id:dp5747059121755034] 
		\draw   (110.5,121.25) .. controls (110.5,107.44) and (121.69,96.25) .. (135.5,96.25) .. controls (149.31,96.25) and (160.5,107.44) .. (160.5,121.25) .. controls (160.5,135.06) and (149.31,146.25) .. (135.5,146.25) .. controls (121.69,146.25) and (110.5,135.06) .. (110.5,121.25) -- cycle ;
		%Straight Lines [id:da7563514915187524] 
		\draw    (135.5,146.25) -- (135.67,212.93) ;
		\draw [shift={(135.68,214.93)}, rotate = 269.85] [color={rgb, 255:red, 0; green, 0; blue, 0 }  ][line width=0.75]    (10.93,-3.29) .. controls (6.95,-1.4) and (3.31,-0.3) .. (0,0) .. controls (3.31,0.3) and (6.95,1.4) .. (10.93,3.29)   ;
		%Straight Lines [id:da629661718182843] 
		\draw    (160.5,121.25) -- (218.34,63.41) ;
		\draw [shift={(219.75,62)}, rotate = 135] [color={rgb, 255:red, 0; green, 0; blue, 0 }  ][line width=0.75]    (10.93,-3.29) .. controls (6.95,-1.4) and (3.31,-0.3) .. (0,0) .. controls (3.31,0.3) and (6.95,1.4) .. (10.93,3.29)   ;
		%Straight Lines [id:da5104711086376901] 
		\draw    (160.68,239.93) -- (353.61,206.03) ;
		\draw [shift={(355.58,205.68)}, rotate = 170.03] [color={rgb, 255:red, 0; green, 0; blue, 0 }  ][line width=0.75]    (10.93,-3.29) .. controls (6.95,-1.4) and (3.31,-0.3) .. (0,0) .. controls (3.31,0.3) and (6.95,1.4) .. (10.93,3.29)   ;
		%Straight Lines [id:da04914014574391645] 
		\draw    (160.68,239.93) -- (218.52,182.09) ;
		\draw [shift={(219.93,180.68)}, rotate = 135] [color={rgb, 255:red, 0; green, 0; blue, 0 }  ][line width=0.75]    (10.93,-3.29) .. controls (6.95,-1.4) and (3.31,-0.3) .. (0,0) .. controls (3.31,0.3) and (6.95,1.4) .. (10.93,3.29)   ;
		%Shape: Circle [id:dp17806981866288485] 
		\draw   (219.93,180.68) .. controls (219.93,166.87) and (231.12,155.68) .. (244.93,155.68) .. controls (258.74,155.68) and (269.93,166.87) .. (269.93,180.68) .. controls (269.93,194.49) and (258.74,205.68) .. (244.93,205.68) .. controls (231.12,205.68) and (219.93,194.49) .. (219.93,180.68) -- cycle ;
		%Straight Lines [id:da37111142997282287] 
		\draw    (269.93,180.68) -- (328.58,180.68) ;
		\draw [shift={(330.58,180.68)}, rotate = 180] [color={rgb, 255:red, 0; green, 0; blue, 0 }  ][line width=0.75]    (10.93,-3.29) .. controls (6.95,-1.4) and (3.31,-0.3) .. (0,0) .. controls (3.31,0.3) and (6.95,1.4) .. (10.93,3.29)   ;
		%Straight Lines [id:da15698146477275454] 
		\draw    (244.93,155.68) -- (244.76,89) ;
		\draw [shift={(244.75,87)}, rotate = 89.85] [color={rgb, 255:red, 0; green, 0; blue, 0 }  ][line width=0.75]    (10.93,-3.29) .. controls (6.95,-1.4) and (3.31,-0.3) .. (0,0) .. controls (3.31,0.3) and (6.95,1.4) .. (10.93,3.29)   ;
		%Shape: Circle [id:dp46032212107048154] 
		\draw   (219.75,62) .. controls (219.75,48.19) and (230.94,37) .. (244.75,37) .. controls (258.56,37) and (269.75,48.19) .. (269.75,62) .. controls (269.75,75.81) and (258.56,87) .. (244.75,87) .. controls (230.94,87) and (219.75,75.81) .. (219.75,62) -- cycle ;
		%Shape: Circle [id:dp4786503933434014] 
		\draw   (330.58,180.68) .. controls (330.58,166.87) and (341.77,155.68) .. (355.58,155.68) .. controls (369.39,155.68) and (380.58,166.87) .. (380.58,180.68) .. controls (380.58,194.49) and (369.39,205.68) .. (355.58,205.68) .. controls (341.77,205.68) and (330.58,194.49) .. (330.58,180.68) -- cycle ;
		%Straight Lines [id:da5538781087827809] 
		\draw    (269.75,62) -- (354.23,154.21) ;
		\draw [shift={(355.58,155.68)}, rotate = 227.5] [color={rgb, 255:red, 0; green, 0; blue, 0 }  ][line width=0.75]    (10.93,-3.29) .. controls (6.95,-1.4) and (3.31,-0.3) .. (0,0) .. controls (3.31,0.3) and (6.95,1.4) .. (10.93,3.29)   ;
		%Straight Lines [id:da7742266288631681] 
		\draw    (380.58,180.68) -- (439.23,180.68) ;
		\draw [shift={(441.23,180.68)}, rotate = 180] [color={rgb, 255:red, 0; green, 0; blue, 0 }  ][line width=0.75]    (10.93,-3.29) .. controls (6.95,-1.4) and (3.31,-0.3) .. (0,0) .. controls (3.31,0.3) and (6.95,1.4) .. (10.93,3.29)   ;
		%Shape: Circle [id:dp9151534822094307] 
		\draw   (441.23,180.68) .. controls (441.23,166.87) and (452.42,155.68) .. (466.23,155.68) .. controls (480.04,155.68) and (491.23,166.87) .. (491.23,180.68) .. controls (491.23,194.49) and (480.04,205.68) .. (466.23,205.68) .. controls (452.42,205.68) and (441.23,194.49) .. (441.23,180.68) -- cycle ;
		
		% Text Node
		\draw (78.88,147.48) node [anchor=south east] [inner sep=0.75pt]    {$A$};
		% Text Node
		\draw (78.97,213.62) node [anchor=north east] [inner sep=0.75pt]    {$B$};
		% Text Node
		\draw (188.13,88.23) node [anchor=south east] [inner sep=0.75pt]    {$D$};
		% Text Node
		\draw (137.59,180.59) node [anchor=west] [inner sep=0.75pt]    {$C$};
		% Text Node
		\draw (188.31,206.91) node [anchor=south east] [inner sep=0.75pt]    {$E$};
		% Text Node
		\draw (260.13,226.21) node [anchor=north west][inner sep=0.75pt]    {$F$};
		% Text Node
		\draw (246.84,121.34) node [anchor=west] [inner sep=0.75pt]    {$G$};
		% Text Node
		\draw (300.25,177.28) node [anchor=south] [inner sep=0.75pt]    {$H$};
		% Text Node
		\draw (314.66,105.44) node [anchor=south west] [inner sep=0.75pt]    {$I$};
		% Text Node
		\draw (410.9,177.28) node [anchor=south] [inner sep=0.75pt]    {$J$};
		% Text Node
		\draw (16,172.4) node [anchor=north west][inner sep=0.75pt]    {$10$};
		% Text Node
		\draw (125,112.4) node [anchor=north west][inner sep=0.75pt]    {$20$};
		% Text Node
		\draw (125,232.4) node [anchor=north west][inner sep=0.75pt]    {$30$};
		% Text Node
		\draw (235,171.4) node [anchor=north west][inner sep=0.75pt]    {$40$};
		% Text Node
		\draw (235,53.4) node [anchor=north west][inner sep=0.75pt]    {$50$};
		% Text Node
		\draw (346,172.4) node [anchor=north west][inner sep=0.75pt]    {$60$};
		% Text Node
		\draw (456,173.4) node [anchor=north west][inner sep=0.75pt]    {$70$};
		
		
	\end{tikzpicture}
	
	\end{center}
$ T_E=54 $
% Please add the following required packages to your document preamble:
% \usepackage{booktabs}
% \usepackage{graphicx}
\begin{table}[]
	\centering
	\resizebox{\textwidth}{!}{%
		\begin{tabular}{@{}cccccccc@{}}
			\toprule
			Events & $t^{ij}$ & $EST=T_E^i$ & $LFT=T_L^j$ & $EFT=T_E^i+t_{ij}$ & $LST=T_L^j-t^{ij}$ & $TF=LFT-EFT$ & $FF=TF-(T_L^j-T_E^j)$ \\ \midrule
			10-20 & 15 & 0 & 15 & 15 & 0 & 0 & 0-(15-15)=0 \\
			10-30 & 15 & 0 & 18 & 15 & 3 & 3 & 3-(0)=3 \\
			20-30 & 3 & 15 & 18 & 18 & 15 & 0 & 0-(18-18)=0 \\
			20-50 & 5 & 15 & 37 & 20 & 32 & 17 & 7 \\
			30-40 & 6 & 18 & 26 & 24 & 20 & 0 & 0 \\
			30-60 & 12 & 18 & 40 & 30 & 28 & 10 & 10 \\
			40-50 & 1 & 26 & 37 & 27 & 36 & 10 & 0 \\
			40-60 & 14 & 26 & 40 & 40 & 26 & 0 & 0 \\
			50-60 & 3 & 27 & 40 & 30 & 37 & 10 & 10 \\
			60-70 & 14 & 40 & 54 & 54 & 40 & 0 & 0 \\ \bottomrule
		\end{tabular}%
	}
\end{table}
\FloatBarrier
Now we find total float and free float.\\
Total float=$ LFT-EFT$. If total float is zero it implies its a critical activity.\\
Free float = $ TF- $ Head slack\\
Independent float = FF - Tail slack\\
Head slack = $ T_L^j-T_E^j $\\
Tail slack = $ T^i-T_E^i $
\end{proof}


\section{Project crashing}
Two types

\begin{definition}[Project crashing]
	Process of shortening the critical path to achieve completion of the project earlier is called project crashing.
\end{definition}
There are two types of cost,
\begin{enumerate}
	\item \textbf{ Direct cost: } It includes cost of materials, machinery, man hours etc. When we estimate the duration of various activities in the project it is assumed that various activities in the project it is assumed that the normal amount of labour and machine required to complete these activities. But when we want to complete the project in the shorter period, than the critical path we will need to employ more resources. Hence direct cost will increase. \textbf{yeah this makes no sense dw}
	\item\textbf{ Indirect cost: } It includes rent, overheads, administrative costs. Indirect costs vary with time. They are expressed on per day or per week basis
\end{enumerate}

Total cost = Direct cost + indirect cost

\textbf{Normal time: }Under normal circumstances.
\textbf{Crash time: }Minimum possible time in which activity can be completed.
\textbf{Crash cost: }The direct cost associated with crash time. When the activity is crashed its direct cost will increase.

Crash time < Normal time\\
Crash cost > Normal cost

\begin{definition}[Crash slope/Cost slope]
	The increase in direct cost per day. It is Crash cost - normal cost divided by normal time - crash time
	\[ \frac{\text{Crash cost}-\text{Normal cost}}{\text{Normal time}-\text{Crash time}} \]
\end{definition}

\subsection{Procedure of crashing}
\begin{enumerate}
	\item Find the normal critical path and identify the critical activities.
	\item Calculate the cost slope for the critical activities.
	\item Rank the critical activities in the ascending order of their cost slopes.
	\item Crash critical activities as per the ranking (i.e. the activity with the least cost slope first) to its maximum possible extent. Calculate the revised cost by adding the cost of crashing and subtracting the indirect cost; both proportion to the number of days crashed.
	\item As the critical path duration gets reduced in step 4, some other path may also become critical, i.e. there may be parallel critical paths. This implies that the project duration now has to be reduced simultaneously by crashing the activities on the parallel critical paths.
	\item After crashing the activities in the above manner, a point is reached where further crashing is either not possible or it does not result in reduction of project duration or cost. For different project or durations, project cost should be calculated.
\end{enumerate}

\begin{example}
\begin{table}[!htp]
	\centering
	
		\begin{tabular}{@{}cccc@{}}
			\toprule
			Activity & $t^{ij}$ & Activity & $t^{ij}$ \\ \midrule
			1-2 & 2 & 4-6 & 3 \\
			1-3 & 2 & 5-8 & 1 \\
			1-4 & 1 & 6-9 & 5 \\
			2-5 & 4 & 7-8 & 4 \\
			3-6 & 8 & 8-9 & 3 \\
			3-7 & 5 &  &  \\ \bottomrule
		\end{tabular}%
	
\end{table}
Draw a network for this project. Find total float for each activity. Hence determine critical path for network.
\end{example}
\begin{proof}
	$ T_E=T_L=15 $
\end{proof}




\begin{example}
	Indirect cost is Rs. $ 70 $ per day.
	\begin{table}[!htp]
		\centering
		\begin{tabular}{@{}ccccc@{}}
			\toprule
			Activity & Normal time & Crash time & Normal cost & Crash cost \\ \midrule
			1-2 & 8 & 6 & 100 & 200 \\
			1-3 & 4 & 2 & 150 & 350 \\
			2-4 & 2 & 1 & 50 & 90 \\
			2-5 & 10 & 5 & 100 & 400 \\
			3-4 & 5 & 1 & 100 & 200 \\
			4-5 & 3 & 1 & 80 & 100 \\ \bottomrule
		\end{tabular}
	\end{table}

\end{example}
\begin{proof}
	Consider first the network diagram,
	
	\tikzset{every picture/.style={line width=0.75pt}} %set default line width to 0.75pt        
	
	\begin{tikzpicture}[x=0.75pt,y=0.75pt,yscale=-1,xscale=1]
		%uncomment if require: \path (0,341); %set diagram left start at 0, and has height of 341
		
		%Flowchart: Connector [id:dp5017991924606011] 
		\draw   (0.8,150.4) .. controls (0.8,136.48) and (12.08,125.2) .. (26,125.2) .. controls (39.92,125.2) and (51.2,136.48) .. (51.2,150.4) .. controls (51.2,164.32) and (39.92,175.6) .. (26,175.6) .. controls (12.08,175.6) and (0.8,164.32) .. (0.8,150.4) -- cycle ;
		%Straight Lines [id:da4365682232982502] 
		\draw    (51.2,150.4) -- (118.43,83.17) ;
		\draw [shift={(119.84,81.76)}, rotate = 135] [color={rgb, 255:red, 0; green, 0; blue, 0 }  ][line width=0.75]    (10.93,-3.29) .. controls (6.95,-1.4) and (3.31,-0.3) .. (0,0) .. controls (3.31,0.3) and (6.95,1.4) .. (10.93,3.29)   ;
		%Straight Lines [id:da9853527385509557] 
		\draw    (51.2,150.4) -- (118.31,217.51) ;
		\draw [shift={(119.72,218.92)}, rotate = 225] [color={rgb, 255:red, 0; green, 0; blue, 0 }  ][line width=0.75]    (10.93,-3.29) .. controls (6.95,-1.4) and (3.31,-0.3) .. (0,0) .. controls (3.31,0.3) and (6.95,1.4) .. (10.93,3.29)   ;
		%Flowchart: Connector [id:dp3934684464134668] 
		\draw   (119.84,81.76) .. controls (119.84,67.84) and (131.12,56.56) .. (145.04,56.56) .. controls (158.96,56.56) and (170.24,67.84) .. (170.24,81.76) .. controls (170.24,95.68) and (158.96,106.96) .. (145.04,106.96) .. controls (131.12,106.96) and (119.84,95.68) .. (119.84,81.76) -- cycle ;
		%Flowchart: Connector [id:dp1485057734469888] 
		\draw   (119.72,218.92) .. controls (119.72,205) and (131,193.72) .. (144.92,193.72) .. controls (158.84,193.72) and (170.12,205) .. (170.12,218.92) .. controls (170.12,232.84) and (158.84,244.12) .. (144.92,244.12) .. controls (131,244.12) and (119.72,232.84) .. (119.72,218.92) -- cycle ;
		%Straight Lines [id:da5287011576058043] 
		\draw    (170.12,218.92) -- (233.72,218.92) ;
		\draw [shift={(235.72,218.92)}, rotate = 180] [color={rgb, 255:red, 0; green, 0; blue, 0 }  ][line width=0.75]    (10.93,-3.29) .. controls (6.95,-1.4) and (3.31,-0.3) .. (0,0) .. controls (3.31,0.3) and (6.95,1.4) .. (10.93,3.29)   ;
		%Flowchart: Connector [id:dp2323421555185401] 
		\draw   (235.72,218.92) .. controls (235.72,205) and (247,193.72) .. (260.92,193.72) .. controls (274.84,193.72) and (286.12,205) .. (286.12,218.92) .. controls (286.12,232.84) and (274.84,244.12) .. (260.92,244.12) .. controls (247,244.12) and (235.72,232.84) .. (235.72,218.92) -- cycle ;
		%Straight Lines [id:da9247941297234821] 
		\draw    (170.24,81.76) -- (259.66,192.17) ;
		\draw [shift={(260.92,193.72)}, rotate = 230.99] [color={rgb, 255:red, 0; green, 0; blue, 0 }  ][line width=0.75]    (10.93,-3.29) .. controls (6.95,-1.4) and (3.31,-0.3) .. (0,0) .. controls (3.31,0.3) and (6.95,1.4) .. (10.93,3.29)   ;
		%Straight Lines [id:da689465144796324] 
		\draw    (170.24,81.76) -- (233.84,81.76) ;
		\draw [shift={(235.84,81.76)}, rotate = 180] [color={rgb, 255:red, 0; green, 0; blue, 0 }  ][line width=0.75]    (10.93,-3.29) .. controls (6.95,-1.4) and (3.31,-0.3) .. (0,0) .. controls (3.31,0.3) and (6.95,1.4) .. (10.93,3.29)   ;
		%Flowchart: Connector [id:dp6413729849919538] 
		\draw   (235.84,81.76) .. controls (235.84,67.84) and (247.12,56.56) .. (261.04,56.56) .. controls (274.96,56.56) and (286.24,67.84) .. (286.24,81.76) .. controls (286.24,95.68) and (274.96,106.96) .. (261.04,106.96) .. controls (247.12,106.96) and (235.84,95.68) .. (235.84,81.76) -- cycle ;
		%Straight Lines [id:da9781789544489394] 
		\draw    (260.92,193.72) -- (261.04,108.96) ;
		\draw [shift={(261.04,106.96)}, rotate = 90.08] [color={rgb, 255:red, 0; green, 0; blue, 0 }  ][line width=0.75]    (10.93,-3.29) .. controls (6.95,-1.4) and (3.31,-0.3) .. (0,0) .. controls (3.31,0.3) and (6.95,1.4) .. (10.93,3.29)   ;
		
		% Text Node
		\draw (21,142.4) node [anchor=north west][inner sep=0.75pt]    {$1$};
		% Text Node
		\draw (83.52,112.68) node [anchor=south east] [inner sep=0.75pt]    {$A$};
		% Text Node
		\draw (139,74.4) node [anchor=north west][inner sep=0.75pt]    {$2$};
		% Text Node
		\draw (83.46,188.06) node [anchor=north east] [inner sep=0.75pt]    {$B$};
		% Text Node
		\draw (140,211.4) node [anchor=north west][inner sep=0.75pt]    {$3$};
		% Text Node
		\draw (256,211.4) node [anchor=north west][inner sep=0.75pt]    {$4$};
		% Text Node
		\draw (217.58,134.34) node [anchor=south west] [inner sep=0.75pt]    {$C$};
		% Text Node
		\draw (202.92,222.32) node [anchor=north] [inner sep=0.75pt]    {$E$};
		% Text Node
		\draw (203.04,78.36) node [anchor=south] [inner sep=0.75pt]    {$D$};
		% Text Node
		\draw (256,73.4) node [anchor=north west][inner sep=0.75pt]    {$5$};
		% Text Node
		\draw (262.98,150.34) node [anchor=west] [inner sep=0.75pt]    {$F$};
		
		
	\end{tikzpicture}\\
	Critical path is $ 1-2-5:A-D=18 $days.\\
	Other paths are $ 1-3-4-5:B-E-F=12$days\\
	$ 1-2-4-5:A-C-F=13 $days.\\
	Total normal cost is $=DC+IDC$
	\begin{align*}
		&=DC+IDC\\
		&=580+70*8
		&=1140
	\end{align*}
Now see the cost slope
\begin{align*}
	\text{Cost slope}&=\frac{\Delta \text{time}}{\Delta \text{cost}}
\end{align*}
\begin{table}[!htp]
	\centering
	\begin{tabular}{cccc}
		\hline
		Activity & \begin{tabular}[c]{@{}c@{}}Maximum \\ crashable days\end{tabular} & Cost Slope / day & Rank \\ \hline
		A(1-2) & 8-6=2 & 50 & 4 \\
		B(1-3) & 4-2=2 & 100 & 6 \\
		C(2-4) & 2-1=1 & 40 & 3 \\
		D(2-5) & 10-5=5 & 60 & 5 \\
		E(3-4) & 5-1=4 & 25 & 2 \\
		F(4-5) & 3-1=2 & 10 & 1 \\ \hline
	\end{tabular}
\end{table}

\begin{table}[]
	\centering
	\begin{tabular}{ccccccc}
		\hline
		\multirow{2}{*}{Crashing} & \multicolumn{3}{c}{Paths} & \multicolumn{3}{c}{Cost (Rs.)} \\ \cline{2-7} 
		& A-D & A-C-F & B-E-F & D.C. & I.D.C & T.C. \\ \hline
		Before Crashing & 18 & 13 & 12 & 580 & 1266 & 1840 \\
		\begin{tabular}[c]{@{}c@{}}1st Crashing A\\ by 1\end{tabular} & \begin{tabular}[c]{@{}c@{}}18-1=\\ 17\end{tabular} & \begin{tabular}[c]{@{}c@{}}13-1=\\ 12\end{tabular} & 12 & \begin{tabular}[c]{@{}c@{}}580+(50*1)\\ =630\end{tabular} & \begin{tabular}[c]{@{}c@{}}17*70\\ =1190\end{tabular} & 1820 \\
		\begin{tabular}[c]{@{}c@{}}2nd Crashing A\\ by 2\end{tabular} & \begin{tabular}[c]{@{}c@{}}17-1=\\ 16\end{tabular} & \begin{tabular}[c]{@{}c@{}}12-1=\\ 11\end{tabular} & 12 & \begin{tabular}[c]{@{}c@{}}630+(50)\\ =680\end{tabular} & \begin{tabular}[c]{@{}c@{}}16*70\\ =1120\end{tabular} & 1800 \\
		\begin{tabular}[c]{@{}c@{}}3rd Crashing D\\ by 1\end{tabular} & \begin{tabular}[c]{@{}c@{}}16-1=\\ 15\end{tabular} & 11 & 12 & \begin{tabular}[c]{@{}c@{}}680+60\\ =740\end{tabular} & \begin{tabular}[c]{@{}c@{}}15*70\\ =1050\end{tabular} & 1790 \\
		\begin{tabular}[c]{@{}c@{}}4th Crashing D\\ by 1\end{tabular} & \begin{tabular}[c]{@{}c@{}}15-1=\\ 14\end{tabular} & 11 & 12 & \begin{tabular}[c]{@{}c@{}}740+60\\ =800\end{tabular} & \begin{tabular}[c]{@{}c@{}}14*70\\ =980\end{tabular} & 1780 \\
		\begin{tabular}[c]{@{}c@{}}5h Crashing D\\ by 1\end{tabular} & 13 & 11 & 12 & 860 & 910 & 1770 \\
		\begin{tabular}[c]{@{}c@{}}6th Crashing D \\ by 1\end{tabular} & 12 & 11 & 12 & 920 & 840 & 1760 \\
		\begin{tabular}[c]{@{}c@{}}7th Crashing D \\ by 1 and F by 1\end{tabular} & 11 & 10 & 11 & \begin{tabular}[c]{@{}c@{}}920+60+10\\ =990\end{tabular} & \begin{tabular}[c]{@{}c@{}}11*70\\ =770\end{tabular} & 1760 \\ \hline
	\end{tabular}
\end{table}
Minimum total cost = optimum project cost = 1760

\end{proof}

\section{something}

...which have occurred due to re planning or re scheduling of ...\\
Updating can be done in two ways,
\begin{enumerate}
	\item Use the revised time estimates of incomplete activities and calculate from the completion of time of each event in the usual manner to know the project completion time. and represent all the activities already finished by elapsed time. Events in the revised network diagram are re numbered. And the completion of the completed activities (?????) are taken as the revised time.
\end{enumerate}

\begin{example}
	After 15 days working the following progress is noted of an erection job.
	\begin{enumerate}
		\item Activity 1-2,1-3 and, 1-4 completed as per the original schedule.
		\item Act 2-4, is in progress and will be completed in 3 more days
		\item Act 3-6 is in progress and require 18 days more.
		\item Act 6-7 appears to present some problem and its new estimated time is 12 days.
		\item Act 6-8 can be completed in 5 days instead of 7 days.
	\end{enumerate}
\end{example}
\begin{proof}
	Consider the network diagram
	\begin{center}
		
		
		\tikzset{every picture/.style={line width=0.75pt}} %set default line width to 0.75pt        
		
		\begin{tikzpicture}[x=0.75pt,y=0.75pt,yscale=-1,xscale=1]
			%uncomment if require: \path (0,341); %set diagram left start at 0, and has height of 341
			
			%Flowchart: Connector [id:dp5017991924606011] 
			\draw   (0.8,150.4) .. controls (0.8,136.48) and (12.08,125.2) .. (26,125.2) .. controls (39.92,125.2) and (51.2,136.48) .. (51.2,150.4) .. controls (51.2,164.32) and (39.92,175.6) .. (26,175.6) .. controls (12.08,175.6) and (0.8,164.32) .. (0.8,150.4) -- cycle ;
			%Straight Lines [id:da4365682232982502] 
			\draw    (51.2,150.4) -- (118.43,83.17) ;
			\draw [shift={(119.84,81.76)}, rotate = 135] [color={rgb, 255:red, 0; green, 0; blue, 0 }  ][line width=0.75]    (10.93,-3.29) .. controls (6.95,-1.4) and (3.31,-0.3) .. (0,0) .. controls (3.31,0.3) and (6.95,1.4) .. (10.93,3.29)   ;
			%Straight Lines [id:da9853527385509557] 
			\draw    (51.2,150.4) -- (118.31,217.51) ;
			\draw [shift={(119.72,218.92)}, rotate = 225] [color={rgb, 255:red, 0; green, 0; blue, 0 }  ][line width=0.75]    (10.93,-3.29) .. controls (6.95,-1.4) and (3.31,-0.3) .. (0,0) .. controls (3.31,0.3) and (6.95,1.4) .. (10.93,3.29)   ;
			%Flowchart: Connector [id:dp3934684464134668] 
			\draw   (119.84,81.76) .. controls (119.84,67.84) and (131.12,56.56) .. (145.04,56.56) .. controls (158.96,56.56) and (170.24,67.84) .. (170.24,81.76) .. controls (170.24,95.68) and (158.96,106.96) .. (145.04,106.96) .. controls (131.12,106.96) and (119.84,95.68) .. (119.84,81.76) -- cycle ;
			%Flowchart: Connector [id:dp1485057734469888] 
			\draw   (119.72,218.92) .. controls (119.72,205) and (131,193.72) .. (144.92,193.72) .. controls (158.84,193.72) and (170.12,205) .. (170.12,218.92) .. controls (170.12,232.84) and (158.84,244.12) .. (144.92,244.12) .. controls (131,244.12) and (119.72,232.84) .. (119.72,218.92) -- cycle ;
			%Straight Lines [id:da5287011576058043] 
			\draw    (170.12,218.92) -- (233.72,218.92) ;
			\draw [shift={(235.72,218.92)}, rotate = 180] [color={rgb, 255:red, 0; green, 0; blue, 0 }  ][line width=0.75]    (10.93,-3.29) .. controls (6.95,-1.4) and (3.31,-0.3) .. (0,0) .. controls (3.31,0.3) and (6.95,1.4) .. (10.93,3.29)   ;
			%Flowchart: Connector [id:dp2323421555185401] 
			\draw   (235.72,218.92) .. controls (235.72,205) and (247,193.72) .. (260.92,193.72) .. controls (274.84,193.72) and (286.12,205) .. (286.12,218.92) .. controls (286.12,232.84) and (274.84,244.12) .. (260.92,244.12) .. controls (247,244.12) and (235.72,232.84) .. (235.72,218.92) -- cycle ;
			%Straight Lines [id:da689465144796324] 
			\draw    (170.24,81.76) -- (233.84,81.76) ;
			\draw [shift={(235.84,81.76)}, rotate = 180] [color={rgb, 255:red, 0; green, 0; blue, 0 }  ][line width=0.75]    (10.93,-3.29) .. controls (6.95,-1.4) and (3.31,-0.3) .. (0,0) .. controls (3.31,0.3) and (6.95,1.4) .. (10.93,3.29)   ;
			%Flowchart: Connector [id:dp6413729849919538] 
			\draw   (235.84,81.76) .. controls (235.84,67.84) and (247.12,56.56) .. (261.04,56.56) .. controls (274.96,56.56) and (286.24,67.84) .. (286.24,81.76) .. controls (286.24,95.68) and (274.96,106.96) .. (261.04,106.96) .. controls (247.12,106.96) and (235.84,95.68) .. (235.84,81.76) -- cycle ;
			%Straight Lines [id:da08380556654265248] 
			\draw    (51.2,150.4) -- (114.8,150.4) ;
			\draw [shift={(116.8,150.4)}, rotate = 180] [color={rgb, 255:red, 0; green, 0; blue, 0 }  ][line width=0.75]    (10.93,-3.29) .. controls (6.95,-1.4) and (3.31,-0.3) .. (0,0) .. controls (3.31,0.3) and (6.95,1.4) .. (10.93,3.29)   ;
			%Flowchart: Connector [id:dp7446531983688254] 
			\draw   (116.8,150.4) .. controls (116.8,136.48) and (128.08,125.2) .. (142,125.2) .. controls (155.92,125.2) and (167.2,136.48) .. (167.2,150.4) .. controls (167.2,164.32) and (155.92,175.6) .. (142,175.6) .. controls (128.08,175.6) and (116.8,164.32) .. (116.8,150.4) -- cycle ;
			%Straight Lines [id:da1511415258983968] 
			\draw    (167.2,150.4) -- (230.8,150.4) ;
			\draw [shift={(232.8,150.4)}, rotate = 180] [color={rgb, 255:red, 0; green, 0; blue, 0 }  ][line width=0.75]    (10.93,-3.29) .. controls (6.95,-1.4) and (3.31,-0.3) .. (0,0) .. controls (3.31,0.3) and (6.95,1.4) .. (10.93,3.29)   ;
			%Flowchart: Connector [id:dp8705506203881077] 
			\draw   (232.8,150.4) .. controls (232.8,136.48) and (244.08,125.2) .. (258,125.2) .. controls (271.92,125.2) and (283.2,136.48) .. (283.2,150.4) .. controls (283.2,164.32) and (271.92,175.6) .. (258,175.6) .. controls (244.08,175.6) and (232.8,164.32) .. (232.8,150.4) -- cycle ;
			%Straight Lines [id:da6335929354938967] 
			\draw    (260.92,193.72) -- (258.32,177.57) ;
			\draw [shift={(258,175.6)}, rotate = 80.85] [color={rgb, 255:red, 0; green, 0; blue, 0 }  ][line width=0.75]    (10.93,-3.29) .. controls (6.95,-1.4) and (3.31,-0.3) .. (0,0) .. controls (3.31,0.3) and (6.95,1.4) .. (10.93,3.29)   ;
			%Straight Lines [id:da26211828707625884] 
			\draw    (144.92,193.72) -- (142.32,177.57) ;
			\draw [shift={(142,175.6)}, rotate = 80.85] [color={rgb, 255:red, 0; green, 0; blue, 0 }  ][line width=0.75]    (10.93,-3.29) .. controls (6.95,-1.4) and (3.31,-0.3) .. (0,0) .. controls (3.31,0.3) and (6.95,1.4) .. (10.93,3.29)   ;
			%Straight Lines [id:da08975872326436352] 
			\draw    (283.2,150.4) -- (350.31,217.51) ;
			\draw [shift={(351.72,218.92)}, rotate = 225] [color={rgb, 255:red, 0; green, 0; blue, 0 }  ][line width=0.75]    (10.93,-3.29) .. controls (6.95,-1.4) and (3.31,-0.3) .. (0,0) .. controls (3.31,0.3) and (6.95,1.4) .. (10.93,3.29)   ;
			%Flowchart: Connector [id:dp41780010498780573] 
			\draw   (351.72,218.92) .. controls (351.72,205) and (363,193.72) .. (376.92,193.72) .. controls (390.84,193.72) and (402.12,205) .. (402.12,218.92) .. controls (402.12,232.84) and (390.84,244.12) .. (376.92,244.12) .. controls (363,244.12) and (351.72,232.84) .. (351.72,218.92) -- cycle ;
			%Straight Lines [id:da9226665538697332] 
			\draw    (286.12,218.92) -- (349.72,218.92) ;
			\draw [shift={(351.72,218.92)}, rotate = 180] [color={rgb, 255:red, 0; green, 0; blue, 0 }  ][line width=0.75]    (10.93,-3.29) .. controls (6.95,-1.4) and (3.31,-0.3) .. (0,0) .. controls (3.31,0.3) and (6.95,1.4) .. (10.93,3.29)   ;
			%Straight Lines [id:da6832237097024039] 
			\draw    (145.04,106.96) -- (142.33,123.23) ;
			\draw [shift={(142,125.2)}, rotate = 279.46] [color={rgb, 255:red, 0; green, 0; blue, 0 }  ][line width=0.75]    (10.93,-3.29) .. controls (6.95,-1.4) and (3.31,-0.3) .. (0,0) .. controls (3.31,0.3) and (6.95,1.4) .. (10.93,3.29)   ;
			%Straight Lines [id:da07015174051168338] 
			\draw    (261.04,106.96) -- (258.33,123.23) ;
			\draw [shift={(258,125.2)}, rotate = 279.46] [color={rgb, 255:red, 0; green, 0; blue, 0 }  ][line width=0.75]    (10.93,-3.29) .. controls (6.95,-1.4) and (3.31,-0.3) .. (0,0) .. controls (3.31,0.3) and (6.95,1.4) .. (10.93,3.29)   ;
			
			% Text Node
			\draw (21,142.4) node [anchor=north west][inner sep=0.75pt]    {$1$};
			% Text Node
			\draw (83.52,112.68) node [anchor=south east] [inner sep=0.75pt]    {$9$};
			% Text Node
			\draw (139,74.4) node [anchor=north west][inner sep=0.75pt]    {$2$};
			% Text Node
			\draw (83.46,188.06) node [anchor=north east] [inner sep=0.75pt]    {$10$};
			% Text Node
			\draw (140,211.4) node [anchor=north west][inner sep=0.75pt]    {$3$};
			% Text Node
			\draw (256,211.4) node [anchor=north west][inner sep=0.75pt]    {$6$};
			% Text Node
			\draw (202.92,222.32) node [anchor=north] [inner sep=0.75pt]    {$12$};
			% Text Node
			\draw (203.04,78.36) node [anchor=south] [inner sep=0.75pt]    {$18$};
			% Text Node
			\draw (256,73.4) node [anchor=north west][inner sep=0.75pt]    {$5$};
			% Text Node
			\draw (137,142.4) node [anchor=north west][inner sep=0.75pt]    {$4$};
			% Text Node
			\draw (94.52,146.68) node [anchor=south east] [inner sep=0.75pt]    {$6$};
			% Text Node
			\draw (252,140.4) node [anchor=north west][inner sep=0.75pt]    {$7$};
			% Text Node
			\draw (192,133.4) node [anchor=north west][inner sep=0.75pt]    {$\infty $};
			% Text Node
			\draw (160.92,175.32) node [anchor=north] [inner sep=0.75pt]    {$5$};
			% Text Node
			\draw (279.92,177.32) node [anchor=north] [inner sep=0.75pt]    {$7$};
			% Text Node
			\draw (318.92,222.32) node [anchor=north] [inner sep=0.75pt]    {$7$};
			% Text Node
			\draw (377.92,209.32) node [anchor=north] [inner sep=0.75pt]    {$8$};
			% Text Node
			\draw (158.92,110.32) node [anchor=north] [inner sep=0.75pt]    {$7$};
			% Text Node
			\draw (274.92,110.32) node [anchor=north] [inner sep=0.75pt]    {$8$};
			% Text Node
			\draw (317.92,160.32) node [anchor=north] [inner sep=0.75pt]    {$6$};
			
			
		\end{tikzpicture}
		
	\end{center}
$ T_E=42,T_L=42 $ new duration of $ 2-4 $ is $ 15+3-9=9 $
\begin{enumerate}
	\item Act 2-4 needs 15+3-9=9 days instead of 7.
	\item Act 3-6 needs 15+18-10 = 23 days instead of 12.
	\item Act 6-7=12 days instead of 7 days.
	\item Act 6-8=5 days instead of 7
	\item Act 2-5, 4-7, 5-7, 7-8, 3-4 remain as it is.
	
\end{enumerate}
New network,
\begin{center}
	
	
	\tikzset{every picture/.style={line width=0.75pt}} %set default line width to 0.75pt        
	
	\begin{tikzpicture}[x=0.75pt,y=0.75pt,yscale=-1,xscale=1]
		%uncomment if require: \path (0,341); %set diagram left start at 0, and has height of 341
		
		%Flowchart: Connector [id:dp5017991924606011] 
		\draw   (0.8,150.4) .. controls (0.8,136.48) and (12.08,125.2) .. (26,125.2) .. controls (39.92,125.2) and (51.2,136.48) .. (51.2,150.4) .. controls (51.2,164.32) and (39.92,175.6) .. (26,175.6) .. controls (12.08,175.6) and (0.8,164.32) .. (0.8,150.4) -- cycle ;
		%Straight Lines [id:da4365682232982502] 
		\draw    (51.2,150.4) -- (118.43,83.17) ;
		\draw [shift={(119.84,81.76)}, rotate = 135] [color={rgb, 255:red, 0; green, 0; blue, 0 }  ][line width=0.75]    (10.93,-3.29) .. controls (6.95,-1.4) and (3.31,-0.3) .. (0,0) .. controls (3.31,0.3) and (6.95,1.4) .. (10.93,3.29)   ;
		%Straight Lines [id:da9853527385509557] 
		\draw    (51.2,150.4) -- (118.31,217.51) ;
		\draw [shift={(119.72,218.92)}, rotate = 225] [color={rgb, 255:red, 0; green, 0; blue, 0 }  ][line width=0.75]    (10.93,-3.29) .. controls (6.95,-1.4) and (3.31,-0.3) .. (0,0) .. controls (3.31,0.3) and (6.95,1.4) .. (10.93,3.29)   ;
		%Flowchart: Connector [id:dp3934684464134668] 
		\draw   (119.84,81.76) .. controls (119.84,67.84) and (131.12,56.56) .. (145.04,56.56) .. controls (158.96,56.56) and (170.24,67.84) .. (170.24,81.76) .. controls (170.24,95.68) and (158.96,106.96) .. (145.04,106.96) .. controls (131.12,106.96) and (119.84,95.68) .. (119.84,81.76) -- cycle ;
		%Flowchart: Connector [id:dp1485057734469888] 
		\draw   (119.72,218.92) .. controls (119.72,205) and (131,193.72) .. (144.92,193.72) .. controls (158.84,193.72) and (170.12,205) .. (170.12,218.92) .. controls (170.12,232.84) and (158.84,244.12) .. (144.92,244.12) .. controls (131,244.12) and (119.72,232.84) .. (119.72,218.92) -- cycle ;
		%Straight Lines [id:da5287011576058043] 
		\draw    (170.12,218.92) -- (233.72,218.92) ;
		\draw [shift={(235.72,218.92)}, rotate = 180] [color={rgb, 255:red, 0; green, 0; blue, 0 }  ][line width=0.75]    (10.93,-3.29) .. controls (6.95,-1.4) and (3.31,-0.3) .. (0,0) .. controls (3.31,0.3) and (6.95,1.4) .. (10.93,3.29)   ;
		%Flowchart: Connector [id:dp2323421555185401] 
		\draw   (235.72,218.92) .. controls (235.72,205) and (247,193.72) .. (260.92,193.72) .. controls (274.84,193.72) and (286.12,205) .. (286.12,218.92) .. controls (286.12,232.84) and (274.84,244.12) .. (260.92,244.12) .. controls (247,244.12) and (235.72,232.84) .. (235.72,218.92) -- cycle ;
		%Straight Lines [id:da689465144796324] 
		\draw    (170.24,81.76) -- (233.84,81.76) ;
		\draw [shift={(235.84,81.76)}, rotate = 180] [color={rgb, 255:red, 0; green, 0; blue, 0 }  ][line width=0.75]    (10.93,-3.29) .. controls (6.95,-1.4) and (3.31,-0.3) .. (0,0) .. controls (3.31,0.3) and (6.95,1.4) .. (10.93,3.29)   ;
		%Flowchart: Connector [id:dp6413729849919538] 
		\draw   (235.84,81.76) .. controls (235.84,67.84) and (247.12,56.56) .. (261.04,56.56) .. controls (274.96,56.56) and (286.24,67.84) .. (286.24,81.76) .. controls (286.24,95.68) and (274.96,106.96) .. (261.04,106.96) .. controls (247.12,106.96) and (235.84,95.68) .. (235.84,81.76) -- cycle ;
		%Straight Lines [id:da08380556654265248] 
		\draw    (51.2,150.4) -- (114.8,150.4) ;
		\draw [shift={(116.8,150.4)}, rotate = 180] [color={rgb, 255:red, 0; green, 0; blue, 0 }  ][line width=0.75]    (10.93,-3.29) .. controls (6.95,-1.4) and (3.31,-0.3) .. (0,0) .. controls (3.31,0.3) and (6.95,1.4) .. (10.93,3.29)   ;
		%Flowchart: Connector [id:dp7446531983688254] 
		\draw   (116.8,150.4) .. controls (116.8,136.48) and (128.08,125.2) .. (142,125.2) .. controls (155.92,125.2) and (167.2,136.48) .. (167.2,150.4) .. controls (167.2,164.32) and (155.92,175.6) .. (142,175.6) .. controls (128.08,175.6) and (116.8,164.32) .. (116.8,150.4) -- cycle ;
		%Straight Lines [id:da1511415258983968] 
		\draw    (167.2,150.4) -- (230.8,150.4) ;
		\draw [shift={(232.8,150.4)}, rotate = 180] [color={rgb, 255:red, 0; green, 0; blue, 0 }  ][line width=0.75]    (10.93,-3.29) .. controls (6.95,-1.4) and (3.31,-0.3) .. (0,0) .. controls (3.31,0.3) and (6.95,1.4) .. (10.93,3.29)   ;
		%Flowchart: Connector [id:dp8705506203881077] 
		\draw   (232.8,150.4) .. controls (232.8,136.48) and (244.08,125.2) .. (258,125.2) .. controls (271.92,125.2) and (283.2,136.48) .. (283.2,150.4) .. controls (283.2,164.32) and (271.92,175.6) .. (258,175.6) .. controls (244.08,175.6) and (232.8,164.32) .. (232.8,150.4) -- cycle ;
		%Straight Lines [id:da6335929354938967] 
		\draw    (260.92,193.72) -- (258.32,177.57) ;
		\draw [shift={(258,175.6)}, rotate = 80.85] [color={rgb, 255:red, 0; green, 0; blue, 0 }  ][line width=0.75]    (10.93,-3.29) .. controls (6.95,-1.4) and (3.31,-0.3) .. (0,0) .. controls (3.31,0.3) and (6.95,1.4) .. (10.93,3.29)   ;
		%Straight Lines [id:da26211828707625884] 
		\draw    (144.92,193.72) -- (142.32,177.57) ;
		\draw [shift={(142,175.6)}, rotate = 80.85] [color={rgb, 255:red, 0; green, 0; blue, 0 }  ][line width=0.75]    (10.93,-3.29) .. controls (6.95,-1.4) and (3.31,-0.3) .. (0,0) .. controls (3.31,0.3) and (6.95,1.4) .. (10.93,3.29)   ;
		%Straight Lines [id:da08975872326436352] 
		\draw    (283.2,150.4) -- (350.31,217.51) ;
		\draw [shift={(351.72,218.92)}, rotate = 225] [color={rgb, 255:red, 0; green, 0; blue, 0 }  ][line width=0.75]    (10.93,-3.29) .. controls (6.95,-1.4) and (3.31,-0.3) .. (0,0) .. controls (3.31,0.3) and (6.95,1.4) .. (10.93,3.29)   ;
		%Flowchart: Connector [id:dp41780010498780573] 
		\draw   (351.72,218.92) .. controls (351.72,205) and (363,193.72) .. (376.92,193.72) .. controls (390.84,193.72) and (402.12,205) .. (402.12,218.92) .. controls (402.12,232.84) and (390.84,244.12) .. (376.92,244.12) .. controls (363,244.12) and (351.72,232.84) .. (351.72,218.92) -- cycle ;
		%Straight Lines [id:da9226665538697332] 
		\draw    (286.12,218.92) -- (349.72,218.92) ;
		\draw [shift={(351.72,218.92)}, rotate = 180] [color={rgb, 255:red, 0; green, 0; blue, 0 }  ][line width=0.75]    (10.93,-3.29) .. controls (6.95,-1.4) and (3.31,-0.3) .. (0,0) .. controls (3.31,0.3) and (6.95,1.4) .. (10.93,3.29)   ;
		%Straight Lines [id:da6832237097024039] 
		\draw    (145.04,106.96) -- (142.33,123.23) ;
		\draw [shift={(142,125.2)}, rotate = 279.46] [color={rgb, 255:red, 0; green, 0; blue, 0 }  ][line width=0.75]    (10.93,-3.29) .. controls (6.95,-1.4) and (3.31,-0.3) .. (0,0) .. controls (3.31,0.3) and (6.95,1.4) .. (10.93,3.29)   ;
		%Straight Lines [id:da07015174051168338] 
		\draw    (261.04,106.96) -- (258.33,123.23) ;
		\draw [shift={(258,125.2)}, rotate = 279.46] [color={rgb, 255:red, 0; green, 0; blue, 0 }  ][line width=0.75]    (10.93,-3.29) .. controls (6.95,-1.4) and (3.31,-0.3) .. (0,0) .. controls (3.31,0.3) and (6.95,1.4) .. (10.93,3.29)   ;
		
		% Text Node
		\draw (21,142.4) node [anchor=north west][inner sep=0.75pt]    {$1$};
		% Text Node
		\draw (83.52,112.68) node [anchor=south east] [inner sep=0.75pt]    {$9$};
		% Text Node
		\draw (139,74.4) node [anchor=north west][inner sep=0.75pt]    {$2$};
		% Text Node
		\draw (83.46,188.06) node [anchor=north east] [inner sep=0.75pt]    {$10$};
		% Text Node
		\draw (140,211.4) node [anchor=north west][inner sep=0.75pt]    {$3$};
		% Text Node
		\draw (256,211.4) node [anchor=north west][inner sep=0.75pt]    {$6$};
		% Text Node
		\draw (202.92,222.32) node [anchor=north] [inner sep=0.75pt]    {$23$};
		% Text Node
		\draw (203.04,78.36) node [anchor=south] [inner sep=0.75pt]    {$18$};
		% Text Node
		\draw (256,73.4) node [anchor=north west][inner sep=0.75pt]    {$5$};
		% Text Node
		\draw (137,142.4) node [anchor=north west][inner sep=0.75pt]    {$4$};
		% Text Node
		\draw (94.52,146.68) node [anchor=south east] [inner sep=0.75pt]    {$6$};
		% Text Node
		\draw (252,140.4) node [anchor=north west][inner sep=0.75pt]    {$7$};
		% Text Node
		\draw (192,133.4) node [anchor=north west][inner sep=0.75pt]    {$\infty $};
		% Text Node
		\draw (160.92,175.32) node [anchor=north] [inner sep=0.75pt]    {$5$};
		% Text Node
		\draw (279.92,177.32) node [anchor=north] [inner sep=0.75pt]    {$12$};
		% Text Node
		\draw (318.92,222.32) node [anchor=north] [inner sep=0.75pt]    {$5$};
		% Text Node
		\draw (377.92,209.32) node [anchor=north] [inner sep=0.75pt]    {$8$};
		% Text Node
		\draw (158.92,110.32) node [anchor=north] [inner sep=0.75pt]    {$9$};
		% Text Node
		\draw (274.92,110.32) node [anchor=north] [inner sep=0.75pt]    {$8$};
		% Text Node
		\draw (317.92,160.32) node [anchor=north] [inner sep=0.75pt]    {$6$};
		
		
	\end{tikzpicture}
	
\end{center}
New $ T_E,T_L=51 $ with critical path given as $ 1-3-6-7-8=51$.
\end{proof}



\chapter{Quality control}
\section{Introduction}
Quality here means a level/standard which in turn depends on four M's
\begin{itemize}
	\item Materials
	\item Manpower
	\item Machines
	\item Management
\end{itemize}

A customer expects a good quality product and hence the progress of industry depends on the marketing as well as quality.

\begin{definition}[Chance cause of variation]
	The variation due to these causes is beyond the control of human hand and cannot be prevented or eliminated under any circumstances. 
\end{definition}
It is also called as allowable as we have to allow these variation. The range of such variation is known as natural tolerance of the process.
\begin{definition}[Assignment cause of variation]
	It is due to non random preventable variation. 
\end{definition}
The assignable cause may creep in at any stage of the process, right from the arrival to the raw materials to the final delivery of goods. These causes can be identified and eliminated and are to be discovered in a production process before it goes wrong.

\section{Statistical quality control}


The aim in SQC is to control the manufacturing process so that the proportion of items beyond specified limits is not very large. This is known as process control.\par
The other type of problem is to ensure that lots of manufactured goods do not contain large proportion of defective items. This is known as product control or lot control.\par
These two are distinct problems.
Process control is achieved mainly through the technique of control charts where lot control is achieved through sampling inspection.\par
Many quality characteristics are measurable quantitatively, e.g. length of a screw, life of bulb etc. Most of them are continuous variables.
\par
Sometimes the quality of the products cannot be measured in such cases the items can be classified as defective or non defective.
\section{Control chart}
A typical control chart consists of three horizontal lines. 
\begin{itemize}
	\item \textbf{Control line: }It indicates the desired standard of the process. 
	\item \textbf{Upper control limit: }It indicates the upper limit of tolerance.
	\item \textbf{Lower control limit: }It indicates the lower limit of tolerance.
\end{itemize}

\section{Control charts for variable}
\subsection{For mean}
The $ 3-\sigma  $ limits are $ E[\overline{X}] \pm 3 \sqrt{V(\overline{X})}$.\\
If $ \mu  $ and $ \sigma  $ are known as $ \mu', \sigma'$ then $ UCL=\mu'+3\frac{\sigma'}{\sqrt{n}} $ and $ LCL=\mu'-\frac{3\sigma'}{\sqrt{n}} $. Here $ n $ is sample number not sample size.\par
If $ \mu, \sigma  $ are unknown, then $ \hat{\mu } = \overline{\overline{X}}=\frac{\overline{X_1}+\dots+\overline{X_k}}{k}$ and $ \hat{\sigma } = \frac{\overline{s}}{c_2}=\frac{\overline{R}}{d_2}$. And we have $ UCL=\overline{\overline{X}} +\frac{3 \overline{s}}{c_2 \sqrt{n}}=\overline{\overline{X}}+A_1\overline{s}$ and similar for $ LCL $.



\subsubsection{Construction of $ \overline{X} $ chart} 
Sample No. along the $ X- $axis and $ \overline{X} $ along the $ Y- $axis.\\
Sample points are then plotted against corresponding sample numbers.\\
The central line is drawn bold.\\
UCL and LCL are then plotted.\\
If point falls beyond the control line it is regarded that an assignable cause has thrown the process out of control.\\
The next step is to remove out sample results which are outside the control lines for the remaining samples.

UCL is $ \overline{\overline{X}}+A_2 \overline{R} $
\subsection{For range}
$ UCL=\overline{\overline{\alpha}}+\frac{3\overline{R}}{d_2 \sqrt{n}} =\overline{\overline{X}}+A_2 \overline{R}$ and similar with negative for LCL. 
Or $ UCL=\overline{\overline{X}} +\frac{3 \overline{S}}{c_2 \sqrt{n}}=\overline{\overline{X}}+A_1...$

\subsection{Construction of $ R- $chart}
$ \overline{R} =\frac{1}{k}(R_1+R_2+\cdot+R_k)$. So $ UCL=\overline{R} + 3 \sigma$. $ \sigma_R=d_2\hat{\sigma }$ or $ d_3 \overline{R} /d_2 $.\\
If sigma is known we will take $ UCL=D_2 \sigma, LCL=D_1 \sigma  $. If sigma is known then $ UCL=D_4\overline{R}, LCL= D_3 \overline{R}$.\\
We get $ D_1,D_2,D_3,D_4  $ from the table for different $ n $ values, for $ 2<n<25 $.
\subsection{For standard deviation}
Combination of control chart for mean $ \overline{X} $ and s.d. $ S $ known as $ \overline{X} -S$ chart. It is theoretically more appropriate than a combination of $ \overline{X}$ and $ R $.\\
$ 3-\sigma  $CL is given by $ UCL=B_2 \sigma, LCL=B_1 \sigma $ when sigma is known. And when sigma is unknown $ UCL=B_4 \overline{S} $ and $ LCL=B_3 \overline{S} $ where $ \overline{S} = \frac{\sum S_i}{n}$

\begin{example}
	A machine is said to deliver packets of ????? ???? $ 10 $ samples of size $ 5 $ each were examined and the following results were obtained.

	\begin{table}[!htb]
		\centering
		\begin{tabular}{@{}ccc@{}}
			\toprule
			Sample No. & Mean & \multicolumn{1}{l}{Range} \\ \midrule
			1 & 43 & 5 \\
			2 & 49 & 6 \\
			3 & 37 & 5 \\
			4 & 44 & 7 \\
			5 & 45 & 7 \\
			6 & 37 & 4 \\
			7 & 51 & 8 \\
			8 & 46 & 6 \\
			9 & 43 & 4 \\
			10 & 47 & 6 \\ \bottomrule
		\end{tabular}
	\end{table}

\end{example}

\begin{proof}
	Compute $ \overline{\overline{X}} = 44.2$ and $ \overline{R}=5.8 $.\\
	First do for $ \overline{X} $,
	$ CL=\overline{\overline{X}}=44.2 $, $ UCL=\overline{\overline{X}}+\frac{3\overline{R}}{d_2 \sqrt{n}}=44.2+\frac{3\times 5.8}{2.326 \times \sqrt{5}}=47.546$, and $ LCL=\overline{\overline{X}}-\frac{3 \overline{R}}{d_2 \sqrt{n}}=40.85455 $ \\
	Then do for $ \overline{R} $, we have $ CL=\overline{R} =5.8$, and $ UCL=D_4\overline{R} =2.115 \times 5.8= 12.267$ and $ LCL=D_3 \overline{R} = 0 \times 5.8=0 $
\end{proof}

\FloatBarrier
\section{Control charts for attributes}
The most commonly used control charts under this category are,
\subsection{$ p -$chart}
Control chart for fraction defective. The objective of this chart is to evaluate the quality of the items, i.e. the average fraction defective or percent defective. And note the changes in quality over period of time. The chart can also be used in routine control and guide for correcting causes of bad quality. In many situations these charts suggest where $ \overline{X} $ and $ R $ charts could be used.\par
Steps to make a $ p- $chart,
\begin{enumerate}
	\item $ P=\frac{\text{No. of defectives in the sample}}{\text{Sample size}}=\frac{np}{n} = \frac{d}{n}$, average $ f.d.p=\frac{\sum d}{\sum n} $
	\item $ UCL_P=P+3\sigma_p, LCL=P-3\sigma_p $ where $ \sigma_p=\sqrt{\frac{P(1-P)}{n}} $
\end{enumerate}
\subsubsection{Case I:}
$ P'=P $ then $ LCL_p=P'-3\sqrt{P'(1-P')/n}=P'-A\sqrt{P'(1-P')} $ similar for $ UCL $ and $ CL $.

\subsubsection{Case II:}
$ P=\overline{P} $
same formula as above but with $ \overline{P} $.

If $ LCL_p $ is found to be negative according to the formulas it is taken to be $ 0 $.

If the point goes above the UCL it reflects the lack of statistical control since it has changed for the worse. Such a point known as high spot, indicates deterioration in the lot quality. 

If the point goes below the LCL it again exhibits lak of control as the process has changed for the better. Such a point is known as low spot, indicates improvement in the lot quality.

\subsection{$ np-$chart/$ d-$chart}
Control chart for number of defectives.



If ??????? on each occasion is the same the plotting of actual number of defectives may be more convenient and meaningful than the fraction defective. The difference is that the actual number of defectives (np) in samples of fixed size is plotted instead of fractional defectives and the central line is drawn at np instead of p.

Following values required for np,
\begin{enumerate}
	\item $n=$sample size constant
	\item The number of defectives in each sample $d=np$
	\item Average number of defectives per sample of constant size, $$nP=\frac{\text{total no. of defectives}}{\text{number of samples}}$$ 
\end{enumerate}
The $ 3-\sigma  $ limits are as follows,
\begin{enumerate}
	\item $ UCL_{np} = nP+3\sigma_{np}$
	\item $ LCL_{np} =nP-3\sigma_{np}$
\end{enumerate}
Where, $ P= $the fraction defective true in process (population), $ nP= $the average number of defectives per sample based on all possible sample size $ n $ from the population and $ \sigma_{np}=\sqrt{nP(1-P)} $.

\textbf{Case I: }If $ P=p' $ then $ LCL=np'-3\sqrt{np'(1-p')} $ with CL UCL as expected.

\textbf{Case II: }Suppose we have $ k $ different samples, $ LCL=np-3\sqrt{np(1-p)} $ with CL UCL as expected.


\textbf{$ np -$chart vs $ p- $chart}
... control line and control limits will vary, in such case $ p- $chart is better to use. However, in case of constant sample size for all the sample, any one of $ p- $chart or $ np- $chart can be used. But in practice $ np- $chart is most commonly used.

\subsection{$ c- $chart}
Control chart for number of defects per unit.

The $ c-$chart is used for control of number of defects per unit. Although the applications of $ c- $chart is somewhat limited compared with $ p- $chart there are many instances in industry where it is useful. E.g. for control of number of defects of a material, number of soiled packages in a given consignment or number of weak spots in a given length.

....based on number of defectives per unit. E.g. in case of inspection of fairly complex assembled units such as T.V. set.

\textbf{The construction of control chart.}

The number of defects where the sample was constant was as follows,

\begin{enumerate}
	\item Count the number of defects individually. Obtain average number of defects $ \overline{c} = \frac{\text{no of defects in all samples}}{\text{total no of samples}}$.
	\item Create $ UCL, LCL $ with $ 3-\sigma  $ limits. $ UCL_c=\hat{c} + \sqrt{\hat{c}} $ and $ LCL_c= 3 \sqrt{\hat{c}}$. Where $ \hat{c} =\overline{c}=\frac{c_1+c_2+\cdots+c_k}{k}$ is true average no of defects per product.
\end{enumerate}

When the estimate value is used, the two limits may be written as $ UCL_c=\overline{c} + 3 \sqrt{\overline{c}}$ and $ LCL_c=\overline{c} - \sqrt{\overline{c}}$. Where $ c_1,c_2,\dots,c_k $ are the number of defects observed in $ 1^{st}, \dots, k^{th}$ sample respectively.


The interpretation of this chart is similar to the chart for fraction defective and number of defectives. The evaluation of the standards are also done in the same line.

Although the applications for the $ c- $chart is somewhat limited compared with the $ p- $chart still there are many instances in industry where it is useful. It can be used to identify or to highlight the number of surface(?????) effects observed in roll of galvanized sheet of printed/plated ????? or sheet photographic film.?????????????

\begin{example}
	A sample of 500 wiring boards are taken from the wiring assembly line each day. The number and percentage rejected during 20 consecutive days of production are shown below.
	\begin{table}[!htp]
		\centering
		\begin{tabular}{@{}llll@{}}
			\toprule
			Date sampled & No of Samples & Number rejected & Defective percent \\ \midrule
			8 & 500 & 40 & 8 \\
			9 & 500 & 30 & 6 \\
			10 & 500 & 20 & 4 \\
			11 & 500 & 60 & 12 \\
			12 & 500 & 30 & 6 \\
			15 & 500 & 20 & 4 \\
			16 & 500 & 90 & 18 \\
			17 & 500 & 50 & 10 \\
			18 & 500 & 30 & 6 \\
			19 & 500 & 20 & 4 \\
			22 & 500 & 20 & 4 \\
			23 & 500 & 10 & 2 \\
			24 & 500 & 30 & 6 \\
			25 & 500 & 20 & 4 \\
			26 & 500 & 20 & 4 \\
			29 & 500 & 10 & 2 \\
			30 & 500 & 30 & 6 \\
			31 & 500 & 20 & 4 \\
			1 & 500 & 10 & 2 \\
			2 & 500 & 39 & 7.8 \\ \bottomrule
		\end{tabular}
	\end{table}
\end{example}
\begin{proof}
	We compute $ \overline{P} =\frac{599}{10000}=0.0599$
\end{proof}

\begin{align*}
	\begin{matrix}
		\text{Type} && \text{Standards given} && \text{Standards are not given}\\
		\text{Mean} && CL= \mu' && CL=\overline{\overline{X}}\\
		 && UCL=\mu'+\frac{3 \sigma'}{\sqrt{n}} && UCLc = \overline{\overline{X}}+A_2 \overline{R}\\
		 && LCL = \mu'-\frac{3 \sigma'}{\sqrt{n}} && LCL = \overline{\overline{X}}-A_2 \overline{R}\\
		 \text{Range} && CL=\sigma' && CL= \overline{R}\\
		 && UCL = D_1 \sigma' && UCL=D_4 \overline{R}\\
		 && LCL=D_2 \sigma' && LCL=D_3 \overline{R}
	\end{matrix}
\end{align*}\footnote{$ A_2=\frac{3}{d_2\sqrt{n}} $}

\section{Objectives of statistical quality control}
\begin{enumerate}
	\item To assess the quality of raw material, semi finished goods and finished products at various stages of production process. To achieve better utilization of raw materials, more efficient utilization of equipment etc. To locate and identify the process faults. In order to control the scrap and waste.
	\item  To reduce the ????? men and machine during the process of production.
	\item To see whether the product conforms to the predetermined standards and specifications and whether it satisfies the needs of the customer.
	\item To take necessary corrective steps to maintain the quality of the product.
	\item To protect the manufacturers as well as the consumers against heavy losses due to a rejection of a large quantity of product.
\end{enumerate}

\section{Acceptance sampling}
If the plan is ????? qualitatively then the plan is referred as attribute sampling plan.

And if the plan is ????? then the plan is referred as variable sampling plan

An attribute sampling plan may be given as follows,

Select a sample of size $ n=50 $ and count the number of defectives $ c $ if $ c \leq 3 $ accept the lot otherwise reject.

A variable sampling plan typically uses the normal distribution and might be stated as follows,

Select a random sample of size $ n =50$ and determine the mean tensile strength $ x $ if $ x>12000 $ accept the lot otherwise reject it.

While in attribute sampling plan the number of defectives in the parent population are estimated from binomial and poisson distribution.

\begin{align*}
	\begin{bmatrix}
		\text{Actual condition} && \text{Lot accepted} && \text{Rejected}\\
		\text{Good lot} && \text{Favourable outcome} && \text{Producer risk, type I error}\\
		\text{Bad lot} && \text{Consumer risk, Type II Error} && \text{Favourable outcome}
	\end{bmatrix}
\end{align*}

Producers risk - Type I error ($ \alpha  $)
, Consumers risk - Type II error $ p $ is lot tolerance ($ \beta $)

AQL- Acceptable quality level - good lot.


By producer we mean any person, firm or department that prepares goods to be supplied to another person, firm or department.

Any sampling inspection plan for acceptance or rejection of a lot possesses the disadvantage of occasionally rejecting a lot of satisfactory quality

$ P $ is called producers process average,

$ P $[rejecting a lot, under the sampling inspection plan]

If a good lot is rejected we refer it as a type I error.

A person, firm, department that receives a product from the producer is called the consumer.

$ P $[accepting a lot of unsatisfactory quality on the basis of sampling inspection]=$ \beta $.

If $ p $ ????? that is maximum fraction defective in a lot then the consumer can tolerate. Then it is called as consumers risk.

Consumers want to keep risk low and we refer it as a type II error.

Acceptable quality level is the lot quality which the supplier believes will be accepted by the consumer. That is he believes that a lot with quality $ p $ will have a small chance of being rejected, i.e. his risk is very small. Hence it is called producers risk where $ p $ denotes the proportion of defectives in the lot.

P[the customer will reject a lot with quality p]=$ \alpha $ which is very small.

LTPD - Lot tolerance perfect defective. Can also be called as rejection quality level (RQL) and it is denoted as $ p' $. It is the maximum perfect defective that the customer is ready to accept, in an accepted lot. He believes that the probability of accepting a lot which quality more than $ p $ is very small.

The ??? ???? of quality $ p' $ is equal to $ \beta  $.

Process average fraction defective which is observed over a long period of time which is given as $ \overline{p} $.

\backmatter


\end{document}

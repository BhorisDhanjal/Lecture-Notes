%This template consists of the minimum of a single book.
%Please do not think this template is mandatory and the format must be followed strictly.
%We expect the author adds what he needs.
\documentclass[oneside,11pt,pdftex]{book}%Remove draft when book editing is completed.
\usepackage{mathtools}
\usepackage{amssymb}
\usepackage{stix}
\usepackage{mathrsfs}
\usepackage{textcomp,mathcomp}
%\usepackage[T1]{fontenc}
%\usepackage[utf8]{inputenc}
\usepackage{bm}
\usepackage{array}
\usepackage{cite}
\usepackage[final,hiresbb]{graphicx}
\usepackage[x-1a1]{pdfx}
\usepackage{amsthm}
%\usepackage{tikz}

\frenchspacing
\allowdisplaybreaks
%Be careful when you use commands which align formulas.
%If aligned formulas range to two pages, the formulas should be divided into two environments.
%\makeatletter
%\AtBeginDocument{\let\mathaccentV\AMS@mathaccentV}
%\makeatother
%This is a patch for double bar.
%Activate it if \bar{\bar{a}} doesn't work.

\newskip\thskip
\thskip=0.5\baselineskip plus 0.2\baselineskip minus 0.2\baselineskip

\newdimen\dtest%Remove this when book editing is completed.
\settowidth{\dtest}{letters and symbols here}
\typeout{<<<\the\dtest>>>}

\newtheorem{theorem}{Theorem}[chapter]%Modify these declarations for your need.
\newtheorem{lemma}[theorem]{Lemma}
\newtheorem{corollary}[theorem]{Corollary}
\newtheorem{example}[theorem]{Example}
\newtheorem{definition}[theorem]{Definition}

\newtheorem{xca}[theorem]{Exercise}

\newtheorem{remark}[theorem]{Remark}

\numberwithin{section}{chapter}
\numberwithin{equation}{chapter}

\makeindex

\newcommand{\R}{\mathbb{R}}
\newcommand{\Q}{\mathbb{Q}}
\newcommand{\C}{\mathbb{C}}
\newcommand{\Z}{\mathbb{Z}}
\newcommand{\N}{\mathbb{N}}
\newcommand{\D}{\mathbb{D}}
\newcommand{\F}{\mathbb{F}}

\begin{document}


\frontmatter

\thispagestyle{empty}
\begin{flushright}
{\LARGE \textbf{Bhoris Dhanjal}}%Input your name here.
\end{flushright}
\vfill
\begin{center}
{\fontsize{29.86truept}{0truept}\selectfont \textbf{Calculus IV}}%Input the book title here.
%Below is for a book with a subtitle.
%{\fontsize{29.86truept}{0truept}\selectfont \textbf{The Book Title}} \\
%\vspace{6.5truept}
%{\Large, \LARGE, etc. \textbf{The Subtitle}}
\end{center}
\vfill
\begin{flushleft}
{\LARGE \textbf{Lecture Notes}} \\
\hspace{-1.75truept}
{\large \textbf{for SMAT401}}
\end{flushleft}
\newpage

\tableofcontents


\mainmatter

\chapter{Tutorial 1}
Show that the limits as the function approaches (0,0) dont exist
\section{Question 1}
\[ \frac{x^2-y^2}{x^2+y^2} \] use x and y axis

\section{Question 2}
\[ \frac{x^3y}{x^6+y^2} \] consider $ y=0 $ then we have limit equal 0. Consider now $ y=x^3 $ so now limit of $ \frac{x^6}{2x^6} =\frac{1}{2}$.



\section{Question 3}
\[ \frac{\sin(x^2+y)}{x+y} \] along the x axis (y=0) we get $ \sin(x^2) /x $ and the lim is 0. But for y axis (x=0) we get $ \sin(y)/y$ and the lim is 1. 

\section{Question 4}
\[ 	\frac{x^3+y^3}{x-y} \] take the line $ y=0 $ we get 
$ \frac{x^3+m^3x^3}{x-mx} =\frac{(1+m)x^3}{x(1-m)}$ is 0 but with $ y=x-x^3$ is equal to 2.

Try with $ y=x-x^2 $ we get
\begin{align*}
	\lim \frac{x^3+(x-x^3)^3}{x-(x-x^3)}&=\lim \frac{x^3}{x^3}+\frac{()x-x^3)^3}{x^3}\\
	&= 1+ \lim \frac{(x-x^3)^3}{x^3} = 1+ 1=2
\end{align*}



\section{Question 5}
\[ \frac{x^2y^2}{x^2y^2+(x-y)^2} \] consider the line $ x=0 $ then the limit is obviously 0. Now consider $ y=x $ then the limit of $ \frac{x^4}{x^4} =1$.


\section{Question 6}
\[ \frac{2xy^2}{x^3+y^3} \] take y=0 and x=y.
\backmatter

\chapter{Tutorial 2}
\section{Question 1}
\[ \lim_{(x,y)\rightarrow (0,0)} xy \sin \left(\frac{1}{x^2+y^2} \right)=0\]
\thispagestyle{empty}%If your book ends with the even numbered page, copy and paste it twice. With the odd numbered page, do it three times.
{\ }
\newpage

\end{document}

%This template consists of the minimum of a single book.
%Please do not think this template is mandatory and the format must be followed strictly.
%We expect the author adds what he needs.
\documentclass[oneside,11pt,pdftex,final]{book}%Remove draft when book editing is completed.
\usepackage{graphicx}
\usepackage{amsmath}
%\usepackage{fontawesome5}
\usepackage{booktabs}
\usepackage{amssymb}
\usepackage{longtable}
\usepackage{amsthm}
\usepackage{inputenc}
\usepackage{multirow}
\usepackage[activate={true,nocompatibility},final,tracking=true,kerning=true,spacing=true,factor=1100,stretch=10,shrink=10]{microtype}
\usepackage[toc,page]{appendix}
\usepackage[nottoc]{tocbibind}
\numberwithin{equation}{section}
\graphicspath{ {./Images/} }
%\usepackage[raggedright]{titlesec}
\usepackage{placeins}
\usepackage{mathtools}
\usepackage{epigraph}
\usepackage{fancyhdr}
\usepackage{hyperref}
%Be careful when you use commands which align formulas.
%If aligned formulas range to two pages, the formulas should be divided into two environments.
%\makeatletter
%\AtBeginDocument{\let\mathaccentV\AMS@mathaccentV}
%\makeatother
%This is a patch for double bar.
%Activate it if \bar{\bar{a}} doesn't work.

\newskip\thskip
\thskip=0.5\baselineskip plus 0.2\baselineskip minus 0.2\baselineskip

\newdimen\dtest%Remove this when book editing is completed.
\settowidth{\dtest}{letters and symbols here}
\typeout{<<<\the\dtest>>>}


\makeatletter
\renewcommand{\@chapapp}{}% Not necessary...
\newenvironment{chapquote}[2][2em]
{\setlength{\@tempdima}{#1}%
	\def\chapquote@author{#2}%
	\parshape 1 \@tempdima \dimexpr\textwidth-2\@tempdima\relax%
	\itshape}
{\par\normalfont\hfill--\ \chapquote@author\hspace*{\@tempdima}\par\bigskip}
\makeatother


\newtheorem{theorem}{Theorem}[chapter]%Modify these declarations for your need.
\newtheorem{lemma}[theorem]{Lemma}
\newtheorem{corollary}[theorem]{Corollary}
\newtheorem{example}[theorem]{Example}
\newtheorem{definition}[theorem]{Definition}
\newtheorem{xca}[theorem]{Exercise}
\newtheorem{remark}[theorem]{Remark}
\numberwithin{section}{chapter}
\numberwithin{equation}{chapter}
\setcounter{secnumdepth}{5}
\setcounter{tocdepth}{5}
\makeindex

\newcommand{\R}{\mathbb{R}}
\newcommand{\Q}{\mathbb{Q}}
\newcommand{\C}{\mathbb{C}}
\newcommand{\Z}{\mathbb{Z}}
\newcommand{\N}{\mathbb{N}}
\newcommand{\D}{\mathbb{D}}
\newcommand{\F}{\mathbb{F}}

\begin{document}
	
	
	\frontmatter

\thispagestyle{empty}
\begin{flushright}
{\LARGE \textbf{Bhoris Dhanjal}}%Input your name here.
\end{flushright}
\vfill
\begin{center}
{\fontsize{29.86truept}{0truept}\selectfont \textbf{Differential equations}}%Input the book title here.
%Below is for a book with a subtitle.
%{\fontsize{29.86truept}{0truept}\selectfont \textbf{The Book Title}} \\
%\vspace{6.5truept}
%{\Large, \LARGE, etc. \textbf{The Subtitle}}
\end{center}
\vfill
\begin{flushleft}
{\LARGE \textbf{Lecture Notes}} \\
\hspace{-1.75truept}
{\large \textbf{for SMAT403}}
\end{flushleft}
\newpage

\tableofcontents


\mainmatter
\chapter*{Introduction}
\begin{chapquote}{Gian-Carlo Rota}
	``This course is justly viewed as the
	most unpleasant undergraduate course in mathematics, by both teachers and students. Some of
	my colleagues have publicly announced that they would rather resign from MIT than lecture in
	sophomore differential equations''
\end{chapquote}
These are lecture notes for the SMAT403 Differential Equations course. If you spot mistakes just message me and let me know. There probably might be some typos.
\par Differential equations might well be the most boring topic in maths I think I will ever study. 
\par Hopefully these notes make it more tolerable. If I remember to do it, there might be an appendix with common derivates and integrals at the end.
\chapter{First order ordinary linear differential equations}
\begin{chapquote}{Isaac Newton}
	``Just use Mathematica bro.''
\end{chapquote}
\section{Homogenous equations}
The differential equation
\[ M(x,y)\, dx + N(x,y)\, dy=0\] is said to be homogenous if $ M $ and $ N $ are of the same degree.\par
Substitute $ y=vx $ to solve a homogenous ODE.
\begin{example}
	Solve $ (x+y)\ dx - (x-y)\ dy =0$
\end{example}
\begin{proof}
	\begin{align*}
		\frac{dy}{dx}&=\frac{x+y}{x-y}
		\intertext{Let $ y=vx $}
		\frac{dy}{dx}&=v+x\frac{dv}{dx}\\
		v+x\frac{dv}{dx}&=\frac{x+vx}{x-vx}\\
		\frac{(1-v)}{1+v^2}\, dv&=\frac{1}{x}\, dx
		\intertext{Integrate both sides}
		\arctan v - \frac{1}{2}\log(1+v^2)&=\log(x)+c\\
		\arctan\left(\frac{y}{x}\right)&=\log(\sqrt{x^2+y^2})+c
	\end{align*}
\end{proof}

\section{Exact differential equation}
	

IF you have $ M\, dx + N\, dy=0 $ and if $ \frac{\partial M}{\partial y}=\frac{\partial N}{\partial x} $ then the differential equation is exact.\par
The solution is $ f(x,y) =c$ where $ \frac{\partial f}{\partial x} =M, \frac{\partial f}{\partial y}=N$
\begin{example}
	$ e^y\, dx+(xe^y+2y)\, dy=0$
\end{example}
\begin{proof}
	Here we have, $ M=e^y, N=xe^y+2y $\\
	\begin{align*}
		\frac{\partial M}{\partial y}=\frac{\partial}{\partial y} (e^y)=e^y\\
		\frac{\partial N}{\partial x}=\frac{\partial}{\partial x}(xe^y+2y)=e^y
	\end{align*}
	Therefore, it is exact.
	\begin{align*}
		\frac{\partial f}{\partial x}&=e^y\\
		f(x,y)&=\int e^y\, dx= xe^y+g(y)\\
		\frac{\partial f}{\partial y}&=xe^y+\frac{dg}{dy}=N=xe^y+2y\\
		\frac{dg}{dy}&=2y
	\end{align*}

\end{proof}
\chapter{Higher order ordinary linear differential equations}
\begin{chapquote}{Euclid, \textit{When a time traveller showed him differential equations}}
	``Damn this is boring.''
\end{chapquote}
\section{Second order linear differential equations}
\subsection{Definition}
One dependent variable $ y $ and one independent variable $ x $.\\
The general second order linear differential equation is \[ \frac{d^2y}{dx^2} + P(x)\frac{dy}{dx}+Q(x)y=R(x)\]
or 
\begin{align} \label{def:secondorder}
	y''+P(x) y'+Q(x)y=R(x)
\end{align}

\subsection{Existence and uniqueness theorem}
\begin{theorem}
Let $ P(x), Q(x), R(x)$ be continuous functions on a closed interval $ [a,b].$ If $ x_0 $ is any point in $ [a,b] $ and if $ y_0, y_0' $ are any numbers. Then eq. \ref{def:secondorder} has one and only one solution $ y(x) $ on the entire interval such that $ y(x_0) =y_0$ and $ y'(x_0) =y_0'$
\end{theorem}
\begin{proof}
	Not covered in class. Just trust me bro.
\end{proof}

\begin{example}
	Find the largest interval where $ (x^2-1)y''+3xy'+(\cos x)y =e^x, y(0)=4, y'(0)=5$ is guaranteed to have a unique solution.
\end{example}

\begin{proof}
	We need to divide by $(x^2-1)$ but this automatically implies that from $ \R $ we cannot include $ -1,1 $. \\
	$ P(x) =\frac{3x}{x^2-1}$, it is continuous over $ \R \setminus \{-1,1\} $,
	$ Q(x)=\frac{\cos(x)}{x^2-1} $ same case,
	$ R(x) =\frac{e^x}{x^2-1}$ same.\\
	Therefore the interval is simply, $ (-\infty,-1)\bigcap (-1,1) \bigcap (1, \infty) $ we will choose just the middle interval since we need $ 0 $.\\
	Therefore the largest interval in which the DE is guaranteed to have a unique solution is (-1,1).
\end{proof}

\begin{example}
	$ (x+2)y''+xy'+(\cot x)y=x^2+1, y(2) =11, y'(2)=-2$
\end{example}
\begin{proof}
	Divide by $ (x+2) $ so $ x=-2 $ cannot be included. Also $ \cot x $ isnt defined for $ x=n\pi  $.\\
	So the required largest interval would be $ (0,\pi) $
\end{proof}

\begin{example}
	Find the largest interval where $ (x^2-4x) y''+3xy' +4y=2, y(3)=0, y'(3)=-1$
\end{example}
\begin{proof}
	We need to divide by $ (x^2-4x) $ so we cannot include the points $ 0, 4 $ so the interval is $ (0,4) $.
\end{proof}

\begin{example}
	$(x-3)y''+xy'+\log|x| y=0, y(1)=0,y'(1)=1$
\end{example}
\begin{proof}
	We need to divide by $(x-3)$ so $ x\neq 3 $. Also $ \log x $ is not defined for $ x=0 $. So the required interval is just $ (0,3) $
\end{proof}

\subsection{Homogenous equation}
\begin{definition}\label{def:2ndhomo}
	The equation \begin{align*}
		y''+P(x) y'+Q(x)=0
	\end{align*}  is called a homogenous equation.
\end{definition}

\begin{theorem}
	If $ y_p(x) $ is a fixed particular solution of eq \ref{def:secondorder} and $ y(x) $ is any general solution of eq. \ref{def:secondorder}, then $ y(x) -y_p(x)$ is a solution of \ref{def:2ndhomo}
\end{theorem}
\begin{proof}
	Let $ y_1 $ be some solution to \ref{def:secondorder} and $ y_2 $ be some other solution to \ref{def:secondorder}. Then $ y_1-y_2 $ will be a solution to \ref{def:2ndhomo}.??? (\emph{Check this later it makes no sense})
\end{proof}

\begin{theorem}
	If $ y_1(x) $ and $ y_2(x) $ are any two solutions of eq. \ref{def:2ndhomo}, then $ c_1y_1(x) +c_2 y_2 (x)$ is also a solution for any constants $ c_1 $ and $ c_2 $.
\end{theorem}
\begin{proof}
	The proof is trivial and is left as an exercise to the next person who reads this.\\
\end{proof}

\begin{example}
	Verify that $ y=c_1 \cos x + c_2 \sin x $ is a solution of $ y''+y=0 $ Find the solution which satisfies (A)(???) and
	\begin{itemize}
		\item $ y(0) =0, y'(0)=1$
		\item $ y(0)=1, y'(0)=0 $
	\end{itemize}
\end{example}

\begin{proof}
	Consider case $ i $,
	\begin{align*}
		y(0)&=c_1 \cos (0)+ c_2 \sin (0)\\
		&=c_1
	\end{align*}
	Therefore, $ c_1=0 $ but $ c_2 $ is undecided.... \emph{finish this later}
\end{proof}

\begin{example}
	Solve $ y''+y'=0 $
\end{example}
\begin{proof}
	\begin{align*}
	y''+y'&=0 \\ 
	y'=t(x) \ \to y''&=t'(x) 
	\\ t'+tz&=0 \\ 
	\frac{dt}{t}+dx&=0 \\
	 \ln |t|+x&=C \\ 
	 \ln|t| + \ln e^x &= \ln e^{C} \\ 
	 t \cdot  e^x&=C_1 \\ 
	 y' \cdot e^x &=C_1 \\ 
	 dy=&C_1\frac{dx}{e^x} \\ 
	 \int dy=&C_1 \int \frac{dx}{e^x} \\ ...
	\end{align*}
General solution is $ y=c_1e^{-x}+c_2 $	
\end{proof}

\begin{example}
	Solve $ x^2y''+2xy'-2y=0 $
\end{example}

\begin{proof}
	Let $ y=x^m $ so $ y'=mx^{m-1} $ and $ y''=m(m-1)x^{m-2} $
	\begin{align*}
		x^2m(m-1)x^{m-2}+2x(mx^{m-1})-2x^m&=0\\
		x^m(m^2-m+2m-2)&=0
		\intertext{$m=1$ or $m=-2$}
		\therefore y_1(x)=x'=x \text{ and } y_2(x)&=x^{-2}\\
		\text{General solution } y(x)&=c_1x+c_2x^{-2}
		\end{align*}
	
\end{proof}

\begin{example}
	Verify that $ y_1=1, y_2=x^2 $ are solutions of $ xy''-y'=0 $ and write down the general solution.
\end{example}
\begin{proof}
	We will not insult the intelligence of the reader by verifying the solution.\\
	For the general solution consider that $ 1, x^2 $ are Linearly independent, so the general solution is $ y(x)=c_1(1)+c_2(x^2) $
\end{proof}

\section{General solution of the homogenous system}

\begin{definition}\label{def:lindep}
	If two functions $ f(x) ,g(x)$ are defined on an interval $ I $ and have the property that one is a constant multiple of the other, then they are said to be linearly dependent on $ I $. Otherwise they are called linearly independent.
\end{definition}
Note that, if $ f(x) \equiv 0$ and $ g(x) $ are linearly dependent for every function $ g(x) $.


\begin{definition}\label{def:wronskian}
	Let $ y_1(x) ,y_2(x)$ be linearly independent solutions of the homogenous equations $ y''+P(x) y'+Q(x)y=0$. Then the function $ W(y_1,y_2)=y_1y_2'-y_1'y_2 $ is called the Wronskian of $ y_1,y_2 $.\\
	\[ W(y_1,y_2)= \begin{vmatrix}
		y_1 & y_2 \\ 
		y_1' & y_2' 
	\end{vmatrix}\]
\end{definition}
If two functions are dependent then their Wronskian is identically zero.

\begin{lemma}
	If $ y_1(x) , y_2(x)$ are any two solutions to eq. \ref{def:2ndhomo} on interval $ I $ then their Wronskian is either identically zero or never zero on $ I $.
 \end{lemma}
\begin{proof}
	I trust my professor.
\end{proof}
\begin{example}
	By eliminating $ c_1 $ and $ c_2 $, find the differential equation for the family of the curves $ y=c_1e^x+c_2e^{-3x} $. Then use Abel's formula to find the Wronskian.
\end{example}
\begin{proof}
	\begin{align}
		y'&=c_1e^x-3c_2e^{-3x}\\
		y''&=c_1e^{x}+9c_2e^{-3x}
		\intertext{Now do $ y-y' $}
		y-y''&=4c_2e^{3x} \implies c_2=\frac{y-y'}{4e^{-3x}}
		\intertext{Now do $ y''-y' $}
		y''-y'&=12c_2e^{-3x}\implies c_2=\frac{y''-y'}{12e^{-3x}}
		\intertext{But now we have the following equality}
		y''-y'&=3(y-y')
		\intertext{Expanding this gives us the differential equation required as,}
		y''+2y'-3y&=0
	\end{align}

\end{proof}

\begin{lemma}\label{lem:ldwronskian}
	If $ y_1(x) $ and $ y_2(x) $ are two solutions of eq. \ref{def:2ndhomo} on $ I $, then they are linearly dependent on this interval $ \iff $ their Wronskian is identically 0.
\end{lemma}
\begin{proof}
	\textbf{Case 1: }If the function is linearly dependent then its Wronskian is equal to 0. \begin{align*}
		W(y_1,y_2)&=0\\
		y_1y_2'-y_1'y_2&=0\\
		y_1y_2'-y_1'y_2&=0\\
		(cy_2)y_2'-(cy_2')y_2&=0
	\end{align*}
		\textbf{Case 2:} If the Wronskian is identically equal to 0 then we have to prove the function is linearly dependent.\\
		\textbf{Case 2a:} If $ y_1 \equiv 0 \rightarrow y_1 $ is the zero function then $ y_1, y_2 $ are L.D.\\
		\textbf{Case 2b:} If $ y_1 \not \equiv 0 \implies y_1(x_0)\neq 0, $ for some $ x_0 \in I $. This implies $ \exists [c,d] \subseteq I $ s.t. $ y_1(x_0)\neq0 \forall x_0 \in [c,d]$.\\
		Also $ W=0$ on $ [c,d]  \implies y_1 y_2'-y_1'y_2=0 \implies \frac{y_1y_2'-y_2 y_1'}{y_1^2} =0 \implies \left( \frac{y_2}{y_1}\right)'$. And we get $ y_2(x)=ky_1(x) $ for some $ [c,d] \in I$. We need to extend this to all $ I $. $ y_2(x_0)=ky_1(x_0)=y_0 \forall x_0 \in [c,d] $ $ y_2'(x_0)=ky_1'(x_0)=y_0' \forall x_0 \in [c,d] $, then use existence and uniqueness theorem.
\end{proof}

\begin{lemma}
	If $ y_1(x) $ and $ y_2(x) $ are two solutions of eq. \ref{def:2ndhomo} on $ I $, then they are linearly independent on this interval iff their Wronskian is never zero on $ I $.
\end{lemma}
\begin{proof}
	This proof is very boring and is left as an exercise to the reader. Use lemma \ref{lem:ldwronskian}.\\
	Assume that $ y_1, y_2 $ are L.I. then we have to show $ W \neq 0 $. Assume $ W=0 $ then use lemma \ref{lem:ldwronskian}.\\
	If $ W $ is never zero then show that $ y_1, y_2 $ are L.I.
\end{proof}

\begin{theorem}
	Let $ y_1(x) $ and $ y_2(x) $ be linearly independent solutions of the homogenous equation \ref{def:2ndhomo} on $ I $. Then $ c_1 y_1(x) + c_2 y_2 (x)$ is the general soltuion of equation \ref{def:2ndhomo} on $ I $.
\end{theorem}
\begin{proof}
	Show $ c_1 y_1+c_2y_2 $ be a solution of eq \ref{def:2ndhomo}. Next, let $ y(x) $ be any other solution of \ref{def:2ndhomo} then show that there exists $ c_1, c_2 \in \R $ such taht $ y(x)=c_1y_1+c_2y_2 $. That is, to show that for some $ x_0 \in I $ we can find $ c_1, c_2 $ s.t. $ c_1 y_1(x_0) +c_2'y_2(x_0)=y(x_0)$ and $ c_1y_1'(x_0)+c_2y_2'(x_0) $.
	That is we have to show,
	\[ \begin{bmatrix}
	y_1(x_0)	& y_2(x_0)\\ 
	y_1'(x_0)	& y_2'(x_0) 
	\end{bmatrix} \begin{bmatrix}
	c_1 \\ c_2
\end{bmatrix}=\begin{bmatrix}
y(x_0)\\ y'(x_0)
\end{bmatrix} \]
Complete this later its boring
\end{proof}

\begin{example}
	Show that $ y=c_1 \sin x + c_2 \cos x $ is the general solution of $ y''+y=0 $ on any interval, and find the particular solution for which $ y(0)=2 $ and $ y'(0)=3$.
\end{example}
\begin{proof}
	\begin{align*}
		y_1&=\sin x, y_2= \cos x
	\end{align*}
It is easy to see that it a solution. Now we just have to show that it is linearly independent and them by the previous theorem we can say its linear combination is a general solution. Using the initial conditions we then solve and find the values for $ c_1 $ and $ c_2 $
\end{proof}


\begin{example}[H.W]
	Show that the space of solutions for the homogenous equation $ y''+P(x)y'+Q(x)y=0 $ is a vector space over $ \R$.
\end{example}

\begin{example}[Review]
	Show that $ e^x $ and $ e^{-x} $ are linearly independent solutions of $ y''-y=0 $ in any interval.
\end{example}
\begin{proof}
	Verify that it is a solution. Consider then its Wronskian.
\end{proof}
\subsection{Using a known solution to find another}
\begin{theorem}[Use of a known solution to find another]\footnote{Just memorize this garbage.}
	Consider $ y''+P(x)y'+Q(x)y=0 \to (1)$. Assume $ y_1(x) $ is a known non-zero solution of (1). Assume $ y_2=v y_1 $ is another solution to (1), where $ v=v(x) $. Then $ y_2''+P(x)y_2'+Q(x)y_2 =0\implies [vy_1]''+P(x)[vy_1]'+Q(x)[vy_1]=0$.\[ v=\int \frac{1}{y_1^2} e^{-\int P\, dx}\, dx\]
\end{theorem}

\begin{example}
	Show that $ y_1=x $ is a solution of $ x^2y''+xy'-y=0 $. Find the general solution.
\end{example}
\begin{proof}
	$ y_1'=1,y_1''=0 $, therefore we have
	\begin{align*}
		x^2y''+xy'-y&=0\\
		x^2(0)+x(1)-x&=0
	\end{align*}
Therefore, it is a solution.
Now we need to find $ y_2 $ another solution. Here in the standard form $ P(x)=\frac{1}{x}, x\neq=0$.\\
Now compute $ v $
\begin{align*}
	v&=\int \frac{1}{y_1^2}e^{-\int P\, dx}\, dx\\
	&= \int \frac{1}{x^2} \frac{1}{x}\, dx\\
	&= \int \frac{1}{x^3}\, dx\\
	&= - \frac{1}{2x^2}
\end{align*}
So now we can say $ y_2=vy_1=\frac{-1}{2x^2}x=\frac{-1}{2x} $.\\
Therefore, the general solution for the given differential solution is $ y(x) = c_1 y_1 + c_2 y_2=c_1 x + c_2 \frac{-1}{2x}$ in any interval not containing $ 0 $.
\end{proof}

\begin{example}
	Use the method of this section to find $ y_2 $ and the general solution of each of the following equations from the given solution $ y_1 $
	\begin{enumerate}
		\item $ y''+y=0, y_1=\sin x $
		\item $ y''-y=0, y_1=e^x $
		\item $ x^2y''+xy'-4y=0, y_1=x^2 $
	\end{enumerate}
\end{example}

\begin{proof}
	Consider 1) first, we have $ P(x)=0$. We will now find $ v $,
	\begin{align*}
		v&=\int \frac{1}{y_1^2} e^{- \int P\, dx}\, dx\\
		&= \int \frac{1}{\sin x^2} 1\, dx\\
		&= -\cot x
	\end{align*}
So now we can say $ y_2=v y_1= - \cos x  $. And the general solution is given by $ y(x) =c_1 \sin x + c_2 \cos x $\\
Consider 2) now, we have again $ P(x) =0$.
We will now find $ v $,
\begin{align*}
	v&=\int \frac{1}{e^{2x}}\, dx\\
	&=\frac{-e^{-2x}}{2}
\end{align*}
\textbf{Complete this up} $ \cdots $\\
Now consider 3), we have $ P(x)=\frac{1}{x} $. We will now find $ v $,
\begin{align*}
	v&=\int \frac{1}{y_1^2} e^{-\int P\, dx}\, dx\\
	&= \int \frac{1}{x^4} \frac{1}{x}\, dx\\
	&= \int \frac{1}{x^5}\, dx\\
	&= \frac{-1}{4x^4}
\end{align*}
Now we can get $ y_2=vy_1=\frac{-1}{4x^2} $. And get the general solution.
\end{proof}

\begin{example}
	Show that $ y_1=x $ is a solution to $ (1-x^2)y''+2xy'+2y=0 $. Find the general solution.
\end{example}
\begin{proof}
	First make the DE into standard form by dividing by $ 1-x^2 $. So we have $ P=\frac{-2x}{1-x^2} $. We will now compute $ v $,
	\begin{align*}
		v&=\int \frac{1}{x^2} e^{- \int \frac{-2x}{1-x^2}\, dx }\, dx\\
		&= \int \frac{1}{x^2} \frac{1}{x^2-1}\, dx\\
		&= \int \frac{1}{x^4-x^2}\, dx\\
		&= \frac{1}{x}+\frac{1}{2}\log(1-x)-\frac{1}{2} \log(x+1)
	\end{align*}
	Now we compute $ y_2=v y_1=1+\frac{1}{2} \log (1-x)-\frac{1}{2} x \log(1+x)$
\end{proof}


\subsection{Homogenous equation with constants}
\begin{theorem}
	If we have $ y''+P(x) y'+Q(x)y=0$ and $ P(x) ,Q(x)$ are constants. Consider the Auxiliary equation $ m^2+pm+q=0 $ and let the roots of the auxiliary equation be $ m_1, m_2 $.
	\begin{enumerate}
		\item If the roots are real and distinct ($ m_1\neq m_2 $), $ y=c_1e^{m_1x}+c_2e^{m_2x} $
		\item If the roots are real and repeated ($ m=m_1=m_2 $), $ y=c_1e^{mx}+c_2xe^{mx} $
		\item If the roots are complex ($ m=\alpha\pm \beta i $), $ y=e^{\alpha x}[c_1 \cos (\beta x)+c_2 \sin (\beta x)] $
	\end{enumerate}
\end{theorem}

\begin{example}
	$ y''+3y'-y=0 $
\end{example}
\begin{proof}
	Assume $ e^{mx} $ is a solution to the following differential equation, then we have the following,
	\begin{align*}
		y''+3y'-y&=0\\
		m^2 e^{mx}+ 3m e^{mx}-e^{mx}&=0
	\end{align*}
We can know solve for $ m $
	\begin{align*}
		m&= \frac{-3}{2}\pm\frac{\sqrt{13}}{2}
	\end{align*}
\textbf{Complete this $\dots$}
\end{proof}


\begin{example}
	Find the general solution for each of the following equations:
	\begin{enumerate}
		\item $ y''+y'-6y=0 $
		\item $ y''+3y'+y=0$
	\end{enumerate}
\end{example}
\begin{proof}
	Consider 1) first,
	use the auxiliary equation that appeared out of a hat then,
	\begin{align*}
		k^2+pk+q&=0\\
		k^2+k-6&=0\\
		k=-3, k&=2
	\end{align*}
	So now the general solution is $ y=c_1e^{-3x}+c_2e^{2x} $
	
\end{proof}

\begin{example}
	Solve $ 2y''+5y'-12y=0 $
\end{example}
\begin{proof}
	Use the auxiliary equation,
	\begin{align*}
		2m^2+5m-12=0
	\end{align*}
We get the values of $ m=3/4, -4 $. These are real and distinct roots. So the general solution is \[ y=c_1 e^{4x}+c_2e^{3/2x} \].
\end{proof}

\section{Method of undetermined coefficients (UDC)}
This is a method to solve non-homogenous linear DE of order 2.
\begin{align}
	y''+P(x)y'+Q(x)y&=R(x)\label{eq:nonhomo}\\
		y''+P(x)y'+Q(x)y&=0 \label{eq:asshomo}
\end{align}
The second equation is just the associated homogenous equation of the first.\\
Let $ y_g $ be the general solution of eq. \ref{eq:asshomo} be known as a complementary function (CF).\\
Let $ y_p $ be a particular solution of eq. \ref{eq:nonhomo} then it is known as particular integral.
\\
You begin by computing $ y_g $ then compute $ y_p $ depending on one of the three stupid cases and their subcases \footnote{Just have to memorize it unfortunately. You want a intuitive reason for doing it? Cry more lmao.}. Once you have the $ y_p $ you just add that (\textbf{without} a constant) to the $ y_g $ and you have the general solution for \ref{eq:nonhomo}
\subsection{Assumptions of UDC}
\begin{enumerate}
	\item $ P(x),Q(x) $ are constants
	\item $ R(x) $ is either exponential, sine or cosine or polynomial.
\end{enumerate}

\subsection{Case 1: Exponential}
\begin{theorem}\label{th:udcexp}


\begin{align}
	y''+py'+qy=e^{ax} \label{eq:udcexp}
\end{align}
\begin{enumerate}
	\item If $ a $ is not a root of AE, then $ y_p =Ae^{ax}$
	\item If $ a $ is a simple root (i.e. multiplicity 1) of AE, then $ y_p=Ax e^{ax} $
	\item If $ a $ is a double root of AE, then $ y_p=A x^2 e^{ax} $
\end{enumerate}
\end{theorem}
\begin{example}
	$ y''-4y'+4y=e^{2x} $
\end{example}
\begin{proof}
	Here $ a=2 $ and computing $ m $ we get a double root with $ m=2,2 $.
\end{proof}

\subsubsection{Subcase 1: a is not a root of AE}
If $ a $ is not a root of the $ AE $ then you take $ y_p=Ae^{ax}$. So you have the following,
\begin{align*}
	y_p&=Ae^{ax}\\
	y_p'&=Aae^{ax}\\
	y_p''&=Aa^2e^{ax}
\end{align*}
Plug this into the original DE then just append the $ y_p $ to $ y_g $ without constants for your general solution.\\

For subcase i, you can compute $ A $ directly as follows, \[ A=\frac{1}{a^2+pa+q} \]


\begin{example}
	$ y''-5y'+6y=e^{4x} $
\end{example}
\begin{proof}
	We get $ a=4 $ and this is not a root to the auxiliary equation ($ m=2,3 $).\\
	So now,
	\begin{align*}
		 y_p&=Ae^{4x}\\
		 y_p'&=4Ae^{4x}\\
		 y_p''&=16Ae^{4x}
	\end{align*}
Now plug this into the DE
\begin{align*}
	y''-5y'+6y&=e^{4x}\\
	16Ae^{4x}-5(4Ae^{4x})+6(Ae^{4x})&=e^{4x}
\end{align*}
So we get $ A=1/2 $.
\end{proof}

\subsubsection{Subcase 2: a is a simple root of AE}
If $ a $ is a simple root of the $ AE $ then you take $ y_p=Axe^{ax} $. So you have the following,
\begin{align*}
	y_p&=Axe^{ax}=Ae^{ax}(x)\\
	y_p'&=Ae^{ax}(ax+1)\\
	y_p''&=Ae^{ax}(a^2x+2a)
\end{align*}
Now just plug these values into the original differential equation. You will get $ y_p $ just append that to $ y_g $ and you have your solution.\\

For subcase ii, you can compute $ A $ directly as follows, \[ A=\frac{1}{2a+p} \]

\begin{example}
	Solve $ y''- 5y'+6y=3e^{2x}$
\end{example}
\begin{proof}
	The AHE is $ y''-5y'+6y=0 $, we get the auxiliary equation as $ m^2-5m+6=0 $ we get $ m=2,3$. This is a simple root.
	 \begin{align*}
		y_p&=Axe^{2x}\\
		y_p'&=Ae^{2x}+2Axe^{2x}\\
		y_p''&=4Ae^{2x}+4Axe^{2x}
	\end{align*}
Now plug this into the DE,
\begin{align*}
	y''- 5y'+6y&=3e^{2x}\\
	4Ae^{2x}+4Axe^{2x}-5(Ae^{2x}+2Axe^{2x})+6(Axe^{2x})&=3e^{2x}
\end{align*}
We get $ A=-3 $.
\end{proof}
\subsubsection{Subcase 3: a is a double root of AE}
If $ a $ is a double root of the $ AE $ then you take $ y_p=Ax^2e^{ax} $. So you have the following,
\begin{align*}
	y_p&=Ax^2e^{ax}=Ae^{ax}(x^2)\\
	y_p'&=Ae^{ax} (ax^2+2x)\\
	y_p''&=Ae^{ax}(a^2x^2+4ax+2)
\end{align*}


\begin{example}[Review]
	Find the general solution of $ y''+3y'-10y=6e^{4x} $
\end{example}
\begin{proof}
	Consider the auxiliary equation,
	\begin{align*}
		m^2+3m-10&=0
	\end{align*}
We have $ m=-5,2 $ real and distinct roots but $ a $ is not a root of $ AE $. The solution to the AE is $ y=c_1e^{2x}+c_2e^{-5x} $
So assume the general solution is given as,
\begin{align*}
	y_p&= Ae^{4x}\\
	y_p'&=4Ae^{4x}\\
	y_p''&=16Ae^{4x}
\end{align*}
Substitute this into the original equation to get $ A=\frac{1}{3} $.
So the general solution is given as $ y=c_1e^{2x}+c_2e^{-5x}+\frac{1}{3}e^{4x} $

\end{proof}

\subsection{Case 2: Trigonometric}
If $ y''+py'+qy=\sin kx $ or $ \cos kx $
\subsubsection{Subcase 1: If it is not a solution of AHE}
If we have $ \sin kx  $  or $ \cos kx  $ not being a solution to the AHE we take $ y_p=A \sin kx + B \cos kx $. Using this we get the following,
\begin{align*}
	y_p&=A \sin kx + B \cos kx\\
	y_p'&=k(A \cos kx - B \cos kx)\\
	y_p''&=-k^2(A \sin (kx)+B \cos kx)
\end{align*}
Plug this ugly mess into the original differential equation and you will get $ y_p $ append that without a constant to $ y_g $ for the general solution.

\subsubsection{Subcase 2: If it is a solution of AHE}
If we have $ \sin kx  $ or $ \cos kx  $ is a solution of the AHE then we take $ y_p=x(A \sin kx + B \cos kx) $. Using this we get the following,
\begin{align*}
	y_p&=x(A \sin kx + B \cos kx)\\
	y_p'&=A(\sin kx + kx \cos kx)+B(\cos kx - kx \sin kx)\\
	y_p''&=A(2k \cos kx - k^2 x \sin kx)+B(-x k^2 \cos kx - 2k \sin kx)
\end{align*}
\begin{example}
	Solve $ y''+y=\sin x $
\end{example}
\begin{proof}
	AHE is given $ y''+y=0 $ and AE $ m^2+1=0 $ so $ m=\pm i $.
	\begin{align*}
		y_g=c_1\cos x + c_2 \sin  x
	\end{align*}
Since $ \sin x $ is a solution of $ y_p $ let 
\begin{align*}
	 y_p&=x(A \sin x + B \cos x) \\
	 y_p'&=A \sin x + B \cos x + x (A \cos x - B \sin x)\\
	 y_p''&= A \cos x - B \sin x + (A \cos x - B \sin x)+ x(-A \sin x - B \cos x)
\end{align*}
Plug this into the diff eq.
\begin{align*}
	y''+y&=\sin x
\end{align*}
\begin{align*}
	A \cos x - B \sin x + (A \cos x - B \sin x)+ x(-A \sin x - B \cos x) + x(A \sin x + B \cos x)&= \sin x
\end{align*}
$ 2A=0, B=\frac{-1}{2} $ so we have $ y_p=-1/2 x \cos x $ and then $ y(x)=y_g+y_p=c_1\cos x + c_2 \sin x - 1/2 x \cos x  $
\end{proof}



\subsection{Case 3: Polynomial}
For $ y''+py'+qy=a_0+a_1x+\cdots+a_nx^n $
\subsubsection{Subcase 1: If $ \mathbf{q\neq0} $}
Take $ y_p $ as follows
\begin{align*}
	y_p&=A_0+A_1x+\cdots A_nx^n\\
	y_p'&=A_1+\cdots+n A_nx^{n-1}\\
	y_p''&=2A_2+\cdots+n(n-1)A_n^{n-2}
\end{align*}
\subsubsection{Subcase 2: If $ \mathbf{q=0, p \neq 0} $}
Take $ y_p $ as follows, the derivates are obvious,
$$y_p=x(A_0+A_1x+\cdots A_nx^n) $$
\subsubsection{Subcase 3: If $ \mathbf{q=0,p=0} $}
Take $ y_p $ as follows,
$$ y_p=x^2(A_0+A_x+\cdots + A_n x^n) $$


\begin{example}
	Find the general solution of $ y''-y'-2y=4x^2 $
\end{example}
\begin{proof}
	AHE $ y''-y'-2y=0 $ and AE $ m^2-m-2=0 $ so $ m=2,-1 $ so $ y_g=c_1e^{2x}+c_2e^{-x} $.\\
	Let 
	\begin{align*}
		y_p&=A_0+A_1x+A_2x^2 \\
		y_p'&=A_1+2A_2x\\
		y_p''&=2A_2
	\end{align*}
Substitute this into the original DE
\begin{align*}
	2A_2-A_1-2A_2x-2(A_0+A_1x+A_2x^2)&=4x^2
\end{align*}
We get $ A_0=-3,A_1=2,A_2=-2 $.
So the general solution is $ y(x)=c_1e^{2x}+c_2e^{-x}-3+2x-2x^2 $
\end{proof}

\begin{example}
	Find general solution of $ y''-2y'+5y=25x^2+12 $
\end{example}
\begin{proof}
	AHE $ y''-2y'+5y=0 $ and AE $ m^2-2m+5=0 $ we get $ m=1\pm2i $.So we have $ \alpha=1, \beta=2 $.\\
	So we get $ y_g=e^{x}(c_1 \cos 2x + c_2 \sin 2x) $.\\
	Let
	\begin{align*}
		y_p&=A_0+A_1x+A_2x^2\\
		y_p'&=A_1+2A_2x\\
		y_p''&=2A_2
	\end{align*}
Substitute this into the original DE
\begin{align*}
	2A_2-2(A_1+2A_2x)+5(A_0+A_1x+A_2x^2)&=25x^2+12
\end{align*}
Upon solving we get $ A_0=2, A_1=4, A_2=5 $. So now we have $ y(x)=y_g+y_p=e^{x}(c_1 \cos 2x + c_2 \sin 2x)+2+4x+5x^2 $
\end{proof}

\begin{example}[Review]
	Solve $y''-2y'=12x-10$
\end{example}
\begin{proof}
	Begin with AHE $ y''-2y'=0 $ then see HE $ m^2-2m=0 $ so we got $ m= 0, 2$ so me and my homies say $ y_g=1+e^{2x} $.\\
	Now since $ q=0 $ we say that
	\begin{align*}
		  y_p&=A_1x^2+A_0x\\
		  y_p'&=2A_1x+A_0\\
		  y_p''&=2A_1
	\end{align*}
Now plug this back into the original differential equation.
\begin{align*}
	2A_1-2(2A_1x+A_0)=12x-10
\end{align*}
Upon solving, a smart dog or a slow student would see that $ A_0=2, A_1=-3 $. This gives us $ y_p=-3x^2+2x $. And as such our final solution is indubitably,
\begin{align*}
	y(x)&=y_g+y_p\\
	&=c_1+c_2(e^{2x})+(-3x^2+2x)
\end{align*} 
\end{proof}

\section{Variation of Parameters (VOP) method}
Consider the non-homogenous equation
\[ y''+P(x)y'+Q(x)y=R(x) \]
Consider its AHE. Say it has two solutions $ y_1, y_2$. So that $ y_g= c_1y_1+c_2y_2$.\\
We will say that the required particular solution $ y_p $ is a combination of $ y_g $ as follows,\[ y_p=v_1 y_1+v_2y_2 \]
Now our job is just to get $ v_1,v_2 $ such that
\[ y_p(x)=v_1y_1+v_2y_2 \]
Now begin differentiating this by repeated use of the product rule \footnote{See why this is so boring yet?}.
\begin{align*}
	y_p'&=v_1y_1'+v_1'y_1+v_2y_2'+v_2'y_2\\
	y_p''&=v_1y_1''+2v_1'y_1'+v_1''y_1+v_2y_2''+2v_2'y_2'+v_2''y_2
\end{align*}
We also now choose $ v_1'y_1+v_2'y_2=0 $ to simplify the derivatives \footnote{\label{lmao}lmao},
Now with those terms we can restate the original differential equation as follows, 
\begin{align*}
	y_p'&=v_1y_1'+v_2y_2'\\
	y_p''&=v_1y_1''+v_1'y_1'+v_2y_2''+v_2'y_2'
\end{align*}
Substitute these into the original differential equation and you get,
\[ v_1(y_1''+Py_1'+Qy_1)+v_2(y_2''+Py_2'+Qy_2)+v_1'y_1'+v_2'y_2'=R(x) \]
Since $ y_1, y_2 $ are solutions to the AHE the terms in the parentheses vanish and we are left with
\[ v_1'y_1'+v_2'y_2'=R(x) \]
Recall from footnote \ref{lmao} that we now have a system of equations as follows,
\begin{align*}
	v_1'+v_2'y_2&=0\\
	v_1'y_1'+v_2'y_2'&=R(x)
\end{align*}
Now upon solving we get 
\begin{align*}
	v_1' &= \frac{-y_2R(x)}{W(y_1,y_2)} && &\text{and} && v_2' &= \frac{y_1R(x)}{W(y_1,y_2)}\\
	v_1 &= -\int \frac{y_2R(x)}{W(y_1,y_2)}\, dx && &\text{and} && v_2 &= \int \frac{R(x)y_1}{W(y_1,y_2)}\, dx
\end{align*}



\begin{example}
	Find a particular solution of $ y''+y=\csc x $
\end{example}
\begin{proof}
	Find the AHE $ y''+y=0 $ the solution for this is $$ y_p=c_1 \cos x + c_2 \sin x $$.
	We have $ R(x)=\csc x $ and we also know that $ W(y_1,y_2)= 1$\\
	So we got 
	\begin{align*}
		v_1=\int \frac{- \sin x (\csc x)}{1}\, dx = =x\\
		v_2= \int \frac{\cos x \sec x}{1}\, dx = \log (|\sin x|)
	\end{align*}
\end{proof}

\begin{example}
	Find a particular solution to $ y''-2y'+y=2x $
\end{example}
\begin{proof}
	Take the AHE $ y''-2y'+y=0 $
	and the HE $ m^2-2m+1=0 $ so we get $ y_p=c_1 e^{x}+c_2 x e^{x} $.\\
	The Wronskian is equal to $ W(y_1,y_2)= e^{2x}$.\\
	Now do some magic and get $$ v_2= \int \frac{2x e^{x}}{e^{2x}}\, dx = -2e^{-x}(x+1) $$
	and \[ v_1= \int \frac{-2x^2e^{x} }{e^{2x}}\, dx = 2e^{-x}(x^2+2x+2)\]
	So the final solution is given as follows
	\begin{align*}
		y&=v_1y_1+v_2y_2\\
		&=2e^{-x}(x^2+2x+2)(e^{x})+(-2e^{-x}(x+1))(xe^{x})\\
		&=2x+4
	\end{align*}
\end{proof}

\begin{example}
	Find a particular solution of $ y''-y'-6y=e^{x} $ with UDC then with VOP
\end{example}
\begin{proof}
	First do with UDC find the AHE and HE $ m^2-m-6=0 $ so we get $ m=-2, m=3 $ and the $ y_g = c_1e^{-2x}+c_2e^{3x}$. A is not a root so we go
	\begin{align*}
		y_p&=Ae^{x}\\
		y_p'&=Ae^{x}\\
		y_p''&=Ae^{x}
	\end{align*}
Plug thing into the original differential equation
\begin{align*}
	Ae^x-Ae^{x}-6Ae^x=e^x
\end{align*}
we got $ A=-1/6 $. So the general solution is $ y=e^{-2x}+e^{3x}-1/6e^{x} $.
\\
Now do VOP, from AHE and HE we get $ y_g=c_1e^{-2x}+c_2e^{3x} $.
We also now got the Wronskian as $ 5e^x $.
\textbf{Complete this later...}
\begin{align*}
	content...
\end{align*}
\end{proof}

\begin{example}
	Find a particular solution for each of the following equations
	\begin{enumerate}
		\item $ y''+4y=\tan 2x $
		\item $y''+2y'+y=e^{-x}\log x$
	\end{enumerate}
\end{example}


\begin{example}[Review]
	Solve $ y''-2y'-3y=64xe^{-x} $ using
	VOP and UDC.
\end{example}
\begin{proof}
	Solving with VOP first. Consider its AHE and its HE $ m^2-2m-3=0 $ we get $ m=-1, 3$ so $ y_g=c_1e^{-1}+c_2e^{3}$.\\
	The Wronskian is $ -4e^{2x} $.
	We now find $ v_1, v_2 $ as follows,
	\begin{align*}
		v_1 &= -\int \frac{y_2R(x)}{W(y_1,y_2)}\, dx\\
		&=-\int \frac{-e^{-x} 64 xe^{-x}}{-4e^{2x}}\, dx\\
		&=e^{-4x}(4x+1)
	\end{align*}
\begin{align*}
	 v_2 &= \int \frac{R(x)y_1}{W(y_1,y_2)}\, dx\\
	 &= \int \frac{e^{3x} 64xe^{-x}}{-4e^{2x}}\, dx\\
	 &=-8x^2 
\end{align*}
So the particular solution is given by,
\begin{align*}
	y_p=v_1y_1+v_2y_2=-e^{-4x}(4x+1)e^{3x}-8x^2(e^{-x})
\end{align*}
Then use this for the general solution.
\\
Now do it with UDC.\\
If we just had $ x $ we would use $ y_{p_1}=\alpha_0+\alpha_1x $. If we just had $ e^{-x} $ we would use $ y_{p_2}=\beta xe^{-x} $.
\\
We will use $ y_{p_1}y_{p_2}=(\alpha_0+\alpha_1x)(\beta x e^{-x})=\alpha_0 \beta x e^{-x}+ \alpha_1 \beta x^2 e^{-x}$ this simplifies to the following trial solution,
\[ y_p=A_0xe^{-x}+A_1x^2e^{-x} \]
Differentiate it and get the following,
\begin{align*}
	y_p&=A_0xe^{-x}+A_1x^2e^{-x}\\
	y_p'&=e^{-x}[A_0(1-x)-A_1(x-2)x]\\
	y_p''&=e^{-x}[A_0(x-2)+A_1(x^2-4x+2)]
\end{align*}
Plug this behemoth into the original differential equation,
\begin{align*}
	e^{-x}[A_0(x-2)+A_1(x^2-4x+2)]-2(e^{-x}[A_0(1-x)-A_1(x-2)x])-3(A_0xe^{-x}+A_1x^2e^{-x})=64xe^{-x}
\end{align*}
Simplify this 
$ A_0=-4,A_1=-8 $

so $ y_p=-4xe^{-x}-8x^2e^{-x} $
\end{proof}

\section{Higher order linear equations}
\begin{definition}
	content...
\end{definition}

\begin{example}
	Solve $ y^{(4)}-5y''+4y=0 $.
\end{example}
\begin{proof}
	The auxiliary equation is $ m^4-5m^2+4=0 $. Substitute $ m^2=p $ so we got $ p^2-5p+4=0 $ so $ p=4,1 $ and then $ m=\pm2,\pm1 $
	General solution is then $ y=c_1e^{2x}+c_2e^{-2x}+c_3 e^{x}+c_4e^{-x} $
\end{proof}

\begin{example}
	Solve $ y^{(4)}-8y''+16y=0 $
\end{example}
\begin{proof}
	Auxiliary equation is $ m^2-8m+16=0 $ substitute $ p^2=m $  and solve we get $ m=\pm2, \pm 2 $  so general solution is $ y=(c_1e^{2x}+c_2xe^{2x})+(c_2e^{-2x}+c_4xe^{-2x}) $
\end{proof}

\begin{proof}
	Solving $ y^{(4)}+2y'''+2y''-2y'+y=0 $
\end{proof}
\begin{proof}
	Take the auxiliary equation $ m^4-2m^3+2m^2-2m+1=0 $. We got the thing as $ m=1,1,\pm i $. So the general solution is given by 
	\[ y=(c_1e^{x}+c_2xe^{x})+(c_3 \cos x + c_4 \sin x)  \]
\end{proof}

\begin{example}
	Solve $ y'''+2y''-y'=3x^2-2x+1 $
\end{example}
\begin{proof}
	AHE and HE so auxiliary equation is $ m^3+2m^2-m=0 $ we get $ m=0,-1\pm \sqrt{2} $
\end{proof}

\begin{example}[Review-9]
	Find the general solution of $ y''-3y''+2y'=0 $.
\end{example}
\begin{proof}
	Consider the AHE and HE, $ m^3-3m^2+2m=0 $ we get the solutions as $ m=0,1,2 $. So the general solution is just given by
	\[ y=c_1+c_2e^{x}+c_3e^{2x} \]
\end{proof}

\section{Operator method}
\begin{itemize}
	\item Consider the differential equation $ y^{(n)}+a_1y^{(n-1)}+\cdots+a_{n-1}y'+a_ny=R(x) $
	\item Using the differential operator D, we can re write it as $ D^ny+a_1 D^{n-1}y+\cdots+a_{n-1}Dy+a_ny=R(x) $\\
	\item $ \implies p(D)y=R(x)$ where $ p(m) $ is called the axuliary polynomial and $ p(m)=(m-m_1)(m-m_2)\cdot(m-m_n) $ where $ m_i $ are roots of the auxiliary equation.
\end{itemize}

\begin{example}
	$ p(D)y=R(x) \implies y= \frac{1}{p(D)}R(x) $
\end{example}
\begin{proof}
	\begin{align*}
		y&=\frac{1}{D}R(x)\\
		&=\int R(x)\, dx
	\end{align*}
\end{proof}
\begin{example}
	$ (D-m)y=R(x) \implies y = \frac{1}{(D-m)}R(x)$
\end{example}
\begin{proof}
	\begin{align}
		y&=e^{mx} \int e^{-mx} R(x)\, dx\\
		\frac{1}{D-m}R(x)&=e^{mx}\int e^{-mx} R(x)\, dx \label{eq:operator}
	\end{align}
\end{proof}

\subsection{Successive integration}
\begin{align*}
	y&=\frac{1}{p(D)} R(x)=\frac{1}{[(D-m_1)(D-m_2)\cdots (D-m_n)]} R(x)\\
\end{align*}
Using \ref{eq:operator} successively we get the particular solution.

\begin{example}
	Find a particular solution of $ y''-3y'+2y=x e^{x} $
\end{example}
\begin{proof}
	\begin{align*}
		(D^2-3D+2) y&=xe^{x}\\
		(D-1)(D-2)y&=xe^{x}\\
		y&=\frac{1}{D-1}\left[\frac{1}{D-2} xe^{x}\right]
	\end{align*}
Here $ R(x) = xe^{x},  m_1=2 $. So we do
\begin{align*}
	\frac{1}{D-2}x e^{x}=e^{2x}\int e^{-2x} x e^{x}\, dx= -(1+x)e^{x}
\end{align*}
Step 2,
\begin{align*}
	y&= \frac{1}{D-1}[-(1+x)e^x]\\
	\frac{1}{D-1}[(1+x)e^x]&= e^x \int e^{-x} (1+x) e^x\, dx\\
	&=e^x\left( .... \right)
\end{align*}
\end{proof}

\subsection{Partial fraction decomposition}
\begin{align*}
	p(D)y=R(x)\implies y=\frac{1}{(D-m_1)(D-m_2)\cdots (D-m_n)}
\end{align*}
Use partial fractions... \textbf{complete this...}

\begin{example}
	Find a particular solution of $ y''-3y'+2y=xe^x $ using partial fraction decomposition.
\end{example}
\begin{proof}
	\begin{align*}
		(D^2-3D+2)y&=xe^x\\
		\implies y &= \frac{1}{(D-1)(D-2)}xe^x\\
		y&=\left(\frac{-1}{D-1}+\frac{1}{D-2}\right) xe^{x}
	\end{align*}
\textbf{i am not typing this out lmao}
\end{proof}

\begin{example}
	\begin{align*}
		\frac{1}{1-r}=1+r+r^2+r^3+\cdots, \text{ if} |r|<1\\
		\frac{1}{1+r}=1-r+r^2-r^3+\cdots, \text{ if} |r|<1
	\end{align*}
\end{example}

\begin{example}[Review-10]
	Find a particular solution of $ y''-4y=e^{2x} $ by using methods 1 and 2.
\end{example}
\begin{proof}
	Consider the eq as follows,
	\begin{align*}
		D^2y-4Dy=e^{2x}
	\end{align*}
\end{proof}


\subsection{Series expansion}
Used when $ R(x) $ is a polynomial.
\[ y=\frac{1}{p(D)} R(x) = [1+b_1D+b_2D^2+\cdots]R(x)\] higher order derivates vanish

\begin{example}
	Find particular solution of $ y'''-2y''+y'=x^4+2x+5 $
\end{example}
\begin{proof}
	\begin{align*}
		D^3-2D^2+Dy&=x^4+2x+5\\
		\implies y&= \frac{1}{1-(2D^2-D^3)} R(x)
		&=\left[1+(2D^2-D^3)+(2D^2-D^3)^2+(2D^2-D^3)^3+\cdots \right] R(x)\\
		&=\left[1+2D^2-D^+4D^4\right]
	\end{align*}
	\textbf{complete later...}
\end{proof}

\subsection{Exponential shift rule}
If $ R(x)=e^{kx}g(x) $
\begin{align*}
	y&= \frac{1}{p(D)}e^{kx}g(x)\\
	y&=e^{kx}\left[\frac{1}{p(D+k)}g(x)\right]
\end{align*}

\begin{example}
	Find a particular solution of $ y''-3y'+2y=xe^x $
\end{example}
\begin{proof}
	\begin{align*}
		(D^2-3D+2)y&=e^{x}x\\
		y&=\frac{1}{p(D)} e^x x\\
		&= e^x \frac{1}{p(D)} x\\
		&=e^x \frac{1}{p(D+1)}x
		\intertext{magick}
		&=-e^{x}\frac{1}{1-D}\frac{x^2}{2}\\
		&=\frac{-e^x}{2}[1+D+D^2+\cdots]
	\end{align*}
\end{proof}

\begin{example}
	Find a particular solution of $ y''-y=x^2e^{2x} $ using methods 1,2,4 then find general solution.
\end{example}
\begin{proof}
	\begin{align*}
		(D^2-1)y&=x^2e^{2x}\\
		y&=\frac{1}{D^2-1}x^{2}e^{2x}\\
		&= \frac{1}{2}\left(\frac{1}{(D-1)}-\frac{1}{(D+1)}\right) x^2e^{2x}\\
		&=\frac{1}{2}(-2-2D^2-2D^4+\cdots)x^2e^{2x}
	\end{align*}
\end{proof}

\begin{example}
	Find a particular solution of $ y''-y'+y=x^3-3x^2+1 $
\end{example}
\begin{proof}
	\begin{align*}
		(D^2-D+1)y&=x^3-3x^2+1\\
		y&=\frac{1}{D^2-D+1}x^3-3x^2+1\\
		&=(1+D-D^3-D^4+\cdots)x^3-3x^2+1\\
		&=(x^3-3x^2+1)+(3x^2-6x)-6\\
		&=x^3-6x-5
	\end{align*}
\end{proof}









\chapter{Linear systems of ordinary differential equations}
\begin{chapquote}{Someone I don't like}
	``I love differential equations. It is so fun. It has so many real life applications.''
\end{chapquote}

\section{Linear system of DE}
We are mainly concerned with first order linear system of ODEs.\\

Observe that the single $ n^{th} $ order equation \[ y^{(n)}=f(x,y,y') \]
Is in fact equivalent to the system 
\begin{align*}
	y_1&=y_2\\
	y_2'&=y_3\\
	\vdots &\\
	y_n'&=f(x,y_1,y_2,\dots,y_n)
\end{align*}
We will only see systems of two first order equations in two unknown functions of the following form,

\begin{align*}
	\dfrac{dx}{dt}&=F(t,x,y)\\
	\dfrac{dy}{dt}&=G(t,x,y)
\end{align*}


More specifically we have \textbf{linear} systems of the form,

\begin{definition}[Linear system of two ODE]\label{def:syslinear}
	\begin{align*}
		\dfrac{dx}{dt}&=a_1(t)+b_1(t)+f_1(t)\\
		\dfrac{dy}{dt}&=a_2(t)+b_2(t)+f_2(t)
	\end{align*} 
\end{definition}


\begin{definition}[Homogenous linear system of two ODE]\label{def:homosyslinear}
	\begin{align*}
		\dfrac{dx}{dt}&=a_1(t)+b_1(t)\\
		\dfrac{dy}{dt}&=a_2(t)+b_2(t)
	\end{align*} 
\end{definition}


We assume that $ a_i(t),b_i(t),f_i(t) $ for $ i=1,2 $ are continuous on some closed interval $ [a,b] $ on the $ t$-axis.\\
If $ f_i(t) $ are both identically zero, then the system is called homogenous else.
\\
A solution of \ref{def:syslinear} is of the following form,
\begin{align*}
	x&=x(t)\\
	y&=y(t)
\end{align*}



\section{Existence and uniqueness theorems}
\begin{theorem}
	If $ t_0 $ is any point of the interval $ [a,b] $ and if $ x_0 $ and $ y_0 $ are any numbers then def. \ref{def:syslinear} has one and only one solution 
	\begin{align*}
		x&=x(t)\\
		y&=y(t)
	\end{align*}
valid throughout $ [a,b] $, such that $ x(t_0)=x_0, y(t_0)=y_0 $.
\end{theorem}


\section{Homogenous linear system of ODE in two variables}
Now consider the system of linear homogenous equations (def \ref{def:homosyslinear}).

\begin{theorem}\label{th:linearsolution}
	If the homogenous system (def. \ref{def:homosyslinear}) has two solutions on the interval $ [a,b] $
	\[ \begin{cases}
		x=x_1(t)\\
		y=y_1(t)
	\end{cases} \text{and } \begin{cases}
		x=x_2(t)\\
		y=y_2(t)
	\end{cases}\]
	then we also have another solution of the form 
	\[ \begin{cases}
		x=c_1x_1(t)+c_2x_2(t)\\
		y=c_1y_1(t)+c_2y_2(t)
	\end{cases} \]
	for any constants $ c_1, c_2 $.
\end{theorem}


\section{Wronskian of homogenous linear system of ODE}
\begin{theorem}
	If $ W(t) $ is the Wronskian of two solutions of the homogenous system then $ W(t) $ is either identically zero or nowhere zero on $ [a,b] $.
\end{theorem}


\section{Linearly independent solutions}
god nose what fits here

\section{General solution of Homogenous linear system of ODE in two variables}

\begin{theorem}
	If the two solutions for the homogenous system \ref{def:homosyslinear} have a Wronskian that does not vanish on $ [a,b] $ then its linear combination of the solutions as described in theorem \ref{th:linearsolution} is the general solution of the homogenous system \ref{def:homosyslinear} on that interval.
\end{theorem}


\begin{theorem}
	If the two solutions of the homogeneous system are linearly independent then the linear combination \ref{th:linearsolution} is its general solution.
\end{theorem}


\section{Non-homogenous linear system in two variables}
\begin{theorem}
	If the two solutions for the homogenous system (Th. \ref{th:linearsolution}) are linearly independent on $ [a,b] $ and if 
	\[ \begin{cases}
		x&=x_p(t)\\
		y&=y_p(t)
	\end{cases} \]
	is any particular solution of the non-homogenous linear system of ODEs (def: \ref{def:syslinear}) then 
	\[ \begin{cases}
		x&=c_1x_1(t)+c_2x_2(t)+x_p(t)\\
		y&=c_1y_1(t)+c_2y_2(t)+y_p(t)
	\end{cases} \]
is the general solution of the non homogenous system \ref{def:syslinear}.
\end{theorem}

\section{Homogenous linear systems with constant coefficients}
Technically this section doesn't seem to be in her syllabus but I'm pretty sure its gonna be covered so I've done it anyway.\\
In this section we will examine the following system of linear ODEs,
\begin{align}\label{def:constanthomo}
	\begin{cases}
		\frac{dx}{dt}&=a_1x+b_1y\\
		\frac{dy}{dt}&=a_2x+b_2y
	\end{cases}
\end{align}
where $ a_i,b_i $ are constants.\\
We are concerned with the following auxiliary equation that is related to this system,
\begin{align}\label{def:sysaux}
	m^2-(a_1+b_2)m+(a_1b_2-a_2b_1)=0
\end{align}


\subsection{Distinct real roots}
If eq. \ref{def:sysaux} has distinct real roots $ m_1, m_2 $ then the general solution of eq. \ref{def:constanthomo} is given as,
\begin{align*}
	\begin{cases}
		x=c_1A_1e^{m_1t}+c_2A_2e^{m_2t}\\
		y=c_1B_1e^{m_1t}+c_2B_2e^{m_2t}
	\end{cases}
\end{align*}

\subsection{Equal real root}
If eq. \ref{def:sysaux} has equal real roots $m= m_1=m_2 $ then the general solution of eq. \ref{def:constanthomo} is given as,
\begin{align*}
	\begin{cases}
		x&=c_1Ae^{mt}+c_2(A_1+A_2t)e^{mt}\\
		y&=c_1Be^{mt}+c_2(B_1+B_2t)e^{mt}
	\end{cases}
\end{align*}

\subsection{Distinct complex roots}
If eq. \ref{def:sysaux} has distinct complex roots $a\pm ib $ then the general solution of eq. \ref{def:constanthomo} is given as,
\begin{align*}
	\begin{cases}
		x&=e^{at}[c_1(A_1 \cos bt - A_2 \sin bt)+c_2(A_1 \sin bt + A_2 \cos bt)]\\
		y&=e^{at}[c_1(B_1 \cos bt- B_2 \sin bt)+c_2(B_1 \sin bt + B_2 \cos bt)]
	\end{cases}
\end{align*}

\chapter{Partial differential equations}

\begin{chapquote}{Gauss}
	``Stop studying differential equations.''
\end{chapquote}


\begin{definition}[Partial derivatives]
	Partial derivates are defined as derivatives of a function of multiple variables when all but the variable of interest are held fixed during the differentiation.\\
	For a function $ f $ in $ n $ variables $ x_1,x_2,\dots, x_n $ we can define the $ m^{th} $ partial derivative as,
	\[ f_{x_m}=\frac{\partial f}{\partial x_m} = \lim_{h \rightarrow 0}\frac{f(x_1,\dots, x_m+h,\dots,x_n)-f(x_1,\dots,x_m, \dots,x_n)}{h}\]
	Partial derivatives can be taken with respect to multiple variables and are denoted as follows,
	\begin{align*}
		\frac{\partial^2 f}{\partial x^2}&=f_{xx}\\
		\frac{\partial^2 f}{\partial x \partial y}&=f_{xy}\\
		\frac{\partial^3 f}{\partial x^2 \partial y}&=f_{xxy}
	\end{align*}
\end{definition}
Differential equations that use partial derivates are called PDEs.


\section{Classification of Second order PDE}
Second order PDE are usually divided into three types.
\begin{definition}[General form of a second order PDE]
	\[ A u_{xx}+2B u_{xy}+C u_{yy}+Du_x+E u_y +F=0 \]
\end{definition}
Linear second order PDEs are classified according to the properties of the following $ 2\times 2 $ matrix,
\begin{align}\label{def:matrix}
	Z=\begin{bmatrix}
		A & B\\
		C & D
	\end{bmatrix}
\end{align}

\subsection{Elliptic PDE}
If $ Z $ (eq. \ref{def:matrix}) has determinant strictly greater than $ 0 $, it is called an elliptic PDE, i.e. $ \det Z > 0 $.

\subsection{Hyperbolic PDE}
If $ Z $ (eq. \ref{def:matrix}) has determinant strictly lesser than $ 0 $, it is called an hyperbolic PDE, i.e. $ \det Z < 0 $.

\subsection{Parabolic PDE}
If $ Z $ (eq. \ref{def:matrix}) has determinant equal to $ 0 $, it is called an parabolic PDE, i.e. $ \det Z =f 0 $.

\section{One dimensional wave equation}
\begin{definition}[One dimension wave equation]
	The one dimensional wave equation is given by,\[ a^2 \frac{\partial^2y}{\partial x^2}=\frac{\partial^2y}{\partial t^2} \]
	where $ a $ is a positive constant.
\end{definition}

\subsection{Vibration of an infinite string}
\subsection{Vibration of an semi-infinite string}
\subsection{Vibrating of a finite string}



\section{Laplace equation}
\begin{definition}[Laplacian]
	The Laplacian of a three dimensional function $ \phi $ is given as follows,
	\[ \Delta f= \frac{\partial^2 f}{\partial x^2}+\frac{\partial^2 f}{\partial y^2}+\frac{\partial^2 f}{\partial z^2} \]
	This is generalized for higher dimensions in the expected way.
\end{definition}

\begin{definition}[Laplace's equation]
	Laplace's equation is the following PDE
	\[ \Delta f = 0 \]
\end{definition}
\subsection{Green's function}
\begin{definition}[Green's function]
	content...
\end{definition}

\section{Heat conduction principle}
\begin{definition}[General heat equation]
	The temperature function $ w $ satisfies the following heat equation
	\[a^2 \Delta w = \frac{\partial w}{\partial t}\]
\end{definition}
\subsection{Infinite rod case}
\subsection{Finite rod case}


\backmatter

\chapter*{Appendix}
If you are seeing this, I forgot to do it.


\addcontentsline{toc}{Chapter}{Appendix}


\thispagestyle{empty}%If your book ends with the even numbered page, copy and paste it twice. With the odd numbered page, do it three times.
{\ }
\newpage

\end{document}

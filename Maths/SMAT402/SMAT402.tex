%This template consists of the minimum of a single book.
%Please do not think this template is mandatory and the format must be followed strictly.
%We expect the author adds what he needs.
\documentclass[oneside,11pt,pdftex]{book}%Remove draft when book editing is completed.
\usepackage{graphicx}
\usepackage{amsmath}
%\usepackage{fontawesome5}
\usepackage{booktabs}
\usepackage{amssymb}	
\usepackage{longtable}
\usepackage{amsthm}
\usepackage{multirow}
\usepackage[activate={true,nocompatibility},final,tracking=true,kerning=true,spacing=true,factor=1100,stretch=10,shrink=10]{microtype}
\usepackage[toc,page]{appendix}
\usepackage[nottoc]{tocbibind}
\numberwithin{equation}{section}
\graphicspath{ {./Images/} }
%\usepackage[raggedright]{titlesec}
\usepackage{placeins}
\usepackage{mathtools}


\usepackage{fancyhdr}
\usepackage{hyperref}
%Be careful when you use commands which align formulas.
%If aligned formulas range to two pages, the formulas should be divided into two environments.
%\makeatletter
%\AtBeginDocument{\let\mathaccentV\AMS@mathaccentV}
%\makeatother
%This is a patch for double bar.
%Activate it if \bar{\bar{a}} doesn't work.

\newskip\thskip
\thskip=0.5\baselineskip plus 0.2\baselineskip minus 0.2\baselineskip

\newdimen\dtest%Remove this when book editing is completed.
\settowidth{\dtest}{letters and symbols here}
\typeout{<<<\the\dtest>>>}

\newtheorem{theorem}{Theorem}[chapter]%Modify these declarations for your need.
\newtheorem{lemma}[theorem]{Lemma}
\newtheorem{corollary}[theorem]{Corollary}
\newtheorem{example}[theorem]{Example}
\newtheorem{definition}[theorem]{Definition}

\newtheorem{xca}[theorem]{Exercise}

\newtheorem{remark}[theorem]{Remark}

\numberwithin{section}{chapter}
\numberwithin{equation}{chapter}

\makeindex

\newcommand{\R}{\mathbb{R}}
\newcommand{\Q}{\mathbb{Q}}
\newcommand{\C}{\mathbb{C}}
\newcommand{\Z}{\mathbb{Z}}
\newcommand{\N}{\mathbb{N}}
\newcommand{\D}{\mathbb{D}}
\newcommand{\F}{\mathbb{F}}

\begin{document}
	
	
	\frontmatter

\thispagestyle{empty}
\begin{flushright}
{\LARGE \textbf{Bhoris Dhanjal}}%Input your name here.
\end{flushright}
\vfill
\begin{center}
{\fontsize{29.86truept}{0truept}\selectfont \textbf{Algebra IV}}%Input the book title here.
%Below is for a book with a subtitle.
%{\fontsize{29.86truept}{0truept}\selectfont \textbf{The Book Title}} \\
%\vspace{6.5truept}
%{\Large, \LARGE, etc. \textbf{The Subtitle}}
\end{center}
\vfill
\begin{flushleft}
{\LARGE \textbf{Lecture Notes}} \\
\hspace{-1.75truept}
{\large \textbf{for SMAT402}}
\end{flushleft}
\newpage

\tableofcontents


\mainmatter

\chapter{Groups and subgroups}
\section{Binary operation}
For a set $ V $ a function from $ f:V \times V \rightarrow V $ is called a binary function if the following properties hold.
\begin{enumerate}
	\item $ f $ is defined for all pairs of elements of $ V $.
	\item $ f $ is closed.
\end{enumerate}

\begin{example}
	$ G=\{1,2,3\} $, then $+ $ is not a binary operation as it is not closed under addition.
\end{example}

\begin{example}
	$ G=\{-1,0,1\} $, then $+ $ is a binary operation.
\end{example}

\begin{example}
	$ \N $, then both $ +, \times $ are binary operations.
\end{example}

\section{Group axioms}
A group is an ordered pair (G, *) where G is a non empty set and * is a binary operation on G satisfying the following axioms:\\
\begin{enumerate}
	\item \textbf{Closure:} $\forall$ a, b $\in$ G, a * b, is also in G
	\item \textbf{Associativity:} (a * b) * c = a * (b * c), $\forall$ a, b, c $\in$ G
	\item \textbf{Identity:} $\exists$ e $\in$ G, called an identity of G, s.t. $\forall$ a $\in$ G we have a * e = e * a = a
	\item \textbf{Inverse:} $\forall$ a $\in$ G $\exists \ a^{-1} \in$ G, called an inverse of a, s.t. a * $a^{-1}$ = $a^{-1}$ * a = e.
\end{enumerate}
\section{Examples of Groups}
\begin{example}
	$(\N,+)$ is not a group since it lacks additive identity.
\end{example}
\begin{example}
	$(\Z,+)$ is a group while $ (\Z, \times) $ is not a a group since it lacks multiplicative inverses.
\end{example}
\begin{example}
	$ (\Q, \times) $ is not a group since $ 0 $ doesn't have an inverse. However $ (\Q \setminus 0, \times) $ is a group.
\end{example}

\begin{example}
	$ n \Z=\{\dots,-2n, -n, 0, n, 2n \dots\} $ with addition are subgroups of $ (\Z,+) $.
\end{example}

\begin{example}
	$ S=\{1,-1,i,-1\} $, with multiplication is a cyclic group generated by $ i $. Exercise make a Cayley table.
\end{example}

\begin{example}
	$ M_{n\times n } (\R)$ for $ n \times n  $ matrices over $ \R $ forms a group under addition but not under matrix multiplication (because of lack of inverses).
\end{example}

\begin{example}
	$ GL_n(\R) $ (i.e. General linear group - matrices with positive determinant) forms a group under multiplication. 
\end{example}

\begin{example}
	$SL_n(\R)$ (i.e. Special linear group - matrices with det=1) forms a group under multiplication.
\end{example}

\subsection{Group of integers modulo n}
	
\begin{definition}[Congruence class]
	For $ n \in \Z $ define the congruence relation $ R $ as $aRb \iff n | (a-b)$. This is a equivalence relation.
\end{definition}
\begin{definition}[$ \Z/n\Z $ or $ \Z_n $]
	Let $ \Z/ n \Z $ be defined as the $ \{x \in \Z \mid x R n\} $.\\
	\[ \Z_n=\Z/n\Z = \{\overline{0}, \overline{1}, \dots, \overline{n-1} \} \]
	Addition $ \overline{a}+\overline{b}=\overline{a+b} $ and multiplication $ \overline{a}\cdot \overline{b}=\overline{ab} $.\\
	$ (\Z_n,+) $ forms a group for all $ n $, while $ (\Z_n*,\cdot) $ forms a group only when $ n $ is prime.
\end{definition}

\begin{theorem}
	$ \Z_n*  $ forms a group under multiplication iff $ n $ is prime.
\end{theorem}
\begin{proof}
	The proof is trivial.
\end{proof}

\subsection{Klein-4 group (Vierergruppe)}
Denoted by $ V_4 $ the Klein-4 group is the smallest non-cyclic group. It is abelian. It is a group with 4 elements such that the square of all elements is identity. And product of two distinct elements gives a distinct element.\\
The symmetry group of a rectangle is isomorphic to $ V_4 $.

\subsection{Symmetric group}
The symmetric group is the group whose elements are all the bijections from the set to itself. The order of the $ n^{th} $ Symmetric group ($ S_n $) is equal to $ n! $.

Two-Line to Cycle notation for permutations
\[ \left(\begin{matrix}
	1 & 2 & 3 & 4 & 5\\ 
	2 & 5 & 4 & 3 & 1
\end{matrix}\right)=(125)(34)=(34)(125)=(34)(512)=(15)(25)(34) \]
Here, the last form is a case of 2-cycle (transposition).\\
The parity of any permutation $\sigma$ is given by the parity of the number of its 2-cycles (transpositions). In the above example it is odd.

\subsection{Alternating group}
The group of all even permutations from $ S_n $ is called the alternating group $ A_n $.

\subsection{Dihedral group}
This is the group of symmetries of a regular polygon. Denoted by $ D_n, n\geq 3$.
\begin{itemize}
	\item Order of $ D_n=2n $.
	\item $ D_n=\{e,x,x^2,\dots,x^{n-1},y,yx,yx^2,\dots,yx^{n-1}\} $. Here we can interpret $ x $ as rotation by $ 2\pi/n $and $ y $ is reflection about vertical axis.
\end{itemize}

\section{Common properties of groups}

\subsection{Abelian group}
If a group is commutative it is called Abelian. 
\begin{itemize}
	\item If $ a^2 =e \forall a \in G$ then it is Abelian.
\end{itemize}

\subsection{Order of a group}
If there are a finite number of elements in a group then the group is called a finite group and the number of elements is called the group order of the group.
\subsection{Order of element}
The smallest natural number $ n $ such that $ a^n=e $ is called the order of a group element $ a $.

\chapter{Cyclic groups and cyclic subgroups}


\chapter{Lagrange's theorem and group homomorphisms}
\backmatter



\thispagestyle{empty}%If your book ends with the even numbered page, copy and paste it twice. With the odd numbered page, do it three times.
{\ }
\newpage

\end{document}
